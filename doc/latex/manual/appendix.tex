\chapter{Further model derivation}
\label{sec:appendix}

\section{Functional derivatives}
\label{sec:fderiv}

Let
%
\begin{equation}
\begin{split}
  F(u) &=
     F( u_1, \ldots, u_n, \nabla u_1, \ldots, \nabla u_n )
     \\
     &= \int_{\Omega} I(u_1, \ldots, u_n, \nabla u_1, \ldots,
       \nabla u_n ) d\Omega
\end{split}
\end{equation}
%
be a functional on a space of function tuples $u = (u_1,\ldots,u_n)$
where the $u_i$ are defined and periodic on the domain
$\Omega$.  By definition, the functional derivative of $F$ with
respect to $u_i$ at any point $\hat{u} = (\hat{u}_1,\ldots,\hat{u}_n)$
is the functional whose action is defined by
%
\begin{equation}
\begin{split}
 \left < \frac{\delta F}{\delta u_i}(\hat{u}), v \right > = {} &
    \left . \frac{d}{d \epsilon} \int_{\Omega}
    I(\hat{u}_1, \ldots, \hat{u}_i + \epsilon
    v, \ldots, \hat{u}_n, \right .
    \\ &
    \left . \nabla \hat{u}_1, \ldots, \nabla \hat{u}_i + \epsilon
    \nabla v, \ldots, \nabla \hat{u}_n) d\Omega \right |_{\epsilon=0}
    ~~{\rm for~all~}v.
\end{split}
\end{equation}
%
Hence, for all $v$,
%
\begin{eqnarray}
  \left < \frac{\delta F}{\delta
      u_i}(\hat{u}), v \right >
  &=& \int_{\Omega} \left ( \frac{\partial I}{\partial u_i}(\hat{u})v + \frac{\partial
    I}{\partial \nabla u_i}(\hat{u}) \cdot \nabla v \right ) d\Omega
      \\
  &=& \int_{\Omega} \left [ \frac{\partial I}{\partial u_i}(\hat{u})v +
      \nabla \cdot \left ( \frac{\partial
    I}{\partial u_i}(\hat{u}) v \right ) 
  - \left ( \nabla \cdot \frac{\partial
    I}{\partial u_i}(\hat{u}) \right ) v
  \right ] d\Omega \label{eq:divterm} \\
  &=& \left < \frac{\partial I}{\partial u_i}(\hat{u})
  - \nabla \cdot \frac{\partial
    I}{\partial u_i}(\hat{u}) , v
  \right > ,
\end{eqnarray}
%
where the second term in (\ref{eq:divterm}) vanishes due to the
divergence theorem and the assumed periodicity of $v$.
Thus,
%
\begin{equation}
  \frac{\delta F}{\delta u_i} = \frac{\partial I}{\partial u_i}
  - \nabla \cdot \frac{\partial
    I}{\partial \nabla u_i} .  \label{funcderiv}
\end{equation}

%%%%%%%%%%%%%%%%%%%%%%%%%%%%%%%%%%%%%%%%%%%%%%%%%%%%%%%%%%%%%%%%%%%%%%%%%%%%%%%%%
%
\section{Quaternions}
\label{sec:quaternions}

By analogy with complex numbers, a quaternion $q$ can be written as a
linear combination
%
\begin{equation}
  q = (a+ib+jc+kd)
\end{equation}
%
with 3 imaginary dimensions $i,j,k$.  In the following we will assume
that all of the quaternions are normalized, {\it i.e.}
$a^2+b^2+c^2+d^2=1$.  A rotation by an angle $\theta$ around an axis
of direction $(x,y,z)$ can be described by a quaternion
%
\begin{equation}
  q=\cos(\theta/2)+i(x*\sin(\theta/2))+j(y*\sin(\theta/2))+k(z*\sin(\theta/2))
\end{equation}
%
where $(x,y,z)$ is assumed to be normalized.  We denote by
%
\begin{equation}
  q^*=a-ib-jc-kd
\end{equation}
%
the conjugate of $q$.  Using the quaternion multiplication
rules, {\em i.e.}, $ij = k, jk = i, ki = j, ji = -k, kj = -i, ik = -j, i^2 =
j^2 = k^2 = -1$, the rotation between two quaternions, $q_1$ and
$q_2$, can be computed as either $q_1^* q_2$ or $q_2 q_1^*$, depending
on an arbitrary convention.  Due to the non-commutative nature of the
quaternion multiplication, these two are not equivalent, but are the
conjugate of each other.  They correspond to a rotation by the same
angle around an axis of opposite direction.

The formula for the misorientation angle between two unit-length
quaternions is given by
%
\begin{equation}
  |d_{21}| = 2\sin(\theta_{12}/4).
\label{eq:angle}
\end{equation}
%
where $|d_{21}|$ is the distance between $q_1$ and $q_2$ in the 4D
space, and $\theta_{12}$ is the angle between $q_1$ and $q_2$ (note
the factor of 2).  The mapping from unit quaternions to rotations is
2-to-1.  Multiplying a quaternion by an overall factor of -1 has no
physical effect.  Two quaternions on opposite sides of the hypersphere
are a distance $|d_{21}| = 2$ apart, which gives $\theta_{12} = 2\pi$,
which is no rotation at all.  To first order, (\ref{eq:angle}) gives
%
\begin{equation}
   \theta\simeq 2|d_{21}|
\end{equation}
%
and justifies using quaternion differences to approximate local
misorientation \cite{0295-5075-71-1-131}.

%%%%%%%%%%%%%%%%%%%%%%%%%%%%%%%%%%%%%%%%%%%%%%%%%%%%%%%%%%%%%%%%%%%%%%%%%%%%%%%%%
%
\section{Kim, Kim, Suzuki (KKS) model for a binary alloy}
\label{sec:kks}

Following the model proposed by Kim et al. \cite{PhysRevE.60.7186}, we
introduce auxiliary variables $c_l$, $c_\alpha$, and $c_\beta$ that
describe the composition in each of the two crystal phases, such that
%
\begin{equation}
  c = 
    ( 1 - h_\phi ) c_l +
    h_\phi \left[
    ( 1 - h_\eta ) c_\alpha + h_\eta c_\beta
    \right],
\label{eq:cmixappendix}
\end{equation}
%
which is the same as given in (\ref{eq:cmix}).  Then we define the
relation between the variables $(c_l,c_\alpha,c_\beta)$, and
$(c,\phi,\eta)$ by imposing the condition
%
\begin{equation}
  \left.\frac{\partial f^l}{\partial c_l}
    \right|_{c_l=c_l(x,t)} =
  \left.\frac{\partial f^\alpha}{\partial c_\alpha}
    \right|_{c_\alpha=c_\alpha(x,t)} =
  \left.\frac{\partial f^\beta}{\partial c_\beta}
    \right|_{c_\beta=c_\beta(x,t)} =
  \mu(x,t).
\label{eq:mu}
\end{equation}
%
This means that the chemical potential is equal in all phases at the
infinitesimal point $x$, and thus there would be no change in free
energy by exchanging an infinitesimal amount of species between phases
$l$, $\alpha$, and $\beta$.

We also define
%
\begin{equation}
  D_c(c,\phi,\eta,T) = 
  D_c^0(\phi,\eta,T)
  \left( \frac{\partial^2 f}{\partial c^2} \right)^{-1},
\label{eq:diffcoeff}
\end{equation}
%
with
%
$D_c^0(\phi,\eta,T)$ modeled as
%
\begin{equation}
\begin{split}
  D_c^0(\phi,\eta,T) = {} &
    \left( 1 - h_\phi \right)
    D^0_l exp \left( -Q^0_l / RT \right)
  \\ & +
    h_\phi \left[ \left( 1 - h_\eta \right)
    D^0_\alpha exp \left( -Q^0_\alpha / RT \right) +
    \left( 1 - h_\eta \right)
    D^0_\beta exp \left( -Q^0_\beta / RT \right)
    \right].
\end{split}
\end{equation}

Taking derivatives of (\ref{eq:cmixappendix}), we obtain
%
\begin{equation}
  \frac{\partial c}{\partial c} = 1 = 
  ( 1 - h_\phi ) \frac{\partial c_l}{\partial c} +
  h_\phi \left [
  ( 1 - h_\eta ) \frac{\partial c_\alpha}{\partial c} +
  h_\eta \frac{\partial c_\beta}{\partial c} \right].
\label{eq:dcdc}
\end{equation}
%
Since $c$ is independent of $\phi$, we also have
%
\begin{equation}
\begin{split}
  \frac{\partial c}{\partial \phi} = 0 = {}
  & - h'_\phi c_l + ( 1 - h_\phi ) \frac{\partial c_l}{\partial \phi}
  + h'_\phi \left [
  ( 1 - h_\eta ) c_\alpha + h_\eta c_\beta \right]
  \\ & + h_\phi \left [
  ( 1 - h_\eta ) \frac{\partial c_\alpha}{\partial \phi}
  + h_\eta \frac{\partial c_\beta}{\partial \phi} \right].
\label{eq:dcdphi}
\end{split}
\end{equation}
%
And likewise, since $c$ is independent of $\eta$, we have
\begin{equation}
\begin{split}
  \frac{\partial c}{\partial \eta} = 0 = {}
  & ( 1 - h_\phi ) \frac{\partial c_l}{\partial \eta}
  + h_\phi h'_\eta ( c_\beta - c_\alpha )
  \\ & + h_\phi \left [
  ( 1 - h_\eta ) \frac{\partial c_\alpha}{\partial \eta}
  + h_\eta \frac{\partial c_\beta}{\partial \eta} \right].
\label{eq:dcdeta}
\end{split}
\end{equation}
%
From (\ref{eq:fmix}) and (\ref{eq:mu}),
%
\begin{equation}
\begin{split}
  \frac{\partial f}{\partial c} & =
  ( 1 - h_\phi )
    \frac{\partial f^l}{\partial c_l}
    \frac{\partial c_l}{\partial c} +
  h_\phi \left[
    ( 1 - h_\eta )
    \frac{\partial f^\alpha}{\partial c_\alpha}
    \frac{\partial c_\alpha}{\partial c}
    + h_\eta
    \frac{\partial f^\beta}{\partial c_\beta}
    \frac{\partial c_\beta}{\partial c} \right]
  \\ & =
  \mu ( 1 - h_\phi )
    \frac{\partial c_l}{\partial c} +
  \mu h_\phi \left[
    ( 1 - h_\eta )
    \frac{\partial c_\alpha}{\partial c}
    + h_\eta \frac{\partial c_\beta}{\partial c} \right].
\end{split}
\end{equation}
%
Using (\ref{eq:dcdc}), this simplifies to
%
\begin{equation}
  \frac{\partial f}{\partial c} = \mu
\label{eq:dfdc}
\end{equation}
%
which is Eq.~28 of \cite{PhysRevE.60.7186}.  We note here that 
%
\begin{equation}
  \frac{\partial \mu}{\partial c} = 
  \frac{\partial^2 f}{\partial c^2},
\label{eq:dmudc}
\end{equation}
%
From (\ref{eq:fmix}) and (\ref{eq:mu}), we also have
%
\begin{equation}
\begin{split}
  \frac{\partial f}{\partial\phi} = {}
  & - h'_\phi f^l + \omega_\phi g'_\phi +
    \omega_\eta h'_\phi g_\eta
    + h'_\phi \left[
    ( 1 - h_\eta ) f^\alpha + h_\eta f^\beta \right]
  \\
  & + \mu \left\{ ( 1 - h_\phi )
    \frac{\partial c_l}{\partial \phi}
    + h_\phi \left[
    ( 1 - h_\eta ) \frac{\partial c_\alpha}{\partial \phi}
    + h_\eta \frac{\partial c_\beta}{\partial \phi}
    \right] \right\}.
\end{split}
\end{equation}
%
Then using (\ref{eq:dcdphi}) this becomes
%
\begin{equation}
\begin{split}
  \frac{\partial f}{\partial\phi} = {}
  & - h'_\phi f^l + \omega_\phi g'_\phi +
    \omega_\eta h'_\phi g_\eta
    + h'_\phi \left[
    ( 1 - h_\eta ) f^\alpha + h_\eta f^\beta \right]
  \\
  & + \mu h'_\phi c_l - \mu h'_\phi \left[
    ( 1 - h_\eta ) c_\alpha + h_\eta c_\beta \right].
\label{eq:dfdphi}
\end{split}
\end{equation}
%
Using (\ref{eq:dfdphi}) allows the derivation of the phase equation
(\ref{eq:phaseeom}).

Taking a further derivative with respect to $c$ and using
(\ref{eq:mu}) and (\ref{eq:dmudc}) gives
%
\begin{equation}
\begin{split}
  \frac{\partial^2 f}{\partial c \partial \phi} = {}
  & - \mu h'_\phi \frac{\partial c_l}{\partial c}
    + \mu h'_\phi \left[
    ( 1 - h_\eta ) \frac{\partial c_\alpha}{\partial c}
    + h_\eta \frac{\partial c_\beta}{\partial c} \right]
    + \mu h'_\phi \frac{\partial c_l}{\partial c}
    + h'_\phi c_l \frac{\partial^2 f}{\partial c^2}
  \\ & - \mu h'_\phi \left[
    ( 1 - h_\eta ) \frac{\partial c_\alpha}{\partial c}
    + h_\eta \frac{\partial c_\beta}{\partial c} \right]
    - h'_\phi \left[
    ( 1 - h_\eta ) c_\alpha + h_\eta c_\beta \right]
    \frac{\partial^2 f}{\partial c^2},
\end{split}
\end{equation}
%
which simplifies to
%
\begin{equation}
\begin{split}
  \frac{\partial^2 f}{\partial c \partial \phi} = {}
  & h'_\phi \left[ c_l - ( 1 - h_\eta ) c_\alpha
    - h_\eta c_\beta \right]
    \frac{\partial^2 f}{\partial c^2}.
\label{eq:dfdcdphi}
\end{split}
\end{equation}
%
Again from (\ref{eq:fmix}), we have
%
\begin{equation}
\begin{split}
  \frac{\partial f}{\partial\eta} = {}
  & \mu ( 1 - h_\phi ) \frac{\partial c_l}{\partial \eta}
    + \omega_\eta h_\phi g'_\eta
    + h_\phi h'_\eta ( f^\beta - f^\alpha )
  \\ & + \mu h_\phi \left[
    ( 1 - h_\eta ) \frac{\partial c_\alpha}{\partial \eta}
    + h_\eta \frac{\partial c_\beta}{\partial \eta} \right].
\end{split}
\end{equation}
%
Using (\ref{eq:dcdeta}) this becomes
%
\begin{equation}
\begin{split}
  \frac{\partial f}{\partial\eta} = {}
  & \omega_\eta h_\phi g'_\eta
    + h_\phi h'_\eta ( f^\beta - f^\alpha )
  \\ & - \mu h_\phi h'_\eta ( c_\beta - c_\alpha ).
\label{eq:dfdeta}
\end{split}
\end{equation}
%
Using (\ref{eq:dfdeta}) allows the derivation of the phase equation
(\ref{eq:etaeom}).

Taking a further derivative with respect to $c$ and using 
(\ref{eq:mu}) and (\ref{eq:dmudc}) gives
%
\begin{equation}
\begin{split}
  \frac{\partial^2 f}{\partial c \partial \eta} = {}
  & h_\phi h'_\eta ( c_\alpha - c_\beta ) 
    \frac{\partial^2 f}{\partial c^2}.
\label{eq:dfdcdeta}
\end{split}
\end{equation}
%
Since $f$ is a function of $c$, $\phi$, and $\eta$,
%
\begin{equation}
  \frac{\partial}{\partial x}\frac{\partial f}{\partial c} =
  \frac{\partial^2 f}{\partial c^2}\frac{\partial c}{\partial x} +
  \frac{\partial^2 f}{\partial c
  \partial \phi}\frac{\partial \phi}{\partial x} +
  \frac{\partial^2 f}{\partial c
  \partial \eta}\frac{\partial \eta}{\partial x}
\end{equation}
%
and we have
%
\begin{equation}
  \nabla\mu = \nabla\frac{\partial f}{\partial c} =
  \frac{\partial^2 f}{\partial c^2} \nabla c +
  \frac{\partial^2 f}{\partial c\partial\phi} \nabla \phi +
  \frac{\partial^2 f}{\partial c\partial\eta} \nabla \eta
\label{eq:gradmu}
\end{equation}
%
Using (\ref{eq:diffcoeff}), (\ref{eq:dfdcdphi}), (\ref{eq:dfdcdeta}),
and (\ref{eq:gradmu}) allows the derivation of the composition
equation (\ref{eq:ceomkks}).

To actually compute the right hand side of the phase and composition
equations (\ref{eq:phaseeom}), (\ref{eq:etaeom}), and
(\ref{eq:ceomkks}), we need to know $c_l(c,\phi,\eta)$,
$c_\alpha(c,\phi,\eta)$, and $c_\beta(c,\phi,\eta)$.  For that, we
need to know explicit forms of $f^l$, $f^\alpha$, and $f^\beta$, which
we give examples for in Sections~\ref{sec:hbsm} and \ref{sec:calphad}.

%%%%%%%%%%%%%%%%%%%%%%%%%%%%%%%%%%%%%%%%%%%%%%%%%%%%%%%%%%%%%%%%%%%%%%%%%%%%%%%%%
%
\subsection{Hu, Baskes, Stan, Mitchell (HBSM) model}
\label{sec:hbsm}

In \cite{HuBaskesStanMitchell07}, Hu {\em et al.} propose a
phase-field model for a binary alloy.  The two phases are $\alpha$
(body-centered cubic, bcc) and $\beta$ (face-centered cubic, fcc).  We
extend this with a third phase, $l$, or liquid.  The following
explicit forms for $f^l$, $f^\alpha$, and $f^\beta$ are proposed:
%
\begin{align}
  f^l(c_l,T) &=
    A_l \left[ c_l - c_l^\mathit{eq}(T) \right]^2,
    \label{eq:fl} \\
  f^\alpha(c_\alpha,T) &=
    A_\alpha \left[ c_\alpha - c_\alpha^\mathit{eq}(T) \right]^2,
    \label{eq:falpha} \\
  f^\beta(c_\beta,T) &= 
    A_\beta \left[ c_\beta - c_\beta^\mathit{eq}(T) \right]^2.
    \label{eq:fbeta}
\end{align}
%
Using (\ref{eq:mu}), we obtain
%
\begin{equation}
  A_l ( c_l - c_l^\mathit{eq} ) = 
  A_\alpha ( c_\alpha - c_\alpha^\mathit{eq} ) = 
  A_\beta ( c_\beta - c_\beta^\mathit{eq} ),
\end{equation}
%
and thus
%
\begin{align}
  c_l(c_l,T) &=
    c_l^\mathit{eq}(T) + \frac{A_\alpha}{A_l}
    \left[ c_\alpha - c_\alpha^\mathit{eq}(T) \right]
  \\
  c_\alpha(c_\beta,T) &=
    c_\alpha^\mathit{eq}(T) + \frac{A_l}{A_\alpha}
    \left[ c_l - c_l^\mathit{eq}(T) \right]
  \\
  c_\beta(c_\beta,T) &=
    c_\beta^\mathit{eq}(T) + \frac{A_l}{A_\beta}
    \left[ c_l - c_l^\mathit{eq}(T) \right]
\end{align}
%
From (\ref{eq:cmixappendix}), we then get
%
\begin{equation}
  c_l(c,\phi,\eta,T) =
  \frac{ c -
    h_\phi (1 - h_\eta) \left[ c_\alpha^\mathit{eq}(T) - 
      \frac{A_l}{A_\alpha} c_l^\mathit{eq}(T) \right] -
    h_\phi h_\eta \left[ c_\beta^\mathit{eq}(T) -
      \frac{A_l}{A_\beta} c_l^\mathit{eq}(T) \right] }
    { (1 - h_\phi) +
      h_\phi (1 - h_\eta) \frac{A_l}{A_\alpha} +
      h_\phi h_\eta \frac{A_l}{A_\beta} }.
\label{eq:cl}
\end{equation}
%
Likewise,
%
\begin{equation}
  c_\alpha(c,\phi,\eta,T) =
  \frac{ c -
    (1 - h_\phi) \left[ c_l^\mathit{eq}(T) - 
      \frac{A_\alpha}{A_l} c_\alpha^\mathit{eq}(T) \right] -
    h_\phi h_\eta \left[ c_\beta^\mathit{eq}(T) -
      \frac{A_\alpha}{A_\beta} c_\alpha^\mathit{eq}(T) \right] }
    { (1 - h_\phi) \frac{A_\alpha}{A_l} +
      h_\phi (1 - h_\eta) +
      h_\phi h_\eta \frac{A_\alpha}{A_\beta} }.
\label{eq:calpha}
\end{equation}
%
and
%
\begin{equation}
  c_\beta(c,\eta,T) =
  \frac{ c -
    (1 - h_\phi) \left[ c_l^\mathit{eq}(T) - 
      \frac{A_\beta}{A_l} c_\beta^\mathit{eq}(T) \right] -
    h_\phi ( 1 - h_\eta ) \left[ c_\alpha^\mathit{eq}(T) -
      \frac{A_\beta}{A_\alpha} c_\beta^\mathit{eq}(T) \right] }
    { (1 - h_\phi) \frac{A_\beta}{A_l} +
      h_\phi (1 - h_\eta) \frac{A_\beta}{A_\alpha} +
      h_\phi h_\eta }.
\label{eq:cbeta}
\end{equation}
%
These expressions for $c_l$, $c_\alpha$, and $c_\beta$ can then be
used to compute the $f_l$, $f_\alpha$, and $f_\beta$ from
(\ref{eq:fl}), (\ref{eq:falpha}), and (\ref{eq:fbeta}).  Then all of
these can be substituted into the phase and composition equations
(\ref{eq:phaseeom}), (\ref{eq:etaeom}), and (\ref{eq:ceomkks}) to have
right-hand sides that depend explicitly on $\phi$, $\eta$, and $c$.


%%%%%%%%%%%%%%%%%%%%%%%%%%%%%%%%%%%%%%%%%%%%%%%%%%%%%%%%%%%%%%%%%%%%%%%%%%%%%%%%%
%
\subsection{CALPHAD model}
\label{sec:calphad}

For the CALPHAD model of a binary alloy, in the $l$, $\alpha$, and
$\beta$ phases, we have free energies modeled by
%
\begin{equation}
  f^i(c_i,T) = f_0^i(c_i,T) + f_\mathit{mix}^\mathit{ideal}(c_i,T) +
    f^i_\mathit{mix}(c_i,T)
\end{equation}
%
where
%
\begin{equation}
  f_0^i(c_i,T) = c_i f_A^i(T) + (1 - c_i) f_B^i(T),
\end{equation}
%
\begin{equation}
  f_\mathit{mix}^\mathit{ideal}(c,T)  =
    R T [ c_i \ln(c_i ) + (1 - c_i) \ln(1 - c_i ) ],
\end{equation}
%
and
%
\begin{equation}
  f^i_\mathit{mix}(c_i,T) = c_i (1 - c_i)
    [ L_0^i(T) + L_1^i(T) (2 c_i - 1) +
      L_2^i(T) (2 c_i - 1)^2 ]
\end{equation}
%
for $i = l, \alpha$, and $\beta$.
Taking derivatives,
%
\begin{equation}
\begin{split}
  \frac{\partial f^i}{\partial c_i} = {} &
    f_A^i(T) - f_B^i(T) +
    RT \left[ \ln c_i - \ln ( 1 - c_i ) \right] +
    \\ &
    ( 1 - 2 c_i ) \left[ L_0^i + L_1^i ( 2 c_i - 1 ) +
      L_2^i ( 2 c_i - 1 )^2 \right] +
    \\ &
    c_i ( 1 - c_i ) \left[ 2 L_1^i +
      4 L_2^i ( 2 c_i - 1 ) \right]
\label{eq:dfdci0}
\end{split}
\end{equation}
%
Let
%
\begin{equation}
\begin{split}
  \xi_i = {} &
  ( 1 - 2 c_i ) \left[ L_0^i + L_1^i ( 2 c_i - 1 ) +
    L_2^i ( 2 c_i - 1 )^2 \right] +
  \\ &
    c_i ( 1 - c_i ) \left[ 2 L_1^i +
      4 L_2^i ( 2 c_i - 1 ) \right] +
    f_A^i(T) - f_B^i(T),
\end{split}
\end{equation}
%
so that (\ref{eq:dfdci0}) becomes
%
\begin{equation}
  \frac{\partial f^i}{\partial c_i} = 
    RT \ln \frac{c_i}{1 - c_i} + \xi_i
\label{eq:dfdci}
\end{equation}
%
Using (\ref{eq:mu}), we get
%
\begin{align}
  \frac{ c_\alpha (1 - c_l) }{ (1 - c_\alpha) c_l } &=
    \exp \left( \frac{ \xi_l - \xi_\alpha }{ RT } \right) \\
  \frac{ c_\beta (1 - c_l) }{ (1 - c_\beta) c_l } &=
    \exp \left( \frac{ \xi_l - \xi_\beta }{ RT } \right).
\end{align}
%
Together with (\ref{eq:cmix}), we have a nonlinear system of
3~equations and the 3~unknowns $c_l, c_\alpha$, and $c_\beta$
that can be solved by a suitable root-finding algorithm to enable
calculation of the free energies $f^l, f^\alpha$, and $f^\beta$.
    
%%%%%%%%%%%%%%%%%%%%%%%%%%%%%%%%%%%%%%%%%%%%%%%%%%%%%%%%%%%%%%%%%%%%%%%%%%%%%%%%%
%
\section{Elasticity}
\label{sec:elasticityappendix}

This section is essentially based on \cite{LaiRubinKrempl} and
\cite{HBSMZC07}.  A first attempt at adapting the approach for
quaternions is included.

\subsection{Strain tensor}

A typical material point $P$ at location $\vec X$ undergoes a
displacement $\vec u$ so that it arrives at position
%
\begin{equation}
  \vec x=\vec X+\vec u(\vec X).
\end{equation}
%
A neighboring point $Q$ at $\vec X+d\vec X$ arrives at
%
\begin{equation}
  \vec x+d\vec x=\vec X+d\vec X+\vec u(\vec X+d\vec X).
\end{equation}
%
Subtracting the two previous equations, one gets
%
\begin{equation}
  d\vec x=d\vec X+\vec u(\vec X+d\vec X)-\vec u(\vec X).
\end{equation}
%
Using the second order tensor $\nabla\vec u$, given in cartesian coordinates by
%
\begin{equation}
  \nabla\vec u=
  \left(
     \begin{array}{ccc}
       \frac{\partial u_1}{\partial X_1} & \frac{\partial u_1}{\partial X_2} & \frac{\partial u_1}{\partial X_3} \\
       \frac{\partial u_2}{\partial X_1} & \frac{\partial u_2}{\partial X_2} & \frac{\partial u_2}{\partial X_3} \\
       \frac{\partial u_3}{\partial X_1} & \frac{\partial u_3}{\partial X_2} & \frac{\partial u_3}{\partial X_3} \\
     \end{array}
  \right)
\end{equation}
%
we can write
%
\begin{equation}
  d\vec x=d\vec X+(\nabla\vec u) d\vec X=(I+\nabla\vec u)d\vec X.
\end{equation}
%
Now, looking at the length of $d\vec x$, we have
%
\begin{eqnarray}
  d\vec x^T d\vec x&=&d\vec X^T (I+\nabla\vec u)^T (I+\nabla\vec u) d\vec X \nonumber \\
  &=&d\vec X^T \left(
  I+(\nabla\vec u)^T+(\nabla\vec u)+(\nabla\vec u)^T(\nabla\vec u)
  \right) d\vec X).
\end{eqnarray}
%
Assuming small displacements, we have
%
\begin{eqnarray}
  d\vec x^T d\vec x&\approx &d\vec X^T \left(
  I+(\nabla\vec u)^T+(\nabla\vec u)
  \right) d\vec X \nonumber \\
  &=&
  d\vec X^T (I+2\varepsilon) d\vec X
\end{eqnarray}
%
where $\varepsilon$ is defined as the symmetric tensor
%
\begin{equation}
  \varepsilon=\frac{1}{2}\left[(\nabla\vec u)^T+(\nabla\vec u)\right]
\end{equation}
%
$\varepsilon$ is known as the {\bf strain tensor}.

Geometrical meaning: In the simple case of a material elongated along
a cartesian axis $\vec e_i$, $\varepsilon$ is $0$ except for
$\varepsilon_{ii}$ which is given by the increase in length per unit
original length.

%-----------------------------------------------------------------------

\subsection{Stress tensor}

Let $P$ be a typical material point, $S$ a plane passing through $P$
with a normal unit $\vec n$.  The stress vector at $P$ is defined by
%
\begin{equation}
  \vec t_{\vec n}=lim_{\triangle A\rightarrow 0}\frac{\triangle\vec F}{\triangle A}
\end{equation}
%
where $\triangle\vec F$ is a resultant force acting on a small area
$\triangle A$ of $S$ containing $P$.

The Cauchy's stress principle says that the stress vector is the same
at any point in the material living on the same side of a common
plane.  Further, using Newton's second law at static equilibrium, one
can show that
%
\begin{equation}\label{eq:stress_tensor}
  \vec t_{\vec n}=\sigma\vec n
\end{equation}
%
where $\sigma$ is the {\bf stress tensor}.  Also, using the moment of
the momentum equation for a small volume of a material, one can also
show that $\sigma$ is symmetric.

For example, for a solid under hydrostatic pressure $p$, we have
$\sigma=-p I$.

If we consider a small volume $\triangle x_1 \triangle x_2 \triangle
x_3$ at static equilibrium, using the fact that the sum of the forces
vanishes, we obtain
%
\begin{equation}
  \left[\frac{\partial\vec t_{\vec e_1}}{\partial x_1}+ \frac{\partial\vec t_{\vec e_2}}{\partial x_2} + \frac{\partial\vec t_{\vec e_3}}{\partial x_3} \right]=0
\end{equation}
%
where $\vec t_{\vec e_i}$ denotes the stress vector on the surface
orthogonal to $\vec e_i$.  Since $\vec t_{\vec e_i}=\sigma\vec
e_i=\sum_{j=1}^3 \sigma_{ji}\vec e_j$, we have

%
\begin{equation}\label{eq:equilibrium}
  \sum_{i=1}^3 \frac{\partial\sigma_{ji}}{\partial x_i}=0, \;\; j=1,2,3
\end{equation}
%
This can also be written as
%
\begin{equation}
  div(\sigma)\equiv \sum_{i,j=1}^3\frac{\partial\sigma_{ij}}{\partial x_j}\vec e_j=0.
\end{equation}
%

%-----------------------------------------------------------------------

\subsection{Hooke's law}

Hooke's law expresses a general linear relation between the applied
stress and the deformation
%
\begin{equation}\label{eq:hooke}
  \sigma_{ij}=\sum_{k,l=1}^3 C_{ijkl}\varepsilon_{kl}.
\end{equation}
%
where $C$ is a fourth-order tensor, known as {\bf elasticity tensor}.
There are 81 coefficients ($3^4$) in Eq.(\ref{eq:hooke}).  However
using the symmetry of $\varepsilon$ and the symmetry of $\sigma$, one
can choose
%
\begin{equation}
  C_{ijkl}=C_{ijlk}
\end{equation}
%
and
%
\begin{equation}
  C_{ijkl}=C_{jikl}
\end{equation}
%
and reduce the number of independent coefficients to 36.

One can further reduce this number to 21 elastic constants by assuming
that one can introduce a {\bf strain energy function} $U$ such that
%
\begin{equation}\label{eq:dUdepsilon}
  \sigma_{ij}=\frac{\partial U}{\partial \varepsilon_{ij}}.
\end{equation}
%
One can show that Eq.(\ref{eq:dUdepsilon}) implies
%
\begin{equation}
  C_{ijkl}=C_{klij}.
\end{equation}
%
In summary, one can write the general Hooke's law as
%
\begin{equation}\label{eq:general_stress_strain}
  \left(\begin{array}{c}
    \sigma_{11} \\
    \sigma_{22} \\
    \sigma_{33} \\
    \sigma_{23} \\
    \sigma_{31} \\
    \sigma_{12}
  \end{array}\right)
  =
  \left(
    \begin{array}{cccccc}
      C_{1111} & C_{1122} & C_{1133} & C_{1123} & C_{1113} & C_{1112} \\
      C_{1122} & C_{2222} & C_{2233} & C_{2223} & C_{2213} & C_{1222} \\
      C_{1133} & C_{2233} & C_{3333} & C_{2333} & C_{1333} & C_{1233} \\
      C_{1123} & C_{2223} & C_{2333} & C_{2323} & C_{2313} & C_{1223} \\
      C_{1113} & C_{2213} & C_{1333} & C_{2313} & C_{1313} & C_{1213} \\
      C_{1112} & C_{1222} & C_{1233} & C_{1223} & C_{1213} & C_{1212} \\
    \end{array}
  \right)
  \left(\begin{array}{c}
    \varepsilon_{11} \\
    \varepsilon_{22} \\
    \varepsilon_{33} \\
    2\varepsilon_{23} \\
    2\varepsilon_{31} \\
    2\varepsilon_{12}
  \end{array}\right).
\end{equation}
%
This leads to the introduction of a contracted form for the tensor $C$
with 2 indices
%
\begin{equation}\label{eq:stress_matrix}
  \left(\begin{array}{c}
    \sigma_{11} \\
    \sigma_{22} \\
    \sigma_{33} \\
    \sigma_{23} \\
    \sigma_{31} \\
    \sigma_{12}
  \end{array}\right)
  =
  \left(
    \begin{array}{cccccc}
      C_{11} & C_{12} & C_{13} & C_{14} & C_{15} & C_{16} \\
      C_{12} & C_{22} & C_{23} & C_{24} & C_{25} & C_{26} \\
      C_{13} & C_{23} & C_{33} & C_{34} & C_{35} & C_{36} \\
      C_{14} & C_{24} & C_{34} & C_{44} & C_{45} & C_{46} \\
      C_{15} & C_{25} & C_{35} & C_{45} & C_{55} & C_{56} \\
      C_{16} & C_{26} & C_{36} & C_{46} & C_{56} & C_{66} \\
    \end{array}
  \right)
  \left(\begin{array}{c}
    \varepsilon_{11} \\
    \varepsilon_{22} \\
    \varepsilon_{33} \\
    2\varepsilon_{23} \\
    2\varepsilon_{31} \\
    2\varepsilon_{12}
  \end{array}\right).
\end{equation}
%
We write the same equation in the matrix-vector form
%
\begin{equation}
\label{eq:sigmavector}
\vec\sigma=\bar C\vec\varepsilon.
\end{equation}
%
Note that the factors 2 in the definition of $\vec\varepsilon$ gives
the factors 2 needed for the double sum in case of different indexes
in (\ref{eq:hooke}).  The matrix C is known as the {\bf stiffness
matrix} for the elastic solid.

Depending on crystal symmetry, the number of independent coefficients
can be further reduced.  For a cubic symmetry, there are only 3
independent coefficients.
%
\begin{equation}\label{eq:Cijkl_cubic}
  C_{ijkl}=C_{12}\delta_{ij}\delta_{kl}+C_{44}(\delta_{ik}\delta_{jl}+\delta_{il}\delta_{jk})
  + (C_{11}-C_{12}-2C_{44})\sum_{r=1}^3 \delta_{ir}\delta_{jr}\delta_{kr}\delta_{lr}
\end{equation}
%
\begin{equation}
\left(
  \begin{array}{cccccc}
    C_{11} & C_{12} & C_{12} & 0 & 0 & 0 \\
    . & C_{11} & C_{12} & 0 & 0 & 0 \\
    . & . & C_{11} & 0 & 0 & 0 \\
    . & . & . & C_{44} & 0 & 0 \\
    . & . & . & . & C_{44} & 0 \\
    . & . & . & . & . & C_{44} \\
  \end{array}
\right)
\end{equation}
%
In the isotropic case, we have
%
\begin{equation}\label{eq:Cijkl_isotropic}
  C_{ijkl}=K\delta_{ij}\delta_{kl}
    +\mu(\delta_{ik}\delta_{jl}+\delta_{il}\delta_{jk}
    -\frac{2}{3}\delta_{ij}\delta_{kl})
\end{equation}
%
$K$ is the bulk modulus and $\mu$ the shear modulus.  In that case, we
also have $C_{11}=K+4\mu/3$, $C_{12}=K-2\mu/3$, and
$C_{44}=\mu=(C_{11}-C_{12})/2$.

In 2D, Eq.(\ref{eq:general_stress_strain}) reduces to
%
\begin{equation}\label{eq:general_stress_strain2D}
  \left(\begin{array}{c}
    \sigma_{11} \\
    \sigma_{22} \\
    \sigma_{12}
  \end{array}\right)
  =
  \left(
    \begin{array}{cccccc}
      C_{1111} & C_{1122} & C_{1112} \\
      C_{1122} & C_{2222} & C_{1222} \\
      C_{1112} & C_{1222} & C_{1212} \\
    \end{array}
  \right)
  \left(\begin{array}{c}
    \varepsilon_{11} \\
    \varepsilon_{22} \\
    2\varepsilon_{12}
  \end{array}\right).
\end{equation}
%
or in the contracted form
%
\begin{equation}\label{eq:general_stress_strain2D_b}
  \left(\begin{array}{c}
    \sigma_{11} \\
    \sigma_{22} \\
    \sigma_{12}
  \end{array}\right)
  =
  \left(
    \begin{array}{cccccc}
      C_{11} & C_{12} & C_{13} \\
      C_{12} & C_{22} & C_{23} \\
      C_{13} & C_{23} & C_{33} \\
    \end{array}
  \right)
  \left(\begin{array}{c}
    \varepsilon_{11} \\
    \varepsilon_{22} \\
    2\varepsilon_{12}
  \end{array}\right).
\end{equation}

%-----------------------------------------------------------------------

\subsection{Strain energy}

\begin{equation}\label{eq:strain_energy}
  U=\frac{1}{2}\sum_{i,j,k,l=1}^3 C_{ijkl}\varepsilon_{ij}\varepsilon_{kl}
  =\frac{1}{2}\sum_{k,l=1}^3 \sigma_{kl}\varepsilon_{kl}.
\end{equation}
%
Using matrix notations from (\ref{eq:sigmavector}), it can also be
written as
%
\begin{equation}
  U=\frac{1}{2}\vec\varepsilon^T \bar C\vec\varepsilon.
\end{equation}
%
Again, note that the factors 2 in the definition of $\vec\varepsilon$
gives the factors 2 needed for the double sum of different indexes in
(\ref{eq:strain_energy}).

%-----------------------------------------------------------------------

