\chapter{The phase field model}
\label{sec:model}

On a spatial domain $\Omega$, we begin by introducing the total energy functional
%
\begin{equation}
  F_0 \equiv F_0(\phi,c,{\bf q},T)
    \equiv \int_{\Omega} I_0(\phi,c,{\bf q},T) d\Omega,
\label{eq:fdef}
\end{equation}
%
where the energy density, $I_0$, is
%
\begin{equation}
  I_0(\phi,c,{\bf q},T) \equiv
    \frac{\epsilon_{\phi}^2}{2} |\nabla \phi |^2 + f(\phi,c,T) +
    \frac{\epsilon_{\bf q}^2}{2} |\nabla {\bf q} |^2 +
    D_{\bf q}(\phi,T) |\nabla {\bf q} |,
\label{eq:idef}
\end{equation}
%
where $\phi$ is a structural order parameter, $c$ is the composition
of a particular species (here we assume a binary material so that $1-c$ is
the composition of the second species), and ${\bf q} \equiv
(q_1,q_2,q_3,q_4)$ is a quaternion describing local crystallographic
orientation (see Appendix \ref{sec:quaternions}), with the
normalization
%
\begin{equation}
  \sum_{i=1}^4 q_i^2 = 1.
\label{eq:constraint}
\end{equation}
%
$T$ is the temperature and is assumed to be uniform across the
computational domain $\Omega$ in our current model.  The first term of the
energy density (\ref{eq:idef}) yields an energy contribution at
interfaces between the phases identified by $\phi$, with
$\epsilon_{\phi}$ controlling the interface width.  We further assume
that at every point in space we have the possibility of coexistence of
a two-phase mixture.  We denote these two phases S ($\phi=1$) and L
($\phi=0$) by reference to the classical problem of solid-liquid
mixture, but they can be used to represent various other general two-phase problems.  Following the model proposed by Kim et
al. \cite{PhysRevE.60.7186}, we introduce auxiliary variables $c_S$
and $c_L$ that describe the composition in each of the two phases,
such that
%
\begin{equation}
  c=h(\phi)c_S+[1-h(\phi)]c_L,
\label{eq:cmix}
\end{equation}
%
where $h$ is some interpolating monotonic polynomial satisfying
$h(0)=0$ and $h(1)=1$.  For the examples in this paper, we use
%
\begin{equation}
  h(\phi) = \phi^3 \left( 10 - 15 \phi + 6 \phi^2 \right).
\end{equation}
%
The free energy density, $f(c,\phi,T)$, in the
second term of (\ref{eq:idef}) is defined by the mixture rule
%
\begin{equation}
  f(c,\phi,T)=h(\phi)f^S(c_S,T)+[1-h(\phi)]f^L(c_L,T)+\omega g(\phi),
\label{eq:fmix}
\end{equation}
%
where $f^S$ and $f^L$ are the free energy densities of the S and L
phases, and $g(\phi)$ is a double well potential
%
\begin{equation}
  g(\phi)=16 \phi^2(1-\phi)^2.
\label{eq:dwell}
\end{equation}
%
Following \cite{0295-5075-71-1-131}, the third term of (\ref{eq:idef})
is an orientational free energy where
%
\begin{equation}
  | \nabla {\bf q} | = \left ( \sum_{i=1}^4 (\nabla q_i)^2 \right )^{1/2}
\end{equation}
%
and
%
\begin{equation}
  D_{\bf q} (\phi,T) \equiv 2HTp(\phi),
\label{eq:dqdef}
\end{equation}
%
with $H$ a constant, $T$ the local temperature, and $p$ another
interpolating monotonic polynomial satisfying $p(0)=0$ and $p(1)=1$.
This polynomial should have a positive derivative at $\phi=1$. We use
%
\begin{equation}
  p(\phi) = \phi^2.
\end{equation}
%
We note that we have adopted the opposite convention compared to that
used in \cite{0295-5075-71-1-131}, {\em i.e.}, we have replaced the
polynomial $p$ by $1-p$.

The final term of (\ref{eq:idef}) involves $|\nabla {\bf q} |^2$ but
is not scaled with $\phi$.  This is different than other published
models using a similar orientation term.  We have found that
preventing this term from approaching zero as $\phi$ goes to zero is
necessary to prevent physically unmeaningful values of orientation in
low-order regions from affecting the growth of ordered grains by
producing a smooth quaternion solution in such regions.  The addition
of noise terms or the use of a very high relative quaternion mobility
also have an effect on this issue, though its exploration is beyond
the scope of this paper.

We seek to minimize (\ref{eq:fdef}) subject to (\ref{eq:constraint}).
As in \cite{0295-5075-71-1-131}, we use a Lagrange multiplier to
convert the constrained minimization problem to an unconstrained one.
This is accomplished by defining the new functional
%
\begin{equation}
  F(\phi,c,{\bf q},T,\lambda) \equiv \int_{\Omega} I(\phi,c,{\bf
    q},T,\lambda) d\Omega,
\label{eq:ftildedef}
\end{equation}
%
where
%
\begin{equation}
  I(\phi,c,{\bf q},T,\lambda) = I_0(\phi,c,{\bf q},T) +
    \lambda \left ( \sum_i q_i^2 - 1 \right )
\label{eq:integrand}
\end{equation}
%
is the original energy density (\ref{eq:idef}) augmented by the
Lagrange multiplier term.  The extrema of (\ref{eq:fdef}) then
correspond to the critical points of $F$, which satisfy the
Euler-Lagrange equations
%
\begin{equation}
  \frac{\partial F}{\partial \phi} = \frac{\partial
    F}{\partial c} = \frac{\partial F}{\partial
    q_i} = \frac{\partial F}{\partial \lambda} = 0.
\label{eq:critical}
\end{equation}
%
For any particular initial condition $(\phi,c,{\bf q},\lambda)$, it is
likely that one or more of the quantities in (\ref{eq:critical}) is
non-zero.  The essential idea behind the time evolution phase-field
approach is to use the non-zero quantities as source terms in a
time-dependent relaxation to a steady state satisfying
(\ref{eq:critical}).  In particular, for the phase and orientation
variables, we postulate the Allen-Cahn equations \cite{AllenCahn79}
%
\begin{eqnarray}
  \dot{\phi} &=& - M_{\phi} \frac{\delta
    F}{\delta \phi} ,
\label{eq:phirelax} \\
  \dot{q}_i &=& - M_{\bf q} \frac{\delta
    F}{\delta q_i} , \hspace{2em} i=1,\ldots,4,
\label{eq:qrelax}
\end{eqnarray}
%
where dots denote temporal derivatives, $M_{\phi}$ and $M_q$ are
mobility coefficients which may be non-constant, and the functional
derivatives are computed as described in Appendix \ref{sec:fderiv}.
For the composition equation, we postulate the governing
equation to be \cite{PhysRevA.45.7424}
%
\begin{equation}
  \dot{c} = M_c \nabla \cdot D_c(c,\phi,T) \nabla
  \frac{\partial F}{\partial c},
\label{eq:crelax}
\end{equation}
%
where $D_c$ is the diffusivity. In
contrast to (\ref{eq:phirelax}) and (\ref{eq:qrelax}), this equation
evolves the composition $c$ conservatively.  We next evaluate the
right-hand sides of (\ref{eq:phirelax}), (\ref{eq:qrelax}) and
(\ref{eq:crelax}) individually.

%%%%%%%%%%%%%%%%%%%%%%%%%%%%%%%%%%%%%%%%%%%%%%%%%%%%%%%%%%%%%%%%%%%%%%%%%%%%%%%%%
%
\section{Phase equation}

From (\ref{eq:fmix}), (\ref{eq:dqdef}) and (\ref{eq:integrand}), we
have
%
\begin{equation}
  \frac{\partial I}{\partial \nabla \phi} = \epsilon_{\phi}^2 \nabla \phi
\end{equation}
%
and
%
\begin{eqnarray}
  \frac{\partial I}{\partial \phi} =
    & & {} - h'(\phi)\left(f^L(c_L,T)-f^S(c_S,T) - \mu (c_L - c_S) \right)
    \nonumber \\
    & & {} + \omega g'(\phi) +  2HTp'(\phi) | \nabla {\bf q} |.
\end{eqnarray}
%
Here we used the KKS model and the definition of the chemical
potential $\mu$ from (\ref{eq:mu})
%
%\begin{equation}
%  \mu(x,t)=\left.\frac{\partial f^S}{\partial c_S}\right|_{c_S=c_S(x,t)} =
%  \left.\frac{\partial f^L}{\partial c_L}\right|_{c_L=c_L(x,t)}
%\end{equation}
%
and (\ref{eq:dfdphi2}) from Appendix \ref{sec:kks}.

Hence, from (\ref{eq:phirelax}) and the functional derivative formula (\ref{funcderiv})
given in Appendix \ref{sec:fderiv}, we have
%
\begin{eqnarray}
  \dot{\phi} = & & M_\phi\left\{ \epsilon_{\phi}^2 \nabla^2 \phi
    + h'(\phi)\left(f^L(c_L,T)-f^S(c_S,T) -\mu (c_L - c_S) \right)
    \right .
    \nonumber \\
    & & {} - \left . \omega g'(\phi)
    -  2HTp'(\phi) | \nabla {\bf q} | \right\}.
\label{eq:phaseeom}
\end{eqnarray}

In the particular case of a single species material, we have
$c=c_L=c_S=1$ and $f^L$ and $f^S$ are functions of the temperature $T$ only.

In general, $M_\phi$ may be a function of $\phi$ itself, as well as
its derivatives, and possibly other model variables.  For the examples
in this paper, $M_\phi$ will be set to a constant value.

%%%%%%%%%%%%%%%%%%%%%%%%%%%%%%%%%%%%%%%%%%%%%%%%%%%%%%%%%%%%%%%%%%%%%%%%%%%%%%%%%
%
\section{Orientation equation}
\label{sec:orientationmodel}

From (\ref{eq:integrand}), we have, for $i=1,\ldots,4$,
%
\begin{equation}
  \frac{\partial I}{\partial \nabla q_i} =
    \left ( \epsilon_{\bf q} + \frac{D_{\bf q}(\phi) }{| \nabla {\bf q} |} \right )
    \nabla q_i
\label{eq:didnq}
\end{equation}
%
and
%
\begin{equation}
  \frac{\partial I}{\partial q_i} = 2 \lambda q_i.
\label{eq:didq}
\end{equation}
%
Hence, from (\ref{eq:qrelax}) and (\ref{eq:constraint}) we obtain
%
\begin{eqnarray}
  \dot{q}_i - M_{\bf q} (\phi) \left \{ \nabla \cdot \left ( \epsilon_{\bf q} +
    \frac{D_{\bf q}(\phi)}{| \nabla {\bf q} |} \right ) \nabla q_i - 2 \lambda
    q_i \right \} &=& 0, \hspace{1em} i=1,\ldots,4
\label{eq:dae1} \\
  \sum_i q_i^2 - 1 &=& 0.
\label{eq:daeconstraint}
\end{eqnarray}
%
We allow the mobility $M_{\bf q}$ to depend on $\phi$ in order to
limit rotation in the ordered phase, further detailed below.
Equations (\ref{eq:dae1})--(\ref{eq:daeconstraint}) comprise a
semi-explicit, differential-algebraic system of index two (see, {\em
e.g.}, \cite{BCP89} for more information about the theory and
numerical solution of differential-algebraic systems).  Although an
algorithm for the integration of such a system could be pursued, it is
generally the case that the numerical integration of
differential-algebraic systems of index two or higher is facilitated
by first reducing the index of the system.  In the present case, this
is accomplished by replacing (\ref{eq:daeconstraint}) by its time
derivative and substituting (\ref{eq:dae1}), giving
%
\begin{equation}
  0 = 2 \sum_i q_i \dot{q}_i =
    2 \sum_i q_i \left [ \nabla \cdot \left ( \epsilon_{\bf q} +
    \frac{D_{\bf q}(\phi) }{| \nabla {\bf q} |} \right )
    \nabla q_i - 2 \lambda q_i \right ],
\label{eq:difcon}
\end{equation}
%
which yields an explicit expression for $\lambda$.  Upon elimination
of $\lambda$ in (\ref{eq:dae1}), we obtain the ordinary differential
equations
%
\begin{eqnarray}
  \dot{q}_i = M_{\bf q}(\phi) & & \left \{
    \nabla \cdot \left (
    \epsilon_{\bf q} +  \frac{D_{\bf q}(\phi)}{| \nabla {\bf q} |} \right )
    \nabla q_i \right . \nonumber \\
    & & \quad \left . - \frac{q_i}{\sum_\ell q_\ell^2} \sum_k q_k \nabla \cdot
    \left ( \epsilon_q +  \frac{D_{\bf q}(\phi)}{ | \nabla {\bf q} |} \right )
    \nabla q_k \right\},
\label{eq:qeom}
\end{eqnarray}
%
which was originally formulated in \cite{0295-5075-71-1-131} (which
omits the $\epsilon_{\bf q}$ term and includes a noise term).

For any vector $v \equiv (v_1,v_2,v_3,v_4)$, let $\Pi({\bf q})v$
denote the orthogonal projection (with respect to the usual Euclidean
inner product) of $v$ onto ${\bf q}$.  The system (\ref{eq:qeom}) can
then be written as
%
\begin{equation}
  \dot{\bf q} =
    M_{\bf q}(\phi) \left (I - \Pi({\bf q}) \right )
    \nabla \cdot \left ( \epsilon_{\bf q} + 
    \frac{D_{\bf q}(\phi)}{ |\nabla
    {\bf q}|} \right ) \nabla {\bf q}.
\label{eq:qeom2}
\end{equation}
%
In this form, it is clear that solutions of (\ref{eq:qeom2}) also satisfy
the invariant
%
\begin{equation}
  {\bf q} \cdot \dot{\bf q} = 0,
\label{eq:invariant}
\end{equation}
%
which is just a restatement of the differentiated constraint
(\ref{eq:difcon}).  Solutions of (\ref{eq:qeom2}) with an initial
condition on the constraint surface (\ref{eq:daeconstraint}) therefore
remain on the surface at all times.  Differentiation of the constraint
(\ref{eq:daeconstraint}) has thus replaced the problem of integrating
an index two differential-algebraic system with the equivalent problem
of enforcing the invariant (\ref{eq:invariant}) in the integration of
an ordinary differential equation.

As with the mobility for the phase equation, the orientation mobility,
$M_q$, may be a general function.  It is common to use a functional
form that reduces $M_q$ as the phase variable, $\phi$, goes to 1 in
order to slow or prevent the wholesale rotation of ordered grains.
For the examples in this paper, we will set
%
\begin{equation}
  M_q(\phi) = M_q^{\rm min} + m(\phi) ( M_q^{\rm max} - M_q^{\rm min} ),
\label{eq:qmobility}
\end{equation}
%
where $M_q^{\rm max}$ varies with the problem and $M_q^{\rm min} =
10^{-6}$, {\em i.e.},
very near zero, with $m(\phi)$ an interpolating monotonic polynomial
satisfying $m(0) = 1$ and $m(1) = 0$.  We use
%
\begin{equation}
  m(\phi) = 1 - \phi^3 \left( 10 - 15 \phi + 6 \phi^2 \right)
\end{equation}

%%%%%%%%%%%%%%%%%%%%%%%%%%%%%%%%%%%%%%%%%%%%%%%%%%%%%%%%%%%%%%%%%%%%%%%%%%%%%%%%%
%
\section{Composition equation}

The particular form of the composition equation depends upon the
relationship between the variables $(c_S,c_L)$, and $(c,\phi)$ in
(\ref{eq:cmix}).  In Appendix \ref{sec:kks}, the details for the Kim,
Kim and Suzuki (KKS) model are briefly recalled, which results in the
following equation of motion
%
\begin{equation}
  \frac{\partial c}{\partial t} =
    M_c \nabla \cdot D^0_c(\phi,T)\nabla c +
    \nabla \cdot D^0_c(\phi,T) h'(\phi) (c_L - c_S) \nabla \phi.
\label{eq:ceomkks}
\end{equation}
%
To actually compute the right-hand side of (\ref{eq:ceomkks}), we need
to know $c_S(c,\phi)$ and $c_L(c,\phi)$. For that, we need to know the
exact form of $f^S$ and $f^L$.  A specific example for the Hu, Baskes,
Stan and Mitchell (HBSM) model of a binary alloy
\cite{HuBaskesStanMitchell07} is given in Appendix \ref{sec:hbsm}.
