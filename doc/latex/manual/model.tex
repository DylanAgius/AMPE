\chapter{The phase field model}
\label{sec:model}

On a spatial domain $\Omega$, we begin by introducing the total energy
functional
%
\begin{equation}
  F_0 \equiv F_0(\phi,\eta,c,{\bf q},T)
    \equiv \int_{\Omega} I_0(\phi,\eta,c,{\bf q},T) d\Omega,
\label{eq:fdef}
\end{equation}
%
where the energy density, $I_0$, is
%
\begin{equation}
  I_0(\phi,\eta,c,{\bf q},T) \equiv
    \frac{\epsilon_{\phi}^2}{2} |\nabla \phi |^2 +
    \frac{\epsilon_{\eta}^2}{2} |\nabla \eta |^2 +
    f(\phi,\eta,c,T) +
    \frac{\epsilon_{\bf q}^2}{2} |\nabla {\bf q} |^2 +
    D_{\bf q}(\phi,T) |\nabla {\bf q} |,
\label{eq:idef}
\end{equation}
%
where $\phi$ is a structural order parameter indicating matter state
({\em i.e.}, liquid or solid), $\eta$ is a structural order parameter
indicating crystal phase ({\em i.e.}, bcc or fcc), $c$ is the
composition of a particular species (here we assume a binary material
so that $1-c$ is the composition of the second species), and ${\bf q}
\equiv (q_1,q_2,q_3,q_4)$ is a quaternion describing local
crystallographic orientation (see Appendix \ref{sec:quaternions}),
with the normalization
%
\begin{equation}
  \sum_{i=1}^4 q_i^2 = 1.
\label{eq:constraint}
\end{equation}
%
$T$ is the temperature and is assumed to be uniform across the
computational domain $\Omega$ in our current model.  The first term of
the energy density (\ref{eq:idef}) yields an energy contribution at
interfaces between the phases identified by $\phi$, with
$\epsilon_{\phi}$ controlling the interface width.  We further assume
that at every point in space we have the possibility of coexistence of
a three-phase mixture.  We denote these phases $l$ ($\phi=0$),
$\alpha$ ($\phi=1,\eta=0$), and $\beta$ ($\phi=1,\eta=1)$, by
reference to the problem of a mixture of liquid and two solid phases,
but they can be used to represent various other general three-phase
problems.

Define interpolating polynomials
%
\begin{align}
  h_\phi & \equiv h_\phi(\phi) \\
  h_\eta & \equiv h_\eta(\eta),
\end{align}
%
satisfying $h_\eta(0)=0$, $h_\eta(1)=1$, $h_\phi(0)=0$, and
$h_\phi(1)=1$.  We typically choose forms of
%
\begin{equation}
  h(x) = x^3 \left( 10 - 15 x + 6 x^2 \right).
\end{equation}
%

Also define ``double well'' potentials
%
\begin{align}
  g_\phi & \equiv g_\phi(\phi) \\
  g_\eta & \equiv g_\eta(\eta),
\end{align}
%
for which we typically choose forms
%
\begin{equation}
  h(x) = 16 x^2 ( 1 - x )^2.
\end{equation}
%

The free energy density, $f(\phi,\eta,c,T)$, in the
second term of (\ref{eq:idef}) is defined by the mixture rule
%
\begin{equation}
\begin{split}
  f(\phi,\eta,c,T) = {}
    & ( 1 - h_\phi ) f^l(c_l,T)
    \\
    & + h_\phi \left[
      ( 1 - h_\eta ) f^\alpha(c_\alpha,T)
      + h_\eta f^\beta(c_\beta,T) \right]
    \\
    & + \omega_\phi g_\phi +
      \omega_\eta h_\phi g_\eta ,
\label{eq:fmix}
\end{split}
\end{equation}
%
where $f^l$, $f^\alpha$, and $f^\beta$ are the free energy densities
of the $l$, $\alpha$, and $\beta$ phases, and the newly introduced
$c_l$, $c_\alpha$, and $c_\beta$ are related to $c$ by
%
\begin{equation}
  c = 
    ( 1 - h_\phi ) c_l +
    h_\phi \left[
    ( 1 - h_\eta ) c_\alpha + h_\eta c_\beta
    \right],
\label{eq:cmix}
\end{equation}
%
following the model proposed by Kim et al. \cite{PhysRevE.60.7186}.

Following \cite{0295-5075-71-1-131}, the third term of (\ref{eq:idef})
is an orientational free energy where
%
\begin{equation}
  | \nabla {\bf q} | = \left ( \sum_{i=1}^4 (\nabla q_i)^2 \right )^{1/2}
\end{equation}
%
and
%
\begin{equation}
  D_{\bf q} (\phi,T) \equiv 2HTp(\phi),
\label{eq:dqdef}
\end{equation}
%
with $H$ a constant, $T$ the local temperature, and $p$ another
interpolating monotonic polynomial satisfying $p(0)=0$ and $p(1)=1$.
This polynomial should have a positive derivative at $\phi=1$. We use
%
\begin{equation}
  p(\phi) = \phi^2.
\end{equation}
%
We note that we have adopted the opposite convention compared to that
used in \cite{0295-5075-71-1-131}, {\em i.e.}, we have replaced the
polynomial $p$ by $1-p$.

The final term of (\ref{eq:idef}) involves $|\nabla {\bf q} |^2$ but
is not scaled with $\phi$.  This is different than other published
models using a similar orientation term.  We have found that
preventing this term from approaching zero as $\phi$ goes to zero is
necessary to prevent physically unmeaningful values of orientation in
low-order regions from affecting the growth of ordered grains by
producing a smooth quaternion solution in such regions.  The addition
of noise terms or the use of a very high relative quaternion mobility
also have an effect on this issue.

We seek to minimize (\ref{eq:fdef}) subject to (\ref{eq:constraint}).
As in \cite{0295-5075-71-1-131}, we use a Lagrange multiplier to
convert the constrained minimization problem to an unconstrained one.
This is accomplished by defining the new functional
%
\begin{equation}
  F(\phi,\eta,c,{\bf q},T,\lambda) \equiv
    \int_{\Omega} I(\phi,\eta,c,{\bf q},T,\lambda) d\Omega,
\label{eq:ftildedef}
\end{equation}
%
where
%
\begin{equation}
  I(\phi,\eta,c,{\bf q},T,\lambda) = I_0(\phi,\eta,c,{\bf q},T) +
    \lambda \left ( \sum_i q_i^2 - 1 \right )
\label{eq:integrand}
\end{equation}
%
is the original energy density (\ref{eq:idef}) augmented by the
Lagrange multiplier term.  The extrema of (\ref{eq:fdef}) then
correspond to the critical points of $F$, which satisfy the
Euler-Lagrange equations
%
\begin{equation}
  \frac{\partial F}{\partial \phi} =
  \frac{\partial F}{\partial \eta} =
  \frac{\partial F}{\partial c} =
  \frac{\partial F}{\partial q_i} =
  \frac{\partial F}{\partial \lambda} =
  0.
\label{eq:critical}
\end{equation}
%
For any particular initial condition $(\phi,\eta,c,{\bf q},\lambda)$,
it is likely that one or more of the quantities in (\ref{eq:critical})
is non-zero.  The essential idea behind the time evolution phase-field
approach is to use the non-zero quantities as source terms in a
time-dependent relaxation to a steady state satisfying
(\ref{eq:critical}).  In particular, for the phase and orientation
variables, we postulate the Allen-Cahn equations \cite{AllenCahn79}
%
\begin{align}
  \dot{\phi} & = - M_{\phi} \frac{\delta
    F}{\delta \phi} ,
\label{eq:phirelax} \\
  \dot{\eta} & = - M_{\eta} \frac{\delta
    F}{\delta \eta} ,
\label{eq:etarelax} \\
  \dot{q}_i & = - M_{\bf q} \frac{\delta
    F}{\delta q_i} , \hspace{2em} i=1,\ldots,4,
\label{eq:qrelax}
\end{align}
%
where dots denote temporal derivatives, $M_{\phi}$, $M_{\eta}$, and
$M_q$ are mobility coefficients which may be non-constant, and the
functional derivatives are computed as described in
Appendix~\ref{sec:fderiv}.  For the composition equation, we postulate
the governing equation to be \cite{PhysRevA.45.7424}
%
\begin{equation}
  \dot{c} = \nabla \cdot D_c(c,\eta,T) \nabla
  \frac{\partial F}{\partial c},
\label{eq:crelax}
\end{equation}
%
where $D_c$ is the diffusivity. In contrast to (\ref{eq:phirelax})
through (\ref{eq:qrelax}), this equation evolves the composition $c$
conservatively.  We next evaluate the right-hand sides of
(\ref{eq:phirelax}), (\ref{eq:qrelax}) and (\ref{eq:crelax})
individually.

%%%%%%%%%%%%%%%%%%%%%%%%%%%%%%%%%%%%%%%%%%%%%%%%%%%%%%%%%%%%%%%%%%%%%%%
%
\section{Phase equations}

From (\ref{eq:fmix}), (\ref{eq:dqdef}), (\ref{eq:integrand}), and
(\ref{eq:dfdphi}), we have
%
\begin{equation}
  \frac{\partial I}{\partial \nabla \phi} = \epsilon_{\phi}^2 \nabla \phi
\end{equation}
%
and
%
\begin{equation}
\begin{split}
  \frac{\partial I}{\partial \phi} = {}
  & - h'_\phi \left[
    f^l(c_l,T) - ( 1 - h_\eta ) f^\alpha(c_\alpha,T)
    - h_\eta f^\beta(c_\beta,T) \right]
  \\ & + \mu h'_\phi \left[
    c_l - ( 1 - h_\eta ) c_\alpha
    - h_\eta c_\beta \right]
  \\ & + \omega_\phi g'_\phi + \omega_\eta h'_\phi g_\eta
    + 2 H T p'(\phi) | \nabla {\bf q} |.
\end{split}
\end{equation}
%
Hence, from (\ref{eq:phirelax}) and the functional derivative formula
(\ref{funcderiv}) given in Appendix \ref{sec:fderiv}, we have
%
\begin{equation}
\begin{split}
  \dot{\phi} =
  M_\phi \Big(
  & \epsilon_{\phi}^2 \nabla^2 \phi 
    - \omega_\phi g'_\phi - \omega_\eta h'_\phi g_\eta
    - 2 H T p'(\phi) | \nabla {\bf q} |
  \\ &
    + h'_\phi \left[
    f^l(c_l,T) - ( 1 - h_\eta ) f^\alpha(c_\alpha,T)
    - h_\eta f^\beta(c_\beta,T) \right]
  \\ & - \mu h'_\phi \left[
    c_l - ( 1 - h_\eta ) c_\alpha
    - h_\eta c_\beta \right]
  \Big).
\label{eq:phaseeom}
\end{split}
\end{equation}
%
In the particular case of a single species material, we have
$c=c_l=c_\alpha=c_\beta=1$ and $f^l$, $f^\alpha$, and $f^\beta$ are
functions of the temperature $T$ only.
%
In general, $M_\phi$ may be a function of $\phi$ itself, as well as
its derivatives, and possibly other model variables.  For the examples
in this paper, $M_\phi$ will be set to a constant value.

Likewise for the other phase variable $\eta$, using (\ref{eq:dfdeta}),
we have
%
\begin{equation}
  \frac{\partial I}{\partial \nabla \eta} = \epsilon_{\eta}^2 \nabla \eta
\end{equation}
%
and
%
\begin{equation}
\begin{split}
  \frac{\partial I}{\partial \eta} = {}
  & - h_\phi h'_\eta \left[
    f^\alpha(c_\alpha,T) - f^\beta(c_\beta,T)
    - \mu ( c_\alpha - c_\beta ) \right]
  \\ & + \omega_\eta h_\phi g_\eta'.
\end{split}
\end{equation}
%
Hence, from (\ref{eq:etarelax}) and the functional derivative formula
(\ref{funcderiv}) given in Appendix \ref{sec:fderiv}, we have
%
\begin{equation}
\begin{split}
  \dot{\eta} =
  M_\eta \Big(
  & \epsilon_{\eta}^2 \nabla^2 \eta
    - \omega_\eta h_\phi g_\eta'
  \\ &
    + h_\phi h'_\eta \left[
    f^\alpha(c_\alpha,T) - f^\beta(c_\beta,T)
    - \mu ( c_\alpha - c_\beta ) \right]
  \Big).
\label{eq:etaeom}
\end{split}
\end{equation}
%
In general, $M_\eta$ may be a function of $\eta$ itself, as well as
its derivatives, and possibly other model variables.  For the examples
in this paper, $M_\eta$ will be set to a constant value.

%%%%%%%%%%%%%%%%%%%%%%%%%%%%%%%%%%%%%%%%%%%%%%%%%%%%%%%%%%%%%%%%%%%%%%%
%
\section{Orientation equation}
\label{sec:orientationmodel}

From (\ref{eq:integrand}), we have, for $i=1,\ldots,4$,
%
\begin{equation}
  \frac{\partial I}{\partial \nabla q_i} =
    \left ( \epsilon_{\bf q} + \frac{D_{\bf q}(\phi) }{| \nabla {\bf q} |} \right )
    \nabla q_i
\label{eq:didnq}
\end{equation}
%
and
%
\begin{equation}
  \frac{\partial I}{\partial q_i} = 2 \lambda q_i.
\label{eq:didq}
\end{equation}
%
Hence, from (\ref{eq:qrelax}) and (\ref{eq:constraint}) we obtain
%
\begin{align}
  \dot{q}_i - M_{\bf q} (\phi) \left \{ \nabla \cdot \left ( \epsilon_{\bf q} +
    \frac{D_{\bf q}(\phi)}{| \nabla {\bf q} |} \right ) \nabla q_i - 2 \lambda
    q_i \right \} & = 0, \hspace{1em} i=1,\ldots,4
\label{eq:dae1} \\
  \sum_i q_i^2 - 1 & = 0.
\label{eq:daeconstraint}
\end{align}
%
We allow the mobility $M_{\bf q}$ to depend on $\phi$ in order to
limit rotation in the ordered phase, further detailed below.
Equations (\ref{eq:dae1})--(\ref{eq:daeconstraint}) comprise a
semi-explicit, differential-algebraic system of index two (see, {\em
e.g.}, \cite{BCP89} for more information about the theory and
numerical solution of differential-algebraic systems).  Although an
algorithm for the integration of such a system could be pursued, it is
generally the case that the numerical integration of
differential-algebraic systems of index two or higher is facilitated
by first reducing the index of the system.  In the present case, this
is accomplished by replacing (\ref{eq:daeconstraint}) by its time
derivative and substituting (\ref{eq:dae1}), giving
%
\begin{equation}
  0 = 2 \sum_i q_i \dot{q}_i =
    2 \sum_i q_i \left[ \nabla \cdot \left( \epsilon_{\bf q} +
    \frac{D_{\bf q}(\phi) }{| \nabla {\bf q} |} \right)
    \nabla q_i - 2 \lambda q_i \right],
\label{eq:difcon}
\end{equation}
%
which yields an explicit expression for $\lambda$.  Upon elimination
of $\lambda$ in (\ref{eq:dae1}), we obtain the ordinary differential
equations
%
\begin{equation}
\begin{split}
  \dot{q}_i = M_{\bf q}(\phi) \bigg \{
    & \nabla \cdot \left (
    \epsilon_{\bf q} +  \frac{D_{\bf q}(\phi)}{| \nabla {\bf q} |} \right )
    \nabla q_i \\
    & - \frac{q_i}{\sum_\ell q_\ell^2} \sum_k q_k \nabla \cdot
    \left ( \epsilon_q +  \frac{D_{\bf q}(\phi)}{ | \nabla {\bf q} |} \right )
    \nabla q_k \bigg \},
\label{eq:qeom}
\end{split}
\end{equation}
%
which was originally formulated in \cite{0295-5075-71-1-131} (which
omits the $\epsilon_{\bf q}$ term and includes a noise term).

For any vector $v \equiv (v_1,v_2,v_3,v_4)$, let $\Pi({\bf q})v$
denote the orthogonal projection (with respect to the usual Euclidean
inner product) of $v$ onto ${\bf q}$.  The system (\ref{eq:qeom}) can
then be written as
%
\begin{equation}
  \dot{\bf q} =
    M_{\bf q}(\phi) \left (I - \Pi({\bf q}) \right )
    \nabla \cdot \left ( \epsilon_{\bf q} + 
    \frac{D_{\bf q}(\phi)}{ |\nabla
    {\bf q}|} \right ) \nabla {\bf q}.
\label{eq:qeom2}
\end{equation}
%
In this form, it is clear that solutions of (\ref{eq:qeom2}) also satisfy
the invariant
%
\begin{equation}
  {\bf q} \cdot \dot{\bf q} = 0,
\label{eq:invariant}
\end{equation}
%
which is just a restatement of the differentiated constraint
(\ref{eq:difcon}).  Solutions of (\ref{eq:qeom2}) with an initial
condition on the constraint surface (\ref{eq:daeconstraint}) therefore
remain on the surface at all times.  Differentiation of the constraint
(\ref{eq:daeconstraint}) has thus replaced the problem of integrating
an index two differential-algebraic system with the equivalent problem
of enforcing the invariant (\ref{eq:invariant}) in the integration of
an ordinary differential equation.

As with the mobility for the phase equation, the orientation mobility,
$M_q$, may be a general function.  It is common to use a functional
form that reduces $M_q$ as the phase variable, $\phi$, goes to 1 in
order to slow or prevent the wholesale rotation of ordered grains.
For the examples in this paper, we will set
%
\begin{equation}
  M_q(\phi) = M_q^{\rm min} + m(\phi) ( M_q^{\rm max} - M_q^{\rm min} ),
\label{eq:qmobility}
\end{equation}
%
where $M_q^{\rm max}$ varies with the problem and $M_q^{\rm min} =
10^{-6}$, {\em i.e.},
very near zero, with $m(\phi)$ an interpolating monotonic polynomial
satisfying $m(0) = 1$ and $m(1) = 0$.  We use
%
\begin{equation}
  m(\phi) = 1 - \phi^3 \left( 10 - 15 \phi + 6 \phi^2 \right)
\end{equation}

%%%%%%%%%%%%%%%%%%%%%%%%%%%%%%%%%%%%%%%%%%%%%%%%%%%%%%%%%%%%%%%%%%%%%%%
%
\section{Composition equation}

The particular form of the composition equation depends upon the
relationship between the variables $(c_l,c_\alpha,c_\beta)$, and
$(c,\phi,\eta)$ in (\ref{eq:cmix}).  In Appendix \ref{sec:kks}, the
equations for the Kim, Kim and Suzuki (KKS) model with three phases
are derived, which results in the following equation of motion
%
\begin{equation}
\begin{split}
  \frac{\partial c}{\partial t} = {}
  & M_c \nabla \cdot D_c^0(\phi,\eta,T) \nabla c
  \\ &
    + M_c \nabla \cdot D_c^0(\phi,\eta,T) h'_\phi \left[
    c_l - ( 1 - h_\eta ) c_\alpha - h_\eta c_\beta \right]
    \nabla \phi
  \\ &
    + M_c \nabla \cdot D_c^0(\phi,\eta,T) h_\phi h'_\eta
    ( c_\alpha - c_\beta ) \nabla \eta.
\label{eq:ceomkks}
\end{split}
\end{equation}
%
To actually compute the right-hand side of (\ref{eq:ceomkks}), we need
an expression for $D_c^0(\phi,\eta,T)$ as well as expressions for
$c_l(c,\phi,\eta)$, $c_\alpha(c,\phi,\eta)$, and
$c_\beta(c,\phi,\eta)$.  For that, we need to know explicit forms of
$f^l$, $f^\alpha$, and $f^\beta$.  See the appendices for specific
implementations for these.


%%%%%%%%%%%%%%%%%%%%%%%%%%%%%%%%%%%%%%%%%%%%%%%%%%%%%%%%%%%%%%%%%%%%%%%%%%%%%%%%%%%
%
\section{Elasticity}

We write the free energy functional
%
\begin{eqnarray}
  % \nonumber to remove numbering (before each equation)
  \nonumber
  F(\phi,{\bf q},c,T) &=& \int \left[\frac{\varepsilon_{\phi}^2}{2} |\nabla \phi |^2 + f(\phi,c,T) +
      \frac{\varepsilon_{\bf q}^2}{2} |\nabla {\bf q} |^2 +
      D_{\bf q}(\phi,T) |\nabla {\bf q} | \right] \\
  &+&  E_{elast}(\phi,{\bf q},c)
\end{eqnarray}
%
where
%
\begin{equation}\label{eq:eelast}
  E_{elast}(\phi,{\bf q},c) =
  \frac{1}{2}\int \sum_{ijkl} C_{ijkl}(\phi,{\bf q})
  \varepsilon^{el}_{ij}(\phi,{\bf q},c)\varepsilon^{el}_{kl}(\phi,{\bf q},c)
\end{equation}
%
and the elastic strain tensor $\varepsilon^{el}$ satisfies the
mechanical equilibrium equation.  For an inhomogeneous alloy, we write
%
\begin{equation}
  C_{ijkl}(\phi,{\bf q})
  =C_{ijkl}^0+p_C(\phi) (C_{ijkl}^1({\bf q})-C_{ijkl}^0)
\end{equation}
%
$C_{ijkl}$ could also depend on the composition $c$, but we will
neglect this dependence for now.

Even in the absence of stress, strain may be non-zero with respect to
a reference state.  If the reference state is $\phi=c=0$, we can write
the stress-free strain
%
\begin{equation}\label{eq:stress_free_strain}
  \varepsilon_{ij}^*(\phi,{\bf q},c)
  =\varepsilon_0 \delta_{ij} c
  +p_\varepsilon(\phi) \varepsilon_{ij}^{q*}({\bf q}).
\end{equation}
%
The first term corresponds to the variation of the lattice parameter
$a$ with composition $c$,
%
\begin{equation}\label{eq:epsilon0}
  \varepsilon_0=\frac{1}{a}\frac{da}{dc}.
\end{equation}
%
The elastic strain is defined as the difference between the total strain 
$\varepsilon_{ij}$ and the stress-free strain $\varepsilon_{ij}^*$
%
\begin{equation}\label{eq:elstrain}
  \varepsilon_{ij}^{el}(\vec x)
  =\varepsilon_{ij}(\vec x)-\varepsilon_{ij}^*(\vec x)
\end{equation}
%
We define the homogeneous strain as
%
\begin{equation}\label{eq:homstrain}
  \bar\varepsilon_{ij}=\frac{1}{V}\int_{V}\varepsilon_{ij}(\vec x)d\vec x
\end{equation}
%
where $V$ is the volume of the physical domain,
and
%
\begin{equation}\label{eq:depsilon}
  \delta\varepsilon_{ij}(\vec x)
  =\varepsilon_{ij}(\vec x)-\bar\varepsilon_{ij}.
\end{equation}
%
The homogeneous strain value is used to model the macroscopic volume
(fixed volume simulation).
It expresses the volume deformation compared to the equilibrium volume
in phase $\phi=0$.
It is an input parameter for the
simulation.

Given these definitions, the local stress is given by
%
\begin{equation}\label{eq:dsigma}
  \sigma^{el}_{ij}(\vec x)
  =\sum_{kl}C_{ijkl}\left[\delta\varepsilon_{kl}(\vec x)+\bar\varepsilon_{kl}-\varepsilon_{kl}^*(\vec x)\right]
\end{equation}
%
From (\ref{eq:equilibrium}), the mechanical equilibrium equations for
the displacement $\vec u(\vec x)$ are then
%
\begin{equation}\label{eq:equilibrium_i}
  \nabla\cdot A^{(i)}\vec u=b^{(i)}
\end{equation}
%
for $i=1,2,3$, where the matrices $A^{(i)}$ are defined by
%
\begin{equation}\label{eq:Ai}
  A_{jk}^{(i)}=\sum_l C_{ijkl}(\phi,{\bf q})\frac{\partial}{\partial x_l}
\end{equation}
%
and
%
\begin{equation}\label{eq:bi}
  b^{(i)}=-\sum_{j,k,l} \frac{\partial}{\partial x_j}
  \left[C_{ijkl}(\phi,{\bf q})(\bar\varepsilon_{kl}-\varepsilon_{kl}^*(\phi,{\bf q},c)(\vec x))\right].
\end{equation}
%
Note that by solving for the displacement $\vec u(\vec x)$ using
periodic boundary conditions, we assume that the macroscopic volume is
given by the homogeneous strain $\bar\varepsilon$ only.  Since the
elastic energy explicitly depends on the phase variable $\phi$, the
time evolution equation for $\phi$ will include an additional source
term given by
%
\begin{eqnarray}\label{eq:src_elast_phi}
  -M_\phi\frac{\delta E^{elast}}{\delta\phi} =
  -M_\phi & \sum_{ijkl} & 
  \left\{ \frac{1}{2}p'_C(\phi)(C_{ijkl}^1({\bf q}) -C_{ijkl}^0)
  \varepsilon^{el}_{ij}\varepsilon^{el}_{kl} \right. \nonumber \\  
  &-& \left. p'_\varepsilon(\phi)C_{ijkl}(\phi,{\bf q})\varepsilon^{el}_{ij}\varepsilon_{kl}^{q*}({\bf q}) \right\}
\end{eqnarray}
%
Similarly, an additional source term is needed for the quaternion time
evolution equation:
%
\begin{eqnarray}\label{eq:src_elast_q}
  -M_q \frac{\delta E^{elast}}{\delta q_\alpha} =
  -M_q & \sum_{ijkl} & 
  \left\{ \frac{1}{2}p_C(\phi)\frac{\partial C_{ijkl}^1({\bf q})}{\partial q_\alpha} \varepsilon^{el}_{ij}\varepsilon^{el}_{kl} \right. \nonumber
  \\
  &-& \left. p_\varepsilon(\phi) C_{ijkl}(\phi,{\bf q})\varepsilon^{el}_{ij}\frac{\partial\varepsilon_{kl}^{q*}({\bf q})}{\partial q_\alpha} \right\},
\end{eqnarray}
%
and for the concentration, the additional term in the right hand side
of the time evolution equation
%
\begin{equation}\label{eq:src_elast_c}
  \nabla D_c(c,\phi)\nabla \left(
  - \sum_{ijkl} C_{ijkl}(\phi,{\bf q}) \varepsilon^{el}_{ij}\varepsilon_0\delta_{kl} \right).
\end{equation}
%
where
$$
D_c(c,\phi)=D^0_c(c,\phi)\left(\frac{\partial^2 f}{\partial c^2}\right)
$$
In our implementation, we will compute everything in the laboratory
coordinates.

In Ref. \cite{HBSMZC07}, the stiffness matrices are given for the
$\varepsilon$ and $\delta$ phases of the HBSM model.  Values for
$\varepsilon^{q*}$ are also specified.  To use these values in our
model, we need to be able to express $C_{ijkl}^1({\bf q})$ and
$\varepsilon_{kl}^{q*}({\bf q})$ for an arbitrary ${\bf q}$, knowing
its values for the three specific directions corresponding to the
three variants of $\delta$ phase.  We can do that using the rotation
matrices described in Ming's note\cite{MT1}.

%%%%%%%%%%%%%%%%%%%%%%%%%%%%%%%%%%%%%%%%%%%%%%%%%%%%%%%%%%%%%%%%%%%%%%%%%%%%%%%%%%%

