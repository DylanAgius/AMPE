%% foo.tex
\begin{samepage}
\hrule
\begin{center}
\phantomsection
{\large \verb!CVode!}
\label{p:CVode}
\index{CVode}
\end{center}
\hrule\vspace{0.1in}

%% one line -------------------

\noindent{\bf \sc Purpose}

\begin{alltt}
CVode integrates the ODE.
\end{alltt}

\end{samepage}


%% definition  -------------------

\begin{samepage}

\noindent{\bf \sc Synopsis}

\begin{alltt}
function [varargout] = CVode(tout, itask) 
\end{alltt}

\end{samepage}

%% description -------------------

\noindent{\bf \sc Description}

\begin{alltt}
CVode integrates the ODE.

   Usage: [STATUS, T, Y] = CVode ( TOUT, ITASK ) 
          [STATUS, T, Y, YS] = CVode ( TOUT, ITASK )
          [STATUS, T, Y, YQ] = CVode  (TOUT, ITASK )
          [STATUS, T, Y, YQ, YS] = CVode ( TOUT, ITASK )

   If ITASK is 'Normal', then the solver integrates from its current internal 
   T value to a point at or beyond TOUT, then interpolates to T = TOUT and returns 
   Y(TOUT). If ITASK is 'OneStep', then the solver takes one internal time step 
   and returns in Y the solution at the new internal time. In this case, TOUT 
   is used only during the first call to CVode to determine the direction of 
   integration and the rough scale of the problem. In either case, the time 
   reached by the solver is returned in T.

   If quadratures were computed (see CVodeQuadInit), CVode will return their
   values at T in the vector YQ.

   If sensitivity calculations were enabled (see CVodeSensInit), CVode will 
   return their values at T in the matrix YS. Each row in the matrix YS
   represents the sensitivity vector with respect to one of the problem parameters.

   In ITASK =' Normal' mode, to obtain solutions at specific times T0,T1,...,TFINAL
   (all increasing or all decreasing) use TOUT = [T0 T1  ... TFINAL]. In this case
   the output arguments Y and YQ are matrices, each column representing the solution
   vector at the corresponding time returned in the vector T. If computed, the 
   sensitivities are eturned in the 3-dimensional array YS, with YS(:,:,I) representing
   the sensitivity vectors at the time T(I).

   On return, STATUS is one of the following:
     0: successful CVode return.
     1: CVode succeded and returned at tstop.
     2: CVode succeeded and found one or more roots. 


   See also CVodeSetOptions, CVodeGetStats
\end{alltt}






\vspace{0.1in}