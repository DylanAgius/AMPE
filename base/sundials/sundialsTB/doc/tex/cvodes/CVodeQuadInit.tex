%% foo.tex
\begin{samepage}
\hrule
\begin{center}
\phantomsection
{\large \verb!CVodeQuadInit!}
\label{p:CVodeQuadInit}
\index{CVodeQuadInit}
\end{center}
\hrule\vspace{0.1in}

%% one line -------------------

\noindent{\bf \sc Purpose}

\begin{alltt}
CVodeQuadInit allocates and initializes memory for quadrature integration.
\end{alltt}

\end{samepage}


%% definition  -------------------

\begin{samepage}

\noindent{\bf \sc Synopsis}

\begin{alltt}
function CVodeQuadInit(fctQ, yQ0, options) 
\end{alltt}

\end{samepage}

%% description -------------------

\noindent{\bf \sc Description}

\begin{alltt}
CVodeQuadInit allocates and initializes memory for quadrature integration.

   Usage: CVodeQuadInit ( QFUN, YQ0 [, OPTIONS ] ) 

   QFUN     is a function defining the righ-hand sides of the quadrature
            ODEs yQ' = fQ(t,y).
   YQ0      is the initial conditions vector yQ(t0).
   OPTIONS  is an (optional) set of QUAD options, created with
            the CVodeSetQuadOptions function. 

   See also: CVodeSetQuadOptions, CVQuadRhsFn
\end{alltt}






\vspace{0.1in}