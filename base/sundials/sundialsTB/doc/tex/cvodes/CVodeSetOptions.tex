%% foo.tex
\begin{samepage}
\hrule
\begin{center}
\phantomsection
{\large \verb!CVodeSetOptions!}
\label{p:CVodeSetOptions}
\index{CVodeSetOptions}
\end{center}
\hrule\vspace{0.1in}

%% one line -------------------

\noindent{\bf \sc Purpose}

\begin{alltt}
CVodeSetOptions creates an options structure for CVODES.
\end{alltt}

\end{samepage}


%% definition  -------------------

\begin{samepage}

\noindent{\bf \sc Synopsis}

\begin{alltt}
function options = CVodeSetOptions(varargin) 
\end{alltt}

\end{samepage}

%% description -------------------

\noindent{\bf \sc Description}

\begin{alltt}
CVodeSetOptions creates an options structure for CVODES.

   Usage: OPTIONS = CVodeSetOptions('NAME1',VALUE1,'NAME2',VALUE2,...)
          OPTIONS = CVodeSetOptions(OLDOPTIONS,'NAME1',VALUE1,...)

   OPTIONS = CVodeSetOptions('NAME1',VALUE1,'NAME2',VALUE2,...) creates 
   a CVODES options structure OPTIONS in which the named properties have 
   the specified values. Any unspecified properties have default values. 
   It is sufficient to type only the leading characters that uniquely 
   identify the property. Case is ignored for property names. 
   
   OPTIONS = CVodeSetOptions(OLDOPTIONS,'NAME1',VALUE1,...) alters an 
   existing options structure OLDOPTIONS.
   
   CVodeSetOptions with no input arguments displays all property names 
   and their possible values.
   
CVodeSetOptions properties
(See also the CVODES User Guide)

UserData - User data passed unmodified to all functions [ empty ]
   If UserData is not empty, all user provided functions will be
   passed the problem data as their last input argument. For example,
   the RHS function must be defined as YD = ODEFUN(T,Y,DATA).

LMM - Linear Multistep Method [ 'Adams' | {'BDF'} ]
   This property specifies whether the Adams method is to be used instead
   of the default Backward Differentiation Formulas (BDF) method.
   The Adams method is recommended for non-stiff problems, while BDF is
   recommended for stiff problems.
NonlinearSolver - Type of nonlinear solver used [ Functional | {Newton} ]
   The 'Functional' nonlinear solver is best suited for non-stiff
   problems, in conjunction with the 'Adams' linear multistep method,
   while 'Newton' is better suited for stiff problems, using the 'BDF'
   method.
RelTol - Relative tolerance [ positive scalar | {1e-4} ]
   RelTol defaults to 1e-4 and is applied to all components of the solution
   vector. See AbsTol.
AbsTol - Absolute tolerance [ positive scalar or vector | {1e-6} ]
   The relative and absolute tolerances define a vector of error weights
   with components
     ewt(i) = 1/(RelTol*|y(i)| + AbsTol)    if AbsTol is a scalar
     ewt(i) = 1/(RelTol*|y(i)| + AbsTol(i)) if AbsTol is a vector
   This vector is used in all error and convergence tests, which
   use a weighted RMS norm on all error-like vectors v:
     WRMSnorm(v) = sqrt( (1/N) sum(i=1..N) (v(i)*ewt(i))^2 ),
   where N is the problem dimension.
MaxNumSteps - Maximum number of steps [positive integer | {500}]
   CVode will return with an error after taking MaxNumSteps internal steps
   in its attempt to reach the next output time.
InitialStep - Suggested initial stepsize [ positive scalar ]
   By default, CVode estimates an initial stepsize h0 at the initial time
   t0 as the solution of
     WRMSnorm(h0^2 ydd / 2) = 1
   where ydd is an estimated second derivative of y(t0).
MaxStep - Maximum stepsize [ positive scalar | {inf} ]
   Defines an upper bound on the integration step size.
MinStep - Minimum stepsize [ positive scalar | {0.0} ]
   Defines a lower bound on the integration step size.
MaxOrder - Maximum method order [ 1-12 for Adams, 1-5 for BDF | {5} ]
   Defines an upper bound on the linear multistep method order.
StopTime - Stopping time [ scalar ]
   Defines a value for the independent variable past which the solution
   is not to proceed.
RootsFn - Rootfinding function [ function ]
   To detect events (roots of functions), set this property to the event
   function. See CVRootFn.
NumRoots - Number of root functions [ integer | {0} ]
   Set NumRoots to the number of functions for which roots are monitored.
   If NumRoots is 0, rootfinding is disabled.
StabilityLimDet - Stability limit detection algorithm [ {false} | true ]
   Flag used to turn on or off the stability limit detection algorithm
   within CVODES. This property can be used only with the BDF method.
   In this case, if the order is 3 or greater and if the stability limit
   is detected, the method order is reduced.

LinearSolver - Linear solver type [{Dense}|Diag|Band|GMRES|BiCGStab|TFQMR]
   Specifies the type of linear solver to be used for the Newton nonlinear
   solver (see NonlinearSolver). Valid choices are: Dense (direct, dense
   Jacobian), Band (direct, banded Jacobian), Diag (direct, diagonal Jacobian),
   GMRES (iterative, scaled preconditioned GMRES), BiCGStab (iterative, scaled
   preconditioned stabilized BiCG), TFQMR (iterative, scaled transpose-free QMR).
   The GMRES, BiCGStab, and TFQMR are matrix-free linear solvers.
JacobianFn - Jacobian function [ function ]
   This propeerty is overloaded. Set this value to a function that returns
   Jacobian information consistent with the linear solver used (see Linsolver).
   If not specified, CVODES uses difference quotient approximations.
   For the Dense linear solver, JacobianFn must be of type CVDenseJacFn and
   must return a dense Jacobian matrix. For the Band linear solver, JacobianFn
   must be of type CVBandJacFn and must return a banded Jacobian matrix.
   For the iterative linear solvers, GMRES, BiCGStab, and TFQMR, JacobianFn must
   be of type CVJacTimesVecFn and must return a Jacobian-vector product. This
   property is not used for the Diag linear solver.
   If these options are for a backward problem, the corresponding funciton types
   are CVDenseJacFnB for the Dense linear solver, CVBandJacFnB for he band linear
   solver, and CVJacTimesVecFnB for the iterative linear solvers.
KrylovMaxDim - Maximum number of Krylov subspace vectors [ integer | {5} ]
   Specifies the maximum number of vectors in the Krylov subspace. This property
   is used only if an iterative linear solver, GMRES, BiCGStab, or TFQMR is used
   (see LinSolver).
GramSchmidtType - Gram-Schmidt orthogonalization [ Classical | {Modified} ]
   Specifies the type of Gram-Schmidt orthogonalization (classical or modified).
   This property is used only if the GMRES linear solver is used (see LinSolver).
PrecType - Preconditioner type [ Left | Right | Both | {None} ]
   Specifies the type of user preconditioning to be done if an iterative linear
   solver, GMRES, BiCGStab, or TFQMR is used (see LinSolver). PrecType must be
   one of the following: 'None', 'Left', 'Right', or 'Both', corresponding to no
   preconditioning, left preconditioning only, right preconditioning only, and
   both left and right preconditioning, respectively.
PrecModule - Preconditioner module [ BandPre | BBDPre | {UserDefined} ]
   If PrecModule = 'UserDefined', then the user must provide at least a
   preconditioner solve function (see PrecSolveFn)
   CVODES provides the following two general-purpose preconditioner modules:
     BandPre provide a band matrix preconditioner based on difference quotients
   of the ODE right-hand side function. The user must specify the lower and
   upper half-bandwidths through the properties LowerBwidth and UpperBwidth,
   respectively.
     BBDPre can be only used with parallel vectors. It provide a preconditioner
   matrix that is block-diagonal with banded blocks. The blocking corresponds
   to the distribution of the dependent variable vector y among the processors.
   Each preconditioner block is generated from the Jacobian of the local part
   (on the current processor) of a given function g(t,y) approximating
   f(t,y) (see GlocalFn). The blocks are generated by a difference quotient
   scheme on each processor independently. This scheme utilizes an assumed
   banded structure with given half-bandwidths, mldq and mudq (specified through
   LowerBwidthDQ and UpperBwidthDQ, respectively). However, the banded Jacobian
   block kept by the scheme has half-bandwiths ml and mu (specified through
   LowerBwidth and UpperBwidth), which may be smaller.
PrecSetupFn - Preconditioner setup function [ function ]
   If PrecType is not 'None', PrecSetupFn specifies an optional function which,
   together with PrecSolve, defines left and right preconditioner matrices
   (either of which can be trivial), such that the product P1*P2 is an
   aproximation to the Newton matrix. PrecSetupFn must be of type CVPrecSetupFn
   or CVPrecSetupFnB for forward and backward problems, respectively.
PrecSolveFn - Preconditioner solve function [ function ]
   If PrecType is not 'None', PrecSolveFn specifies a required function which
   must solve a linear system Pz = r, for given r. PrecSolveFn must be of type
   CVPrecSolveFn or CVPrecSolveFnB for forward and backward problems, respectively.
GlocalFn - Local right-hand side approximation funciton for BBDPre [ function ]
   If PrecModule is BBDPre, GlocalFn specifies a required function that
   evaluates a local approximation to the ODE right-hand side. GlocalFn must
   be of type CVGlocFn or CVGlocFnB for forward and backward problems, respectively.
GcommFn - Inter-process communication function for BBDPre [ function ]
   If PrecModule is BBDPre, GcommFn specifies an optional function
   to perform any inter-process communication required for the evaluation of
   GlocalFn. GcommFn must be of type CVGcommFn or CVGcommFnB  for forward and
   backward problems, respectively.
LowerBwidth - Jacobian/preconditioner lower bandwidth [ integer | {0} ]
   This property is overloaded. If the Band linear solver is used (see LinSolver),
   it specifies the lower half-bandwidth of the band Jacobian approximation.
   If one of the three iterative linear solvers, GMRES, BiCGStab, or TFQMR is used
   (see LinSolver) and if the BBDPre preconditioner module in CVODES is used
   (see PrecModule), it specifies the lower half-bandwidth of the retained
   banded approximation of the local Jacobian block. If the BandPre preconditioner
   module (see PrecModule) is used, it specifies the lower half-bandwidth of
   the band preconditioner matrix. LowerBwidth defaults to 0 (no sub-diagonals).
UpperBwidth - Jacobian/preconditioner upper bandwidth [ integer | {0} ]
   This property is overloaded. If the Band linear solver is used (see LinSolver),
   it specifies the upper half-bandwidth of the band Jacobian approximation.
   If one of the three iterative linear solvers, GMRES, BiCGStab, or TFQMR is used
   (see LinSolver) and if the BBDPre preconditioner module in CVODES is used
   (see PrecModule), it specifies the upper half-bandwidth of the retained
   banded approximation of the local Jacobian block. If the BandPre
   preconditioner module (see PrecModule) is used, it specifies the upper
   half-bandwidth of the band preconditioner matrix. UpperBwidth defaults to
   0 (no super-diagonals).
LowerBwidthDQ - BBDPre preconditioner DQ lower bandwidth [ integer | {0} ]
   Specifies the lower half-bandwidth used in the difference-quotient Jacobian
   approximation for the BBDPre preconditioner (see PrecModule).
UpperBwidthDQ - BBDPre preconditioner DQ upper bandwidth [ integer | {0} ]
   Specifies the upper half-bandwidth used in the difference-quotient Jacobian
   approximation for the BBDPre preconditioner (see PrecModule).

MonitorFn - User-provied monitoring function [ function ]
   Specifies a function that is called after each successful integration step.
   This function must have type CVMonitorFn or CVMonitorFnB, depending on
   whether these options are for a forward or a backward problem, respectively.
   Sample monitoring functions CVodeMonitor and CvodeMonitorB are provided
   with CVODES.
MonitorData - User-provied data for the monitoring function [ struct ]
   Specifies a data structure that is passed to the MonitorFn function every
   time it is called.

SensDependent - Backward problem depending on sensitivities [ {false} | true ]
   Specifies whether the backward problem right-hand side depends on
   forward sensitivites. If TRUE, the right-hand side function provided for
   this backward problem must have the appropriate type (see CVRhsFnB).


NOTES:

   The properties listed above that can only be used for forward problems
   are: StopTime, RootsFn, and NumRoots.

   The property SensDependent is relevant only for backward problems.


   See also
        CVodeInit, CVodeReInit, CVodeInitB, CVodeReInitB
        CVRhsFn, CVRootFn,
        CVDenseJacFn, CVBandJacFn, CVJacTimesVecFn
        CVPrecSetupFn, CVPrecSolveFn
        CVGlocalFn, CVGcommFn
        CVMonitorFn
        CVRhsFnB,
        CVDenseJacFnB, CVBandJacFnB, CVJacTimesVecFnB
        CVPrecSetupFnB, CVPrecSolveFnB
        CVGlocalFnB, CVGcommFnB
        CVMonitorFnB
\end{alltt}






\vspace{0.1in}