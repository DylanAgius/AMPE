%% foo.tex
\begin{samepage}
\hrule
\begin{center}
\phantomsection
{\large \verb!CVodeSetFSAOptions!}
\label{p:CVodeSetFSAOptions}
\index{CVodeSetFSAOptions}
\end{center}
\hrule\vspace{0.1in}

%% one line -------------------

\noindent{\bf \sc Purpose}

\begin{alltt}
CVodeSetFSAOptions creates an options structure for FSA with CVODES.
\end{alltt}

\end{samepage}


%% definition  -------------------

\begin{samepage}

\noindent{\bf \sc Synopsis}

\begin{alltt}
function options = CVodeSetFSAOptions(varargin) 
\end{alltt}

\end{samepage}

%% description -------------------

\noindent{\bf \sc Description}

\begin{alltt}
CVodeSetFSAOptions creates an options structure for FSA with CVODES.

   Usage: OPTIONS = CVodeSetFSAOptions('NAME1',VALUE1,'NAME2',VALUE2,...)
          OPTIONS = CVodeSetFSAOptions(OLDOPTIONS,'NAME1',VALUE1,...)
          OPTIONS = CVodeSetFSAOptions(OLDOPTIONS,NEWOPTIONS)

   OPTIONS = CVodeSetFSAOptions('NAME1',VALUE1,'NAME2',VALUE2,...) creates 
   a CVODES options structure OPTIONS in which the named properties have 
   the specified values. Any unspecified properties have default values. 
   It is sufficient to type only the leading characters that uniquely 
   identify the property. Case is ignored for property names. 
   
   OPTIONS = CVodeSetFSAOptions(OLDOPTIONS,'NAME1',VALUE1,...) alters an 
   existing options structure OLDOPTIONS.
   
   OPTIONS = CVodeSetFSAOptions(OLDOPTIONS,NEWOPTIONS) combines an existing 
   options structure OLDOPTIONS with a new options structure NEWOPTIONS. 
   Any new properties overwrite corresponding old properties. 
   
   CVodeSetFSAOptions with no input arguments displays all property names 
   and their possible values.
   
CVodeSetFSAOptions properties
(See also the CVODES User Guide)

FSAmethod - FSA solution method [ 'Simultaneous' | {'Staggered'} ]
   Specifies the FSA method for treating the nonlinear system solution for
   sensitivity variables. In the simultaneous case, the nonlinear systems
   for states and all sensitivities are solved simultaneously. In the
   Staggered case, the nonlinear system for states is solved first and then
   the nonlinear systems for all sensitivities are solved at the same time.
ParamField - Problem parameters  [ string ]
   Specifies the name of the field in the user data structure (passed as an
   argument to CVodeMalloc) in which the nominal values of the problem
   parameters are stored. This property is used only if  CVODES will use difference
   quotient approximations to the sensitivity right-hand sides (see SensRhsFn).
ParamList - Parameters with respect to which FSA is performed [ integer vector ]
   Specifies a list of Ns parameters with respect to which sensitivities are to
   be computed. This property is used only if CVODES will use difference-quotient
   approximations to the sensitivity right-hand sides (see SensRhsFn below).
   Its length must be Ns, consistent with the number of columns of yS0
   (see CVodeSensMalloc).
ParamScales - Order of magnitude for problem parameters [ vector ]
   Provides order of magnitude information for the parameters with respect to
   which sensitivities are computed. This information is used if CVODES
   approximates the sensitivity right-hand sides (see SensRhsFn below) or if CVODES
   estimates integration tolerances for the sensitivity variables (see SensReltol
   and SensAbsTol).
SensRelTol - Relative tolerance for sensitivity variables [ positive scalar ]
   Specifies the scalar relative tolerance for the sensitivity variables.
   See also SensAbsTol.
SensAbsTol - Absolute tolerance for sensitivity variables [ row-vector or matrix ]
   Specifies the absolute tolerance for sensitivity variables. SensAbsTol must be
   either a row vector of dimension Ns, in which case each of its components is
   used as a scalar absolute tolerance for the coresponding sensitivity vector,
   or a N x Ns matrix, in which case each of its columns is used as a vector
   of absolute tolerances for the corresponding sensitivity vector.
   By default, CVODES estimates the integration tolerances for sensitivity
   variables, based on those for the states and on the order of magnitude
   information for the problem parameters specified through ParamScales.
SensErrControl - Error control strategy for sensitivity variables [ on | {off} ]
   Specifies whether sensitivity variables are included in the error control test.
   Note that sensitivity variables are always included in the nonlinear system
   convergence test.
SensDQtype - Type of DQ approx. of the sensi. RHS [{Centered} | Forward ]
   Specifies whether to use centered (second-order) or forward (first-order)
   difference quotient approximations of the sensitivity eqation right-hand
   sides. This property is used only if a user-defined sensitivity right-hand
   side function was not provided.
SensDQparam - Cut-off parameter for the DQ approx. of the sensi. RHS [ scalar | {0.0} ]
   Specifies the value which controls the selection of the difference-quotient
   scheme used in evaluating the sensitivity right-hand sides (switch between
   simultaneous or separate evaluations of the two components in the sensitivity
   right-hand side). The default value 0.0 indicates the use of simultaenous approximation
   exclusively (centered or forward, depending on the value of SensDQtype.
   For SensDQparam &gt;= 1, CVODES uses a simultaneous approximation if the estimated
   DQ perturbations for states and parameters are within a factor of SensDQparam,
   and separate approximations otherwise. Note that a value SensDQparam &lt; 1
   will inhibit switching! This property is used only if a user-defined sensitivity
   right-hand side function was not provided.

   See also
        CVodeSensInit, CVodeSensReInit
\end{alltt}






\vspace{0.1in}