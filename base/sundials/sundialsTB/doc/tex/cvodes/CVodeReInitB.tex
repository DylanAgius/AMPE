%% foo.tex
\begin{samepage}
\hrule
\begin{center}
\phantomsection
{\large \verb!CVodeReInitB!}
\label{p:CVodeReInitB}
\index{CVodeReInitB}
\end{center}
\hrule\vspace{0.1in}

%% one line -------------------

\noindent{\bf \sc Purpose}

\begin{alltt}
CVodeReInitB re-initializes backward memory for CVODES.
\end{alltt}

\end{samepage}


%% definition  -------------------

\begin{samepage}

\noindent{\bf \sc Synopsis}

\begin{alltt}
function CVodeReInitB(idxB, tB0, yB0, optionsB) 
\end{alltt}

\end{samepage}

%% description -------------------

\noindent{\bf \sc Description}

\begin{alltt}
CVodeReInitB re-initializes backward memory for CVODES.
   where a prior call to CVodeInitB has been made with the same
   problem size NB. CVodeReInitB performs the same input checking
   and initializations that CVodeInitB does, but it does no 
   memory allocation, assuming that the existing internal memory 
   is sufficient for the new problem.

   Usage:   CVodeReInitB ( IDXB, TB0, YB0 [, OPTIONSB] )

   IDXB     is the index of the backward problem, returned by
            CVodeInitB.
   TB0      is the final value of t.
   YB0      is the final condition vector yB(tB0).  
   OPTIONSB is an (optional) set of integration options, created with
            the CVodeSetOptions function. 

   See also: CVodeSetOptions, CVodeInitB
\end{alltt}






\vspace{0.1in}