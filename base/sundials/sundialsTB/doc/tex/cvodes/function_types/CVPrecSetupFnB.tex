%% # foo.tex
\begin{samepage}
\hrule
\begin{center}
\phantomsection
{\large \verb!CVPrecSetupFnB!}
\label{p:CVPrecSetupFnB}
\index{CVPrecSetupFnB}
\end{center}
\hrule\vspace{0.1in}

%% one line -------------------

\noindent{\bf \sc Purpose}

\begin{alltt}
CVPrecSetupFnB - type for user provided preconditioner setup function for backward problems.
\end{alltt}

\end{samepage}


%% definition  -------------------

\begin{samepage}

\noindent{\bf \sc Synopsis}

\begin{alltt}
This is a script file. 
\end{alltt}

\end{samepage}

%% description -------------------

\noindent{\bf \sc Description}

\begin{alltt}
CVPrecSetupFnB - type for user provided preconditioner setup function for backward problems.

   The user-supplied preconditioner setup function PSETFUN and
   the user-supplied preconditioner solve function PSOLFUN
   together must define left and right preconditoner matrices
   P1 and P2 (either of which may be trivial), such that the
   product P1*P2 is an approximation to the Newton matrix
   M = I - gamma*J.  Here J is the system Jacobian J = df/dy,
   and gamma is a scalar proportional to the integration step
   size h.  The solution of systems P z = r, with P = P1 or P2,
   is to be carried out by the PrecSolve function, and PSETFUN
   is to do any necessary setup operations.

   The user-supplied preconditioner setup function PSETFUN
   is to evaluate and preprocess any Jacobian-related data
   needed by the preconditioner solve function PSOLFUN.
   This might include forming a crude approximate Jacobian,
   and performing an LU factorization on the resulting
   approximation to M.  This function will not be called in
   advance of every call to PSOLFUN, but instead will be called
   only as often as necessary to achieve convergence within the
   Newton iteration.  If the PSOLFUN function needs no
   preparation, the PSETFUN function need not be provided.

   For greater efficiency, the PSETFUN function may save
   Jacobian-related data and reuse it, rather than generating it
   from scratch.  In this case, it should use the input flag JOK
   to decide whether to recompute the data, and set the output
   flag JCUR accordingly.

   Each call to the PSETFUN function is preceded by a call to
   ODEFUN with the same (t,y) arguments.  Thus the PSETFUN
   function can use any auxiliary data that is computed and
   saved by the ODEFUN function and made accessible to PSETFUN.


   The function PSETFUNB must be defined either as
        FUNCTION [JCURB, FLAG] = PSETFUNB(T,Y,YB,FYB,JOK,GAMMAB)
   or as
        FUNCTION [JCURB, FLAG, NEW_DATA] = PSETFUNB(T,Y,YB,FYB,JOK,GAMMAB,DATA)
   depending on whether a user data structure DATA was specified in
   CVodeMalloc. In either case, it must return the flags JCURB and FLAG.

   See also CVPrecSolveFnB, CVodeSetOptions

   NOTE: PSETFUNB is specified through the property PrecSetupFn to
   CVodeSetOptions and is used only if the property LinearSolver was
   set to 'GMRES', 'BiCGStab', or 'TFQMR' and if the property PrecType
   is not 'None'.
\end{alltt}






\vspace{0.1in}