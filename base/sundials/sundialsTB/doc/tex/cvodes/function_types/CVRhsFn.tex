%% # foo.tex
\begin{samepage}
\hrule
\begin{center}
\phantomsection
{\large \verb!CVRhsFn!}
\label{p:CVRhsFn}
\index{CVRhsFn}
\end{center}
\hrule\vspace{0.1in}

%% one line -------------------

\noindent{\bf \sc Purpose}

\begin{alltt}
CVRhsFn - type for user provided RHS function
\end{alltt}

\end{samepage}


%% definition  -------------------

\begin{samepage}

\noindent{\bf \sc Synopsis}

\begin{alltt}
This is a script file. 
\end{alltt}

\end{samepage}

%% description -------------------

\noindent{\bf \sc Description}

\begin{alltt}
CVRhsFn - type for user provided RHS function

   The function ODEFUN must be defined as 
        FUNCTION [YD, FLAG] = ODEFUN(T,Y)
   and must return a vector YD corresponding to f(t,y).
   If a user data structure DATA was specified in CVodeMalloc, then
   ODEFUN must be defined as
        FUNCTION [YD, FLAG, NEW_DATA] = ODEFUN(T,Y,DATA)
   If the local modifications to the user data structure are needed 
   in other user-provided functions then, besides setting the vector YD,
   the ODEFUN function must also set NEW_DATA. Otherwise, it should set
   NEW_DATA=[] (do not set NEW_DATA = DATA as it would lead to
   unnecessary copying).

   The function ODEFUN must set FLAG=0 if successful, FLAG&lt;0 if an
   unrecoverable failure occurred, or FLAG&gt;0 if a recoverable error
   occurred.

   See also CVodeInit
\end{alltt}






\vspace{0.1in}