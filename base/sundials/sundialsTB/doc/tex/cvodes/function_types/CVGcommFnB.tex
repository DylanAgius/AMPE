%% # foo.tex
\begin{samepage}
\hrule
\begin{center}
\phantomsection
{\large \verb!CVGcommFnB!}
\label{p:CVGcommFnB}
\index{CVGcommFnB}
\end{center}
\hrule\vspace{0.1in}

%% one line -------------------

\noindent{\bf \sc Purpose}

\begin{alltt}
CVGcommFn - type for user provided communication function (BBDPre) for backward problems.
\end{alltt}

\end{samepage}


%% definition  -------------------

\begin{samepage}

\noindent{\bf \sc Synopsis}

\begin{alltt}
This is a script file. 
\end{alltt}

\end{samepage}

%% description -------------------

\noindent{\bf \sc Description}

\begin{alltt}
CVGcommFn - type for user provided communication function (BBDPre) for backward problems.

   The function GCOMFUNB must be defined either as
        FUNCTION FLAG = GCOMFUNB(T, Y, YB)
   or as
        FUNCTION [FLAG, NEW_DATA] = GCOMFUNB(T, Y, YB, DATA)
   depending on whether a user data structure DATA was specified in
   CVodeMalloc. 

   The function GCOMFUNB must set FLAG=0 if successful, FLAG&lt;0 if an
   unrecoverable failure occurred, or FLAG&gt;0 if a recoverable error
   occurred.

   See also CVGlocalFnB, CVodeSetOptions

   NOTES:
     GCOMFUNB is specified through the GcommFn property in CVodeSetOptions
     and is used only if the property PrecModule is set to 'BBDPre'.

     Each call to GCOMFUNB is preceded by a call to the RHS function
     ODEFUNB with the same arguments T, Y, and YB. Thus GCOMFUNB can
     omit any communication done by ODEFUNB if relevant to the evaluation
     of G by GLOCFUNB. If all necessary communication was done by ODEFUNB,
     GCOMFUNB need not be provided.
\end{alltt}






\vspace{0.1in}