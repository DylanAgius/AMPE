%% # foo.tex
\begin{samepage}
\hrule
\begin{center}
\phantomsection
{\large \verb!CVJacTimesVecFnB!}
\label{p:CVJacTimesVecFnB}
\index{CVJacTimesVecFnB}
\end{center}
\hrule\vspace{0.1in}

%% one line -------------------

\noindent{\bf \sc Purpose}

\begin{alltt}
CVJacTimesVecFnB - type for user provided Jacobian times vector function for backward problems.
\end{alltt}

\end{samepage}


%% definition  -------------------

\begin{samepage}

\noindent{\bf \sc Synopsis}

\begin{alltt}
This is a script file. 
\end{alltt}

\end{samepage}

%% description -------------------

\noindent{\bf \sc Description}

\begin{alltt}
CVJacTimesVecFnB - type for user provided Jacobian times vector function for backward problems.

   The function JTVFUNB must be defined either as
        FUNCTION [JVB, FLAG] = JTVFUNB(T,Y,YB,FYB,VB)
   or as
        FUNCTION [JVB, FLAG, NEW_DATA] = JTVFUNB(T,Y,YB,FYB,VB,DATA)
   depending on whether a user data structure DATA was specified in
   CVodeMalloc. In either case, it must return the vector JVB, the
   product of the Jacobian of fB(t,y,yB) with respect to yB and a vector
   vB. The input argument FYB contains the current value of f(t,y,yB).

   The function JTVFUNB must set FLAG=0 if successful, or FLAG~=0 if
   a failure occurred.

   See also CVodeSetOptions

   NOTE: JTVFUNB is specified through the property JacobianFn to
   CVodeSetOptions and is used only if the property LinearSolver
   was set to 'GMRES', 'BiCGStab', or 'TFQMR'.
\end{alltt}






\vspace{0.1in}