%% # foo.tex
\begin{samepage}
\hrule
\begin{center}
\phantomsection
{\large \verb!CVGlocalFn!}
\label{p:CVGlocalFn}
\index{CVGlocalFn}
\end{center}
\hrule\vspace{0.1in}

%% one line -------------------

\noindent{\bf \sc Purpose}

\begin{alltt}
CVGlocalFn - type for user provided RHS approximation function (BBDPre).
\end{alltt}

\end{samepage}


%% definition  -------------------

\begin{samepage}

\noindent{\bf \sc Synopsis}

\begin{alltt}
This is a script file. 
\end{alltt}

\end{samepage}

%% description -------------------

\noindent{\bf \sc Description}

\begin{alltt}
CVGlocalFn - type for user provided RHS approximation function (BBDPre).

   The function GLOCFUN must be defined as 
        FUNCTION [GLOC, FLAG] = GLOCFUN(T,Y)
   and must return a vector GLOC corresponding to an approximation to f(t,y)
   which will be used in the BBDPRE preconditioner module. The case where
   G is mathematically identical to F is allowed.
   If a user data structure DATA was specified in CVodeMalloc, then
   GLOCFUN must be defined as
        FUNCTION [GLOC, FLAG, NEW_DATA] = GLOCFUN(T,Y,DATA)
   If the local modifications to the user data structure are needed 
   in other user-provided functions then, besides setting the vector G,
   the GLOCFUN function must also set NEW_DATA. Otherwise, it should set
   NEW_DATA=[] (do not set NEW_DATA = DATA as it would lead to
   unnecessary copying).

   The function GLOCFUN must set FLAG=0 if successful, FLAG&lt;0 if an
   unrecoverable failure occurred, or FLAG&gt;0 if a recoverable error
   occurred.

   See also CVGcommFn, CVodeSetOptions

   NOTE: GLOCFUN is specified through the GlocalFn property in CVodeSetOptions
   and is used only if the property PrecModule is set to 'BBDPre'.
\end{alltt}






\vspace{0.1in}