%% # foo.tex
\begin{samepage}
\hrule
\begin{center}
\phantomsection
{\large \verb!CVQuadRhsFnB!}
\label{p:CVQuadRhsFnB}
\index{CVQuadRhsFnB}
\end{center}
\hrule\vspace{0.1in}

%% one line -------------------

\noindent{\bf \sc Purpose}

\begin{alltt}
CVQuadRhsFnB - type for user provided quadrature RHS function for backward problems
\end{alltt}

\end{samepage}


%% definition  -------------------

\begin{samepage}

\noindent{\bf \sc Synopsis}

\begin{alltt}
This is a script file. 
\end{alltt}

\end{samepage}

%% description -------------------

\noindent{\bf \sc Description}

\begin{alltt}
CVQuadRhsFnB - type for user provided quadrature RHS function for backward problems

   The function ODEQFUNB must be defined either as
        FUNCTION [YQBD, FLAG] = ODEQFUNB(T,Y,YB)
   or as
        FUNCTION [YQBD, FLAG, NEW_DATA] = ODEQFUNB(T,Y,YB,DATA)
   depending on whether a user data structure DATA was specified in
   CVodeMalloc. In either case, it must return the vector YQBD
   corresponding to fQB(t,y,yB), the integrand for the integral to be 
   evaluated on the backward phase.

   The function ODEQFUNB must set FLAG=0 if successful, FLAG&lt;0 if an
   unrecoverable failure occurred, or FLAG&gt;0 if a recoverable error
   occurred.

   See also CVodeQuadInitB
\end{alltt}






\vspace{0.1in}