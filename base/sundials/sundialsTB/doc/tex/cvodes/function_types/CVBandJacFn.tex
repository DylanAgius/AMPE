%% # foo.tex
\begin{samepage}
\hrule
\begin{center}
\phantomsection
{\large \verb!CVBandJacFn!}
\label{p:CVBandJacFn}
\index{CVBandJacFn}
\end{center}
\hrule\vspace{0.1in}

%% one line -------------------

\noindent{\bf \sc Purpose}

\begin{alltt}
CVBandJacFn - type for user provided banded Jacobian function.
\end{alltt}

\end{samepage}


%% definition  -------------------

\begin{samepage}

\noindent{\bf \sc Synopsis}

\begin{alltt}
This is a script file. 
\end{alltt}

\end{samepage}

%% description -------------------

\noindent{\bf \sc Description}

\begin{alltt}
CVBandJacFn - type for user provided banded Jacobian function.

   The function BJACFUN must be defined as 
        FUNCTION [J, FLAG] = BJACFUN(T, Y, FY)
   and must return a matrix J corresponding to the banded Jacobian of f(t,y).
   The input argument FY contains the current value of f(t,y).
   If a user data structure DATA was specified in CVodeMalloc, then
   BJACFUN must be defined as
        FUNCTION [J, FLAG, NEW_DATA] = BJACFUN(T, Y, FY, DATA)
   If the local modifications to the user data structure are needed in
   other user-provided functions then, besides setting the matrix J,
   the BJACFUN function must also set NEW_DATA. Otherwise, it should 
   set NEW_DATA=[] (do not set NEW_DATA = DATA as it would lead to 
   unnecessary copying).

   The function BJACFUN must set FLAG=0 if successful, FLAG&lt;0 if an
   unrecoverable failure occurred, or FLAG&gt;0 if a recoverable error
   occurred.

   See also CVodeSetOptions

   See the CVODES user guide for more informaiton on the structure of
   a banded Jacobian.

   NOTE: BJACFUN is specified through the property JacobianFn to
   CVodeSetOptions and is used only if the property LinearSolver
   was set to 'Band'.
\end{alltt}






\vspace{0.1in}