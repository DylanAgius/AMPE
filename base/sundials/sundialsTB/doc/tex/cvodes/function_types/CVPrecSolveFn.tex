%% # foo.tex
\begin{samepage}
\hrule
\begin{center}
\phantomsection
{\large \verb!CVPrecSolveFn!}
\label{p:CVPrecSolveFn}
\index{CVPrecSolveFn}
\end{center}
\hrule\vspace{0.1in}

%% one line -------------------

\noindent{\bf \sc Purpose}

\begin{alltt}
CVPrecSolveFn - type for user provided preconditioner solve function.
\end{alltt}

\end{samepage}


%% definition  -------------------

\begin{samepage}

\noindent{\bf \sc Synopsis}

\begin{alltt}
This is a script file. 
\end{alltt}

\end{samepage}

%% description -------------------

\noindent{\bf \sc Description}

\begin{alltt}
CVPrecSolveFn - type for user provided preconditioner solve function.

   The user-supplied preconditioner solve function PSOLFN
   is to solve a linear system P z = r in which the matrix P is
   one of the preconditioner matrices P1 or P2, depending on the
   type of preconditioning chosen.

   The function PSOLFUN must be defined as 
        FUNCTION [Z, FLAG] = PSOLFUN(T,Y,FY,R)
   and must return a vector Z containing the solution of Pz=r.
   If PSOLFUN was successful, it must return FLAG=0. For a recoverable 
   error (in which case the step will be retried) it must set FLAG to a 
   positive value. If an unrecoverable error occurs, it must set FLAG
   to a negative value, in which case the integration will be halted.
   The input argument FY contains the current value of f(t,y).

   If a user data structure DATA was specified in CVodeMalloc, then
   PSOLFUN must be defined as
        FUNCTION [Z, FLAG, NEW_DATA] = PSOLFUN(T,Y,FY,R,DATA)
   If the local modifications to the user data structure are needed in
   other user-provided functions then, besides setting the vector Z and
   the flag FLAG, the PSOLFUN function must also set NEW_DATA. Otherwise,
   it should set NEW_DATA=[] (do not set NEW_DATA = DATA as it would
   lead to unnecessary copying).

   See also CVPrecSetupFn, CVodeSetOptions

   NOTE: PSOLFUN is specified through the property PrecSolveFn to
   CVodeSetOptions and is used only if the property LinearSolver was
   set to 'GMRES', 'BiCGStab', or 'TFQMR' and if the property PrecType
   is not 'None'.
\end{alltt}






\vspace{0.1in}