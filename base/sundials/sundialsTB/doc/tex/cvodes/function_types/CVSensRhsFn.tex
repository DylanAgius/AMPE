%% # foo.tex
\begin{samepage}
\hrule
\begin{center}
\phantomsection
{\large \verb!CVSensRhsFn!}
\label{p:CVSensRhsFn}
\index{CVSensRhsFn}
\end{center}
\hrule\vspace{0.1in}

%% one line -------------------

\noindent{\bf \sc Purpose}

\begin{alltt}
CVSensRhsFn - type for user provided sensitivity RHS function.
\end{alltt}

\end{samepage}


%% definition  -------------------

\begin{samepage}

\noindent{\bf \sc Synopsis}

\begin{alltt}
This is a script file. 
\end{alltt}

\end{samepage}

%% description -------------------

\noindent{\bf \sc Description}

\begin{alltt}
CVSensRhsFn - type for user provided sensitivity RHS function.

   The function ODESFUN must be defined as 
        FUNCTION [YSD, FLAG] = ODESFUN(T,Y,YD,YS)
   and must return a matrix YSD corresponding to fS(t,y,yS).
   If a user data structure DATA was specified in CVodeMalloc, then
   ODESFUN must be defined as
        FUNCTION [YSD, FLAG, NEW_DATA] = ODESFUN(T,Y,YD,YS,DATA)
   If the local modifications to the user data structure are needed in
   other user-provided functions then, besides setting the matrix YSD,
   the ODESFUN function must also set NEW_DATA. Otherwise, it should
   set NEW_DATA=[] (do not set NEW_DATA = DATA as it would lead to 
   unnecessary copying).

   The function ODESFUN must set FLAG=0 if successful, FLAG&lt;0 if an
   unrecoverable failure occurred, or FLAG&gt;0 if a recoverable error
   occurred.

   See also CVodeSetFSAOptions

   NOTE: ODESFUN is specified through the property FSARhsFn to
         CVodeSetFSAOptions.
\end{alltt}






\vspace{0.1in}