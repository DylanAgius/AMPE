%% foo.tex
\begin{samepage}
\hrule
\begin{center}
\phantomsection
{\large \verb!CVodeInitB!}
\label{p:CVodeInitB}
\index{CVodeInitB}
\end{center}
\hrule\vspace{0.1in}

%% one line -------------------

\noindent{\bf \sc Purpose}

\begin{alltt}
CVodeInitB allocates and initializes backward memory for CVODES.
\end{alltt}

\end{samepage}


%% definition  -------------------

\begin{samepage}

\noindent{\bf \sc Synopsis}

\begin{alltt}
function idxB = CVodeInitB(fctB, tB0, yB0, optionsB) 
\end{alltt}

\end{samepage}

%% description -------------------

\noindent{\bf \sc Description}

\begin{alltt}
CVodeInitB allocates and initializes backward memory for CVODES.

   Usage:   IDXB = CVodeInitB ( FCTB, TB0, YB0 [, OPTIONSB] )

   FCTB     is a function defining the adjoint ODE right-hand side.
            This function must return a vector containing the current 
            value of the adjoint ODE righ-hand side.
   TB0      is the final value of t.
   YB0      is the final condition vector yB(tB0).  
   OPTIONSB is an (optional) set of integration options, created with
            the CVodeSetOptions function. 

   CVodeInitB returns the index IDXB associated with this backward
   problem. This index must be passed as an argument to any subsequent
   functions related to this backward problem.

   See also: CVRhsFnB
\end{alltt}






\vspace{0.1in}