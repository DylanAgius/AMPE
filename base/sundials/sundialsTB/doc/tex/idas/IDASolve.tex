%% foo.tex
\begin{samepage}
\hrule
\begin{center}
\phantomsection
{\large \verb!IDASolve!}
\label{p:IDASolve}
\index{IDASolve}
\end{center}
\hrule\vspace{0.1in}

%% one line -------------------

\noindent{\bf \sc Purpose}

\begin{alltt}
IDASolve integrates the DAE.
\end{alltt}

\end{samepage}


%% definition  -------------------

\begin{samepage}

\noindent{\bf \sc Synopsis}

\begin{alltt}
function [varargout] = IDASolve(tout,itask) 
\end{alltt}

\end{samepage}

%% description -------------------

\noindent{\bf \sc Description}

\begin{alltt}
IDASolve integrates the DAE.

   Usage: [STATUS, T, Y] = IDASolve ( TOUT, ITASK ) 
          [STATUS, T, Y, YQ] = IDASolve  (TOUT, ITASK )
          [STATUS, T, Y, YS] = IDASolve ( TOUT, ITASK )
          [STATUS, T, Y, YQ, YS] = IDASolve ( TOUT, ITASK )

   If ITASK is 'Normal', then the solver integrates from its current internal 
   T value to a point at or beyond TOUT, then interpolates to T = TOUT and returns 
   Y(TOUT). If ITASK is 'OneStep', then the solver takes one internal time step 
   and returns in Y the solution at the new internal time. In this case, TOUT 
   is used only during the first call to IDASolve to determine the direction of 
   integration and the rough scale of the problem. In either case, the time 
   reached by the solver is returned in T.

   If quadratures were computed (see IDAQuadInit), IDASolve will return their
   values at T in the vector YQ.

   If sensitivity calculations were enabled (see IDASensInit), IDASolve will 
   return their values at T in the matrix YS. Each row in the matrix YS
   represents the sensitivity vector with respect to one of the problem parameters.

   In ITASK =' Normal' mode, to obtain solutions at specific times T0,T1,...,TFINAL
   (all increasing or all decreasing) use TOUT = [T0 T1  ... TFINAL]. In this case
   the output arguments Y and YQ are matrices, each column representing the solution
   vector at the corresponding time returned in the vector T. If computed, the 
   sensitivities are eturned in the 3-dimensional array YS, with YS(:,:,I) representing
   the sensitivity vectors at the time T(I).

   On return, STATUS is one of the following:
     0: IDASolve succeeded and no roots were found.
     1: IDASolve succeded and returned at tstop.
     2: IDASolve succeeded, and found one or more roots. 
    -1: Illegal attempt to call before IDAMalloc
    -2: One of the inputs to IDASolve is illegal. This includes the situation 
        when a component of the error weight vectors becomes &lt; 0 during internal 
        time-stepping.
    -4: The solver took mxstep internal steps but could not reach TOUT. The 
        default value for mxstep is 500.
    -5: The solver could not satisfy the accuracy demanded by the user for some 
        internal step.
    -6: Error test failures occurred too many times (MXNEF = 7) during one internal 
        time step 
        or occurred with |h| = hmin.
    -7: Convergence test failures occurred too many times (MXNCF = 10) during one 
        internal time step or occurred with |h| = hmin.
    -9: The linear solver's setup routine failed in an unrecoverable manner.
   -10: The linear solver's solve routine failed in an unrecoverable manner.


   See also IDASetOptions, IDAGetStats
\end{alltt}






\vspace{0.1in}