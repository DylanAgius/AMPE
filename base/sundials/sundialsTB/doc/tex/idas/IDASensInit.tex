%% foo.tex
\begin{samepage}
\hrule
\begin{center}
\phantomsection
{\large \verb!IDASensInit!}
\label{p:IDASensInit}
\index{IDASensInit}
\end{center}
\hrule\vspace{0.1in}

%% one line -------------------

\noindent{\bf \sc Purpose}

\begin{alltt}
IDASensInit allocates and initializes memory for FSA with IDAS.
\end{alltt}

\end{samepage}


%% definition  -------------------

\begin{samepage}

\noindent{\bf \sc Synopsis}

\begin{alltt}
function [] = IDASensInit(Ns,fctS,yyS0,ypS0,options) 
\end{alltt}

\end{samepage}

%% description -------------------

\noindent{\bf \sc Description}

\begin{alltt}
IDASensInit allocates and initializes memory for FSA with IDAS.

   Usage: IDASensInit ( NS, SFUN, YYS0, YPS0 [, OPTIONS ] ) 

   NS       is the number of parameters with respect to which sensitivities
            are desired
   SFUN     is a function defining the residual of the sensitivity DAEs
            fS(t,y,yp,yS,ypS).
   YYS0, YPS0   Initial conditions for sensitivity variables.
            YYS0 and YPS0 must be matrices with N rows and Ns columns, where N is 
            the problem dimension and Ns the number of sensitivity systems.
   OPTIONS  is an (optional) set of FSA options, created with
            the IDASetFSAOptions function. 

   See also IDASensSetOptions, IDAInit, IDASensResFn
\end{alltt}






\vspace{0.1in}