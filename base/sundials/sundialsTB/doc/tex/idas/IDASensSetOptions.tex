%% foo.tex
\begin{samepage}
\hrule
\begin{center}
\phantomsection
{\large \verb!IDASensSetOptions!}
\label{p:IDASensSetOptions}
\index{IDASensSetOptions}
\end{center}
\hrule\vspace{0.1in}

%% one line -------------------

\noindent{\bf \sc Purpose}

\begin{alltt}
IDASensSetOptions creates an options structure for FSA with IDAS.
\end{alltt}

\end{samepage}


%% definition  -------------------

\begin{samepage}

\noindent{\bf \sc Synopsis}

\begin{alltt}
function options = IDASensSetOptions(varargin) 
\end{alltt}

\end{samepage}

%% description -------------------

\noindent{\bf \sc Description}

\begin{alltt}
IDASensSetOptions creates an options structure for FSA with IDAS.

   Usage: OPTIONS = IDASensSetOptions('NAME1',VALUE1,'NAME2',VALUE2,...)
          OPTIONS = IDASensSetOptions(OLDOPTIONS,'NAME1',VALUE1,...)

   OPTIONS = IDASensSetOptions('NAME1',VALUE1,'NAME2',VALUE2,...) creates 
   a IDAS options structure OPTIONS in which the named properties have 
   the specified values. Any unspecified properties have default values. 
   It is sufficient to type only the leading characters that uniquely 
   identify the property. Case is ignored for property names. 
   
   OPTIONS = IDASensSetOptions(OLDOPTIONS,'NAME1',VALUE1,...) alters an 
   existing options structure OLDOPTIONS.
   
   IDASensSetOptions with no input arguments displays all property names 
   and their possible values.
   
IDASensSetOptions properties
(See also the IDAS User Guide)

method - FSA solution method [ 'Simultaneous' | {'Staggered'} ]
   Specifies the FSA method for treating the nonlinear system solution for
   sensitivity variables. In the simultaneous case, the nonlinear systems
   for states and all sensitivities are solved simultaneously. In the
   Staggered case, the nonlinear system for states is solved first and then
   the nonlinear systems for all sensitivities are solved at the same time.
ParamField - Problem parameters  [ string ]
   Specifies the name of the field in the user data structure (specified through
   the 'UserData' field with IDASetOptions) in which the nominal values of the problem
   parameters are stored. This property is used only if  IDAS will use difference
   quotient approximations to the sensitivity residuals (see IDASensResFn).
ParamList - Parameters with respect to which FSA is performed [ integer vector ]
   Specifies a list of Ns parameters with respect to which sensitivities are to
   be computed. This property is used only if IDAS will use difference-quotient
   approximations to the sensitivity residuals. Its length must be Ns,
   consistent with the number of columns of yS0 (see IDASensInit).
ParamScales - Order of magnitude for problem parameters [ vector ]
   Provides order of magnitude information for the parameters with respect to
   which sensitivities are computed. This information is used if IDAS
   approximates the sensitivity residuals or if IDAS estimates integration
   tolerances for the sensitivity variables (see RelTol and AbsTol).
RelTol - Relative tolerance for sensitivity variables [ positive scalar ]
   Specifies the scalar relative tolerance for the sensitivity variables.
   See also AbsTol.
AbsTol - Absolute tolerance for sensitivity variables [ row-vector or matrix ]
   Specifies the absolute tolerance for sensitivity variables. AbsTol must be
   either a row vector of dimension Ns, in which case each of its components is
   used as a scalar absolute tolerance for the coresponding sensitivity vector,
   or a N x Ns matrix, in which case each of its columns is used as a vector
   of absolute tolerances for the corresponding sensitivity vector.
   By default, IDAS estimates the integration tolerances for sensitivity
   variables, based on those for the states and on the order of magnitude
   information for the problem parameters specified through ParamScales.
ErrControl - Error control strategy for sensitivity variables [ false | {true} ]
   Specifies whether sensitivity variables are included in the error control test.
   Note that sensitivity variables are always included in the nonlinear system
   convergence test.
DQtype - Type of DQ approx. of the sensi. RHS [{Centered} | Forward ]
   Specifies whether to use centered (second-order) or forward (first-order)
   difference quotient approximations of the sensitivity eqation residuals.
   This property is used only if a user-defined sensitivity residual function
   was not provided.
DQparam - Cut-off parameter for the DQ approx. of the sensi. RES [ scalar | {0.0} ]
   Specifies the value which controls the selection of the difference-quotient
   scheme used in evaluating the sensitivity residuals (switch between
   simultaneous or separate evaluations of the two components in the sensitivity
   right-hand side). The default value 0.0 indicates the use of simultaenous approximation
   exclusively (centered or forward, depending on the value of DQtype.
   For DQparam &gt;= 1, IDAS uses a simultaneous approximation if the estimated
   DQ perturbations for states and parameters are within a factor of DQparam,
   and separate approximations otherwise. Note that a value DQparam &lt; 1
   will inhibit switching! This property is used only if a user-defined sensitivity
   residual function was not provided.

   See also
        IDASensInit, IDASensReInit
\end{alltt}






\vspace{0.1in}