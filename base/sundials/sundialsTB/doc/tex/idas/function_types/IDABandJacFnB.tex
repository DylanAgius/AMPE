%% # foo.tex
\begin{samepage}
\hrule
\begin{center}
\phantomsection
{\large \verb!IDABandJacFnB!}
\label{p:IDABandJacFnB}
\index{IDABandJacFnB}
\end{center}
\hrule\vspace{0.1in}

%% one line -------------------

\noindent{\bf \sc Purpose}

\begin{alltt}
IDABandJacFnB - type for banded Jacobian function for backward problems.
\end{alltt}

\end{samepage}


%% definition  -------------------

\begin{samepage}

\noindent{\bf \sc Synopsis}

\begin{alltt}
This is a script file. 
\end{alltt}

\end{samepage}

%% description -------------------

\noindent{\bf \sc Description}

\begin{alltt}
IDABandJacFnB - type for banded Jacobian function for backward problems.

   The function BJACFUNB must be defined either as
        FUNCTION [JB, FLAG] = BJACFUNB(T, YY, YP, YYB, YPB, RRB, CJB)
   or as
        FUNCTION [JB,FLAG,NEW_DATA] = BJACFUNB(T,YY,YP,YYB,YPB,RRB,CJB)
   depending on whether a user data structure DATA was specified in
   IDAMalloc. In either case, it must return the matrix JB, the
   Jacobian (dfB/dyyB + cjB*dfB/dypB)of fB(t,y,yB). The input argument
   RRB contains the current value of f(t,yy,yp,yyB,ypB).

   The function BJACFUNB must set FLAG=0 if successful, FLAG&lt;0 if an
   unrecoverable failure occurred, or FLAG&gt;0 if a recoverable error
   occurred.

   See also IDASetOptions

   See the IDAS user guide for more information on the structure of
   a banded Jacobian.

   NOTE: BJACFUNB is specified through the property JacobianFn to
   IDASetOptions and is used only if the property LinearSolver
   was set to 'Band'.
\end{alltt}






\vspace{0.1in}