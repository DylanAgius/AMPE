%% # foo.tex
\begin{samepage}
\hrule
\begin{center}
\phantomsection
{\large \verb!IDAResFnB!}
\label{p:IDAResFnB}
\index{IDAResFnB}
\end{center}
\hrule\vspace{0.1in}

%% one line -------------------

\noindent{\bf \sc Purpose}

\begin{alltt}
IDAResFnb - type for residual function for backward problems
\end{alltt}

\end{samepage}


%% definition  -------------------

\begin{samepage}

\noindent{\bf \sc Synopsis}

\begin{alltt}
This is a script file. 
\end{alltt}

\end{samepage}

%% description -------------------

\noindent{\bf \sc Description}

\begin{alltt}
IDAResFnb - type for residual function for backward problems

   The function DAEFUNB must be defined either as
        FUNCTION [RB, FLAG] = DAEFUNB(T, YY, YP, YYB, YPB)
   or as
        FUNCTION [RB, FLAG, NEW_DATA] = DAEFUNB(T, YY, YP, YYB, YPB, DATA)
   depending on whether a user data structure DATA was specified in
   IDAMalloc. In either case, it must return the vector RB
   corresponding to fB(t,yy,yp,yyB,ypB).

   The function DAEFUNB must set FLAG=0 if successful, FLAG&lt;0 if an
   unrecoverable failure occurred, or FLAG&gt;0 if a recoverable error
   occurred.

   See also IDAInitB, IDARhsFn
\end{alltt}






\vspace{0.1in}