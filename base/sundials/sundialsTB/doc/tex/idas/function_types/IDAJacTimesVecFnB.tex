%% # foo.tex
\begin{samepage}
\hrule
\begin{center}
\phantomsection
{\large \verb!IDAJacTimesVecFnB!}
\label{p:IDAJacTimesVecFnB}
\index{IDAJacTimesVecFnB}
\end{center}
\hrule\vspace{0.1in}

%% one line -------------------

\noindent{\bf \sc Purpose}

\begin{alltt}
IDAJacTimesVecFn - type for Jacobian times vector function for backward problems.
\end{alltt}

\end{samepage}


%% definition  -------------------

\begin{samepage}

\noindent{\bf \sc Synopsis}

\begin{alltt}
This is a script file. 
\end{alltt}

\end{samepage}

%% description -------------------

\noindent{\bf \sc Description}

\begin{alltt}
IDAJacTimesVecFn - type for Jacobian times vector function for backward problems.

   The function JTVFUNB must be defined either as
        FUNCTION [JVB,FLAG] = JTVFUNB(T,YY,YP,YYB,YPB,RRB,VB,CJB)
   or as
        FUNCTION [JVB,FLAG,NEW_DATA] = JTVFUNB(T,YY,YP,YYB,YPB,RRB,VB,CJB,DATA)
   depending on whether a user data structure DATA was specified in
   IDAMalloc. In either case, it must return the vector JVB, the
   product of the Jacobian (dfB/dyyB + cj * dfB/dypB) and a vector
   vB. The input argument RRB contains the current value of f(t,yy,yp,yyB,ypB).

   The function JTVFUNB must set FLAG=0 if successful, or FLAG~=0 if
   a failure occurred.

   See also IDASetOptions

   NOTE: JTVFUNB is specified through the property JacobianFn to IDASetOptions
   and is used only if the property LinearSolver was set to 'GMRES', 'BiCGStab',
   or 'TFQMR'.
\end{alltt}






\vspace{0.1in}