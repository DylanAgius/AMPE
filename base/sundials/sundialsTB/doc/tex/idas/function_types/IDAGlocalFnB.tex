%% # foo.tex
\begin{samepage}
\hrule
\begin{center}
\phantomsection
{\large \verb!IDAGlocalFnB!}
\label{p:IDAGlocalFnB}
\index{IDAGlocalFnB}
\end{center}
\hrule\vspace{0.1in}

%% one line -------------------

\noindent{\bf \sc Purpose}

\begin{alltt}
IDAGlocalFnB - type for RES approximation function (BBDPre) for backward problems.
\end{alltt}

\end{samepage}


%% definition  -------------------

\begin{samepage}

\noindent{\bf \sc Synopsis}

\begin{alltt}
This is a script file. 
\end{alltt}

\end{samepage}

%% description -------------------

\noindent{\bf \sc Description}

\begin{alltt}
IDAGlocalFnB - type for RES approximation function (BBDPre) for backward problems.

   The function GLOCFUNB must be defined either as
        FUNCTION [GLOCB, FLAG] = GLOCFUNB(T,YY,YP,YYB,YPB)
   or as
        FUNCTION [GLOCB, FLAG, NEW_DATA] = GLOCFUNB(T,YY,YP,YYB,YPB,DATA)
   depending on whether a user data structure DATA was specified in
   IDAMalloc. In either case, it must return the vector GLOCB
   corresponding to an approximation to fB(t,yy,yp,yyB,ypB).

   The function GLOCFUNB must set FLAG=0 if successful, FLAG&lt;0 if an
   unrecoverable failure occurred, or FLAG&gt;0 if a recoverable error
   occurred.

   See also IDAGcommFnB, IDAGlocalFn, IDASetOptions

   NOTE: GLOCFUN and GLOCFUNB are specified through the GlocalFn property
   in IDASetOptions and are used only if the property PrecModule
   is set to 'BBDPre'.
\end{alltt}






\vspace{0.1in}