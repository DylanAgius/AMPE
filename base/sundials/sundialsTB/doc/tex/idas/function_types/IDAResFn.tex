%% # foo.tex
\begin{samepage}
\hrule
\begin{center}
\phantomsection
{\large \verb!IDAResFn!}
\label{p:IDAResFn}
\index{IDAResFn}
\end{center}
\hrule\vspace{0.1in}

%% one line -------------------

\noindent{\bf \sc Purpose}

\begin{alltt}
IDAResFn - type for residual function
\end{alltt}

\end{samepage}


%% definition  -------------------

\begin{samepage}

\noindent{\bf \sc Synopsis}

\begin{alltt}
This is a script file. 
\end{alltt}

\end{samepage}

%% description -------------------

\noindent{\bf \sc Description}

\begin{alltt}
IDAResFn - type for residual function

   The function DAEFUN must be defined as 
        FUNCTION [R, FLAG] = DAEFUN(T, YY, YP)
   and must return a vector R corresponding to f(t,yy,yp).
   If a user data structure DATA was specified in IDAMalloc, then
   DAEFUN must be defined as
        FUNCTION [R, FLAG, NEW_DATA] = DAEFUN(T, YY, YP, DATA)
   If the local modifications to the user data structure are needed 
   in other user-provided functions then, besides setting the vector YD,
   the DAEFUN function must also set NEW_DATA. Otherwise, it should set
   NEW_DATA=[] (do not set NEW_DATA = DATA as it would lead to
   unnecessary copying).

   The function DAEFUN must set FLAG=0 if successful, FLAG&lt;0 if an
   unrecoverable failure occurred, or FLAG&gt;0 if a recoverable error
   occurred.

   See also IDAInit
\end{alltt}






\vspace{0.1in}