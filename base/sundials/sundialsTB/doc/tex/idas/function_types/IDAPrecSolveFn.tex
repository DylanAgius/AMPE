%% # foo.tex
\begin{samepage}
\hrule
\begin{center}
\phantomsection
{\large \verb!IDAPrecSolveFn!}
\label{p:IDAPrecSolveFn}
\index{IDAPrecSolveFn}
\end{center}
\hrule\vspace{0.1in}

%% one line -------------------

\noindent{\bf \sc Purpose}

\begin{alltt}
IDAPrecSolveFn - type for preconditioner solve function.
\end{alltt}

\end{samepage}


%% definition  -------------------

\begin{samepage}

\noindent{\bf \sc Synopsis}

\begin{alltt}
This is a script file. 
\end{alltt}

\end{samepage}

%% description -------------------

\noindent{\bf \sc Description}

\begin{alltt}
IDAPrecSolveFn - type for preconditioner solve function.

   The user-supplied preconditioner solve function PSOLFUN
   is to solve a linear system P z = r, where P is the
   preconditioner matrix.

   The function PSOLFUN must be defined as 
        FUNCTION [Z, FLAG] = PSOLFUN(T,YY,YP,RR,R)
   and must return a vector Z containing the solution of Pz=r.
   If PSOLFUN was successful, it must return FLAG=0. For a recoverable 
   error (in which case the step will be retried) it must set FLAG to a 
   positive value. If an unrecoverable error occurs, it must set FLAG
   to a negative value, in which case the integration will be halted.
   The input argument RR contains the current value of f(t,yy,yp).

   If a user data structure DATA was specified in IDAMalloc, then
   PSOLFUN must be defined as
        FUNCTION [Z, FLAG, NEW_DATA] = PSOLFUN(T,YY,YP,RR,R,DATA)
   If the local modifications to the user data structure are needed in
   other user-provided functions then, besides setting the vector Z and
   the flag FLAG, the PSOLFUN function must also set NEW_DATA. Otherwise,
   it should set NEW_DATA=[] (do not set NEW_DATA = DATA as it would
   lead to unnecessary copying).

   See also IDAPrecSetupFn, IDASetOptions

   NOTE: PSOLFUN and PSOLFUNB are specified through the property
   PrecSolveFn to IDASetOptions and are used only if the property
   LinearSolver was set to 'GMRES', 'BiCGStab', or 'TFQMR'.
\end{alltt}






\vspace{0.1in}