%% # foo.tex
\begin{samepage}
\hrule
\begin{center}
\phantomsection
{\large \verb!IDARootFn!}
\label{p:IDARootFn}
\index{IDARootFn}
\end{center}
\hrule\vspace{0.1in}

%% one line -------------------

\noindent{\bf \sc Purpose}

\begin{alltt}
IDARootFn - type for user provided root-finding function.
\end{alltt}

\end{samepage}


%% definition  -------------------

\begin{samepage}

\noindent{\bf \sc Synopsis}

\begin{alltt}
This is a script file. 
\end{alltt}

\end{samepage}

%% description -------------------

\noindent{\bf \sc Description}

\begin{alltt}
IDARootFn - type for user provided root-finding function.

   The function ROOTFUN must be defined as 
        FUNCTION [G, FLAG] = ROOTFUN(T,YY,YP)
   and must return a vector G corresponding to g(t,yy,yp).
   If a user data structure DATA was specified in IDAMalloc, then
   ROOTFUN must be defined as
        FUNCTION [G, FLAG, NEW_DATA] = ROOTFUN(T,YY,YP,DATA)
   If the local modifications to the user data structure are needed in
   other user-provided functions then, besides setting the vector G,
   the ROOTFUN function must also set NEW_DATA. Otherwise, it should 
   set NEW_DATA=[] (do not set NEW_DATA = DATA as it would lead to 
   unnecessary copying).

   The function ROOTFUN must set FLAG=0 if successful, or FLAG~=0 if
   a failure occurred.

   See also IDASetOptions

   NOTE: ROOTFUN is specified through the RootsFn property in
   IDASetOptions and is used only if the property NumRoots is a
   positive integer.
\end{alltt}






\vspace{0.1in}