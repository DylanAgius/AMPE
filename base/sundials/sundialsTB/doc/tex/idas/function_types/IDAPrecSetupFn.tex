%% # foo.tex
\begin{samepage}
\hrule
\begin{center}
\phantomsection
{\large \verb!IDAPrecSetupFn!}
\label{p:IDAPrecSetupFn}
\index{IDAPrecSetupFn}
\end{center}
\hrule\vspace{0.1in}

%% one line -------------------

\noindent{\bf \sc Purpose}

\begin{alltt}
IDAPrecSetupFn - type for preconditioner setup function.
\end{alltt}

\end{samepage}


%% definition  -------------------

\begin{samepage}

\noindent{\bf \sc Synopsis}

\begin{alltt}
This is a script file. 
\end{alltt}

\end{samepage}

%% description -------------------

\noindent{\bf \sc Description}

\begin{alltt}
IDAPrecSetupFn - type for preconditioner setup function.

   The user-supplied preconditioner setup function PSETFUN and
   the user-supplied preconditioner solve function PSOLFUN
   together must define a preconditoner matrix P which is an 
   approximation to the Newton matrix M = J_yy - cj*J_yp.  
   Here J_yy = df/dyy, J_yp = df/dyp, and cj is a scalar proportional 
   to the integration step size h.  The solution of systems P z = r,
   is to be carried out by the PrecSolve function, and PSETFUN
   is to do any necessary setup operations.

   The user-supplied preconditioner setup function PSETFUN
   is to evaluate and preprocess any Jacobian-related data
   needed by the preconditioner solve function PSOLFUN.
   This might include forming a crude approximate Jacobian,
   and performing an LU factorization on the resulting
   approximation to M.  This function will not be called in
   advance of every call to PSOLFUN, but instead will be called
   only as often as necessary to achieve convergence within the
   Newton iteration.  If the PSOLFUN function needs no
   preparation, the PSETFUN function need not be provided.

   For greater efficiency, the PSETFUN function may save
   Jacobian-related data and reuse it, rather than generating it
   from scratch.  In this case, it should use the input flag JOK
   to decide whether to recompute the data, and set the output
   flag JCUR accordingly.

   Each call to the PSETFUN function is preceded by a call to
   DAEFUN with the same (t,yy,yp) arguments.  Thus the PSETFUN
   function can use any auxiliary data that is computed and
   saved by the DAEFUN function and made accessible to PSETFUN.

   The function PSETFUN must be defined as 
        FUNCTION FLAG = PSETFUN(T,YY,YP,RR,CJ)
   If successful, it must return FLAG=0. For a recoverable error (in    
   which case the setup will be retried) it must set FLAG to a positive
   integer value. If an unrecoverable error occurs, it must set FLAG
   to a negative value, in which case the integration will be halted.
   The input argument RR contains the current value of f(t,yy,yp).

   If a user data structure DATA was specified in IDAMalloc, then
   PSETFUN must be defined as
        FUNCTION [FLAG,NEW_DATA] = PSETFUN(T,YY,YP,RR,CJ,DATA)
   If the local modifications to the user data structure are needed in
   other user-provided functions then, besides setting the flags JCUR
   and FLAG, the PSETFUN function must also set NEW_DATA. Otherwise, it 
   should set NEW_DATA=[] (do not set NEW_DATA = DATA as it would lead
   to unnecessary copying).

   See also IDAPrecSolveFn, IDASetOptions

   NOTE: PSETFUN and PSETFUNB are specified through the property
   PrecSetupFn to IDASetOptions and are used only if the property
   LinearSolver was set to 'GMRES', 'BiCGStab', or 'TFQMR'.
\end{alltt}






\vspace{0.1in}