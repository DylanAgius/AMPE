%% # foo.tex
\begin{samepage}
\hrule
\begin{center}
\phantomsection
{\large \verb!IDAGcommFnB!}
\label{p:IDAGcommFnB}
\index{IDAGcommFnB}
\end{center}
\hrule\vspace{0.1in}

%% one line -------------------

\noindent{\bf \sc Purpose}

\begin{alltt}
IDAGcommFnB - type for communication function (BBDPre) for backward problems.
\end{alltt}

\end{samepage}


%% definition  -------------------

\begin{samepage}

\noindent{\bf \sc Synopsis}

\begin{alltt}
This is a script file. 
\end{alltt}

\end{samepage}

%% description -------------------

\noindent{\bf \sc Description}

\begin{alltt}
IDAGcommFnB - type for communication function (BBDPre) for backward problems.

   The function GCOMFUNB must be defined either as
        FUNCTION FLAG = GCOMFUNB(T, YY, YP, YYB, YPB)
   or as
        FUNCTION [FLAG, NEW_DATA] = GCOMFUNB(T, YY, YP, YYB, YPB, DATA)
   depending on whether a user data structure DATA was specified in
   IDAMalloc. 

   The function GCOMFUNB must set FLAG=0 if successful, FLAG&lt;0 if an
   unrecoverable failure occurred, or FLAG&gt;0 if a recoverable error
   occurred.

   See also IDAGlocalFnB, IDAGcommFn, IDASetOptions

   NOTES:
     GCOMFUNB is specified through the GcommFn property in IDASetOptions
     and is used only if the property PrecModule is set to 'BBDPre'.

     Each call to GCOMFUNB is preceded by a call to the residual function
     DAEFUN with the same arguments T, YY, YP and YYB and YPB.
     Thus GCOMFUNB can omit any communication done by DAEFUNB if relevant
     to the evaluation of G by GLOCFUNB. If all necessary communication
     was done by DAEFUNB, GCOMFUNB need not be provided.
\end{alltt}






\vspace{0.1in}