%% # foo.tex
\begin{samepage}
\hrule
\begin{center}
\phantomsection
{\large \verb!IDASensResFn!}
\label{p:IDASensResFn}
\index{IDASensResFn}
\end{center}
\hrule\vspace{0.1in}

%% one line -------------------

\noindent{\bf \sc Purpose}

\begin{alltt}
IDASensRhsFn - type for user provided sensitivity RHS function.
\end{alltt}

\end{samepage}


%% definition  -------------------

\begin{samepage}

\noindent{\bf \sc Synopsis}

\begin{alltt}
This is a script file. 
\end{alltt}

\end{samepage}

%% description -------------------

\noindent{\bf \sc Description}

\begin{alltt}
IDASensRhsFn - type for user provided sensitivity RHS function.

   The function DAESFUN must be defined as 
        FUNCTION [RS, FLAG] = DAESFUN(T,YY,YP,YYS,YPS)
   and must return a matrix RS corresponding to fS(t,yy,yp,yyS,ypS).
   If a user data structure DATA was specified in IDAMalloc, then
   DAESFUN must be defined as
        FUNCTION [RS, FLAG, NEW_DATA] = DAESFUN(T,YY,YP,YYS,YPS,DATA)
   If the local modifications to the user data structure are needed in
   other user-provided functions then, besides setting the matrix YSD,
   the ODESFUN function must also set NEW_DATA. Otherwise, it should
   set NEW_DATA=[] (do not set NEW_DATA = DATA as it would lead to 
   unnecessary copying).

   The function DAESFUN must set FLAG=0 if successful, FLAG&lt;0 if an
   unrecoverable failure occurred, or FLAG&gt;0 if a recoverable error
   occurred.

   See also IDASetFSAOptions

   NOTE: DAESFUN is specified through the property FSAResFn to
         IDASetFSAOptions.
\end{alltt}






\vspace{0.1in}