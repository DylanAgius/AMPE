%% foo.tex
\begin{samepage}
\hrule
\begin{center}
\phantomsection
{\large \verb!IDAQuadInitB!}
\label{p:IDAQuadInitB}
\index{IDAQuadInitB}
\end{center}
\hrule\vspace{0.1in}

%% one line -------------------

\noindent{\bf \sc Purpose}

\begin{alltt}
IDAQuadInitB allocates and initializes memory for backward quadrature integration.
\end{alltt}

\end{samepage}


%% definition  -------------------

\begin{samepage}

\noindent{\bf \sc Synopsis}

\begin{alltt}
function IDAQuadInitB(idxB, fctQB, yQB0, optionsB) 
\end{alltt}

\end{samepage}

%% description -------------------

\noindent{\bf \sc Description}

\begin{alltt}
IDAQuadInitB allocates and initializes memory for backward quadrature integration.

   Usage: IDAQuadInitB ( IDXB, QBFUN, YQB0 [, OPTIONS ] ) 

   IDXB     is the index of the backward problem, returned by
            IDAInitB.
   QBFUN    is a function defining the righ-hand sides of the
            backward ODEs yQB' = fQB(t,y,yB).
   YQB0     is the final conditions vector yQB(tB0).
   OPTIONS  is an (optional) set of QUAD options, created with
            the IDASetQuadOptions function. 

   See also: IDAInitB, IDASetQuadOptions, IDAQuadRhsFnB
\end{alltt}






\vspace{0.1in}