%% foo.tex
\begin{samepage}
\hrule
\begin{center}
\phantomsection
{\large \verb!IDASetOptions!}
\label{p:IDASetOptions}
\index{IDASetOptions}
\end{center}
\hrule\vspace{0.1in}

%% one line -------------------

\noindent{\bf \sc Purpose}

\begin{alltt}
IDASetOptions creates an options structure for IDAS.
\end{alltt}

\end{samepage}


%% definition  -------------------

\begin{samepage}

\noindent{\bf \sc Synopsis}

\begin{alltt}
function options = IDASetOptions(varargin) 
\end{alltt}

\end{samepage}

%% description -------------------

\noindent{\bf \sc Description}

\begin{alltt}
IDASetOptions creates an options structure for IDAS.

   Usage: OPTIONS = IDASetOptions('NAME1',VALUE1,'NAME2',VALUE2,...)
          OPTIONS = IDASetOptions(OLDOPTIONS,'NAME1',VALUE1,...)

   OPTIONS = IDASetOptions('NAME1',VALUE1,'NAME2',VALUE2,...) creates 
   a IDAS options structure OPTIONS in which the named properties have 
   the specified values. Any unspecified properties have default values. 
   It is sufficient to type only the leading characters that uniquely 
   identify the property. Case is ignored for property names. 
   
   OPTIONS = IDASetOptions(OLDOPTIONS,'NAME1',VALUE1,...) alters an 
   existing options structure OLDOPTIONS.
   
   IDASetOptions with no input arguments displays all property names 
   and their possible values.
   
IDASetOptions properties
(See also the IDAS User Guide)

UserData - User data passed unmodified to all functions [ empty ]
   If UserData is not empty, all user provided functions will be
   passed the problem data as their last input argument. For example,
   the RES function must be defined as R = DAEFUN(T,YY,TP,DATA).

RelTol - Relative tolerance [ positive scalar | {1e-4} ]
   RelTol defaults to 1e-4 and is applied to all components of the solution
   vector. See AbsTol.
AbsTol - Absolute tolerance [ positive scalar or vector | {1e-6} ]
   The relative and absolute tolerances define a vector of error weights
   with components
     ewt(i) = 1/(RelTol*|y(i)| + AbsTol)    if AbsTol is a scalar
     ewt(i) = 1/(RelTol*|y(i)| + AbsTol(i)) if AbsTol is a vector
   This vector is used in all error and convergence tests, which
   use a weighted RMS norm on all error-like vectors v:
     WRMSnorm(v) = sqrt( (1/N) sum(i=1..N) (v(i)*ewt(i))^2 ),
   where N is the problem dimension.
MaxNumSteps - Maximum number of steps [positive integer | {500}]
   IDASolve will return with an error after taking MaxNumSteps internal steps
   in its attempt to reach the next output time.
InitialStep - Suggested initial stepsize [ positive scalar ]
   By default, IDASolve estimates an initial stepsize h0 at the initial time
   t0 as the solution of
     WRMSnorm(h0^2 ydd / 2) = 1
   where ydd is an estimated second derivative of y(t0).
MaxStep - Maximum stepsize [ positive scalar | {inf} ]
   Defines an upper bound on the integration step size.
MaxOrder - Maximum method order [ 1-5 for BDF | {5} ]
   Defines an upper bound on the linear multistep method order.
StopTime - Stopping time [ scalar ]
   Defines a value for the independent variable past which the solution
   is not to proceed.
RootsFn - Rootfinding function [ function ]
   To detect events (roots of functions), set this property to the event
   function. See IDARootFn.
NumRoots - Number of root functions [ integer | {0} ]
   Set NumRoots to the number of functions for which roots are monitored.
   If NumRoots is 0, rootfinding is disabled.

SuppressAlgVars - Suppres algebraic vars. from error test [ on | {off} ]
VariableTypes - Alg./diff. variables [ vector ]
ConstraintTypes - Simple bound constraints [ vector ]

LinearSolver - Linear solver type [{Dense}|Band|GMRES|BiCGStab|TFQMR]
   Specifies the type of linear solver to be used for the Newton nonlinear
   solver. Valid choices are: Dense (direct, dense Jacobian), Band (direct,
   banded Jacobian), GMRES (iterative, scaled preconditioned GMRES),
   BiCGStab (iterative, scaled preconditioned stabilized BiCG), TFQMR
   (iterative, scaled transpose-free QMR).
   The GMRES, BiCGStab, and TFQMR are matrix-free linear solvers.
JacobianFn - Jacobian function [ function ]
   This propeerty is overloaded. Set this value to a function that returns
   Jacobian information consistent with the linear solver used (see Linsolver).
   If not specified, IDAS uses difference quotient approximations.
   For the Dense linear solver, JacobianFn must be of type IDADenseJacFn and
   must return a dense Jacobian matrix. For the Band linear solver, JacobianFn
   must be of type IDABandJacFn and must return a banded Jacobian matrix.
   For the iterative linear solvers, GMRES, BiCGStab, and TFQMR, JacobianFn must
   be of type IDAJacTimesVecFn and must return a Jacobian-vector product.
KrylovMaxDim - Maximum number of Krylov subspace vectors [ integer | {5} ]
   Specifies the maximum number of vectors in the Krylov subspace. This property
   is used only if an iterative linear solver, GMRES, BiCGStab, or TFQMR is used
   (see LinSolver).
GramSchmidtType - Gram-Schmidt orthogonalization [ Classical | {Modified} ]
   Specifies the type of Gram-Schmidt orthogonalization (classical or modified).
   This property is used only if the GMRES linear solver is used (see LinSolver).
PrecModule - Preconditioner module [ BBDPre | {UserDefined} ]
   If PrecModule = 'UserDefined', then the user must provide at least a
   preconditioner solve function (see PrecSolveFn)
   IDAS provides one general-purpose preconditioner module, BBDPre, which can
   be only used with parallel vectors. It provide a preconditioner matrix that
   is block-diagonal with banded blocks. The blocking corresponds to the
   distribution of the dependent variable vector y among the processors.
   Each preconditioner block is generated from the Jacobian of the local part
   (on the current processor) of a given function g(t,y,yp) approximating
   f(t,y,yp) (see GlocalFn). The blocks are generated by a difference quotient
   scheme on each processor independently. This scheme utilizes an assumed
   banded structure with given half-bandwidths, mldq and mudq (specified through
   LowerBwidthDQ and UpperBwidthDQ, respectively). However, the banded Jacobian
   block kept by the scheme has half-bandwiths ml and mu (specified through
   LowerBwidth and UpperBwidth), which may be smaller.
PrecSetupFn - Preconditioner setup function [ function ]
   If PrecType is not 'None', PrecSetupFn specifies an optional function which,
   together with PrecSolve, defines the preconditioner matrix, which must be an
   aproximation to the Newton matrix. PrecSetupFn must be of type IDAPrecSetupFn.
PrecSolveFn - Preconditioner solve function [ function ]
   If PrecType is not 'None', PrecSolveFn specifies a required function which
   must solve a linear system Pz = r, for given r. PrecSolveFn must be of type
   IDAPrecSolveFn.
GlocalFn - Local residual approximation function for BBDPre [ function ]
   If PrecModule is BBDPre, GlocalFn specifies a required function that
   evaluates a local approximation to the DAE residual. GlocalFn must
   be of type IDAGlocFn.
GcommFn - Inter-process communication function for BBDPre [ function ]
   If PrecModule is BBDPre, GcommFn specifies an optional function
   to perform any inter-process communication required for the evaluation of
   GlocalFn. GcommFn must be of type IDAGcommFn.
LowerBwidth - Jacobian/preconditioner lower bandwidth [ integer | {0} ]
   This property is overloaded. If the Band linear solver is used (see LinSolver),
   it specifies the lower half-bandwidth of the band Jacobian approximation.
   If one of the three iterative linear solvers, GMRES, BiCGStab, or TFQMR is used
   (see LinSolver) and if the BBDPre preconditioner module in IDAS is used
   (see PrecModule), it specifies the lower half-bandwidth of the retained
   banded approximation of the local Jacobian block.
   LowerBwidth defaults to 0 (no sub-diagonals).
UpperBwidth - Jacobian/preconditioner upper bandwidth [ integer | {0} ]
   This property is overloaded. If the Band linear solver is used (see LinSolver),
   it specifies the upper half-bandwidth of the band Jacobian approximation.
   If one of the three iterative linear solvers, GMRES, BiCGStab, or TFQMR is used
   (see LinSolver) and if the BBDPre preconditioner module in IDAS is used
   (see PrecModule), it specifies the upper half-bandwidth of the retained
   banded approximation of the local Jacobian block.
   UpperBwidth defaults to 0 (no super-diagonals).
LowerBwidthDQ - BBDPre preconditioner DQ lower bandwidth [ integer | {0} ]
   Specifies the lower half-bandwidth used in the difference-quotient Jacobian
   approximation for the BBDPre preconditioner (see PrecModule).
UpperBwidthDQ - BBDPre preconditioner DQ upper bandwidth [ integer | {0} ]
   Specifies the upper half-bandwidth used in the difference-quotient Jacobian
   approximation for the BBDPre preconditioner (see PrecModule).

MonitorFn - User-provied monitoring function [ function ]
   Specifies a function that is called after each successful integration step.
   This function must have type IDAMonitorFn or IDAMonitorFnB, depending on
   whether these options are for a forward or a backward problem, respectively.
   Sample monitoring functions IDAMonitor and IDAMonitorB are provided
   with IDAS.
MonitorData - User-provied data for the monitoring function [ struct ]
   Specifies a data structure that is passed to the MonitorFn function every time
   it is called.

SensDependent - Backward problem depending on sensitivities [ {false} | true ]
   Specifies whether the backward problem right-hand side depends on
   forward sensitivites. If TRUE, the residual function provided for
   this backward problem must have the appropriate type (see IDAResFnB).


NOTES:

   The properties listed above that can only be used for forward problems
   are: ConstraintTypes, StopTime, RootsFn, and NumRoots.

   The property SensDependent is relevant only for backward problems.

   See also
        IDAInit, IDAReInit, IDAInitB, IDAReInitB
        IDAResFn, IDARootFn
        IDADenseJacFn, IDABandJacFn, IDAJacTimesVecFn
        IDAPrecSetupFn, IDAPrecSolveFn
        IDAGlocalFn, IDAGcommFn
        IDAMonitorFn
        IDAResFnB
        IDADenseJacFnB, IDABandJacFnB, IDAJacTimesVecFnB
        IDAPrecSetupFnB, IDAPrecSolveFnB
        IDAGlocalFnB, IDAGcommFnB
        IDAMonitorFnB
\end{alltt}






\vspace{0.1in}