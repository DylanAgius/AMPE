%% foo.tex
\begin{samepage}
\hrule
\begin{center}
\phantomsection
{\large \verb!IDASolveB!}
\label{p:IDASolveB}
\index{IDASolveB}
\end{center}
\hrule\vspace{0.1in}

%% one line -------------------

\noindent{\bf \sc Purpose}

\begin{alltt}
IDASolveB integrates the backward DAE.
\end{alltt}

\end{samepage}


%% definition  -------------------

\begin{samepage}

\noindent{\bf \sc Synopsis}

\begin{alltt}
function [varargout] = IDASolveB(tout,itask) 
\end{alltt}

\end{samepage}

%% description -------------------

\noindent{\bf \sc Description}

\begin{alltt}
IDASolveB integrates the backward DAE.

   Usage:  [STATUS, T, YB] = IDASolveB ( TOUT, ITASK ) 
           [STATUS, T, YB, YQB] = IDASolveB ( TOUT, ITASK )

   If ITASK is 'Normal', then the solver integrates from its current internal 
   T value to a point at or beyond TOUT, then interpolates to T = TOUT and returns 
   YB(TOUT). If ITASK is 'OneStep', then the solver takes one internal time step 
   and returns in YB the solution at the new internal time. In this case, TOUT 
   is used only during the first call to CVodeB to determine the direction of 
   integration and the rough scale of the problem. In either case, the time 
   reached by the solver is returned in T. 

   If quadratures were computed (see CVodeQuadInitB), CVodeB will return their
   values at T in the vector YQB.

   In ITASK =' Normal' mode, to obtain solutions at specific times T0,T1,...,TFINAL
   (all increasing or all decreasing) use TOUT = [T0 T1  ... TFINAL]. In this case
   the output arguments YB and YQB are matrices, each column representing the solution
   vector at the corresponding time returned in the vector T.

   If more than one backward problem was defined, the return arguments are cell
   arrays, with T{IDXB}, YB{IDXB}, and YQB{IDXB} corresponding to the backward
   problem with index IDXB (as returned by CVodeInitB).

   On return, STATUS is one of the following:
     0: IDASolveB succeeded.
     1: IDASolveB succeded and return at a tstop value (internally set).
    -2: One of the inputs to IDASolveB is illegal.
    -4: The solver took mxstep internal steps but could not reach TOUT. 
        The default value for mxstep is 500.
   -5:  The solver could not satisfy the accuracy demanded by the user for 
        some internal step.
   -6:  Error test failures occurred too many times (MXNEF = 7) during one 
        internal time step or occurred with |h| = hmin.
   -7:  Convergence test failures occurred too many times (MXNCF = 10) during 
        one internal time step or occurred with |h| = hmin.
   -9:  The linear solver's setup routine failed in an unrecoverable manner.
  -10:  The linear solver's solve routine failed in an unrecoverable manner.
  -101: Illegal attempt to call before initializing adjoint sensitivity 
        (see IDAMalloc).
  -104: Illegal attempt to call before IDAMallocB.
  -108: Wrong value for TOUT.

   See also IDASetOptions, IDAGetStatsB
\end{alltt}






\vspace{0.1in}