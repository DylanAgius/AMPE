%% foo.tex
\begin{samepage}
\hrule
\begin{center}
\phantomsection
{\large \verb!N_VMax!}
\label{p:N_VMax}
\index{N_VMax}
\end{center}
\hrule\vspace{0.1in}

%% one line -------------------

\noindent{\bf \sc Purpose}

\begin{alltt}
N_VMax returns the largest element of x
\end{alltt}

\end{samepage}


%% definition  -------------------

\begin{samepage}

\noindent{\bf \sc Synopsis}

\begin{alltt}
function ret = N_VMax(x,comm) 
\end{alltt}

\end{samepage}

%% description -------------------

\noindent{\bf \sc Description}

\begin{alltt}
N_VMax returns the largest element of x

   Usage:  RET = N_VMax ( X [, COMM] )

If COMM is not present, N_VMax returns the maximum value of 
the local portion of X. Otherwise, it returns the global
maximum value.
\end{alltt}





 
%% source -------------------

\noindent{\bf \sc Source Code}

\begin{lstlisting}[linerange={1-1,9-26}]
function ret = N_VMax(x,comm)
%N_VMax returns the largest element of x
%
%   Usage:  RET = N_VMax ( X [, COMM] )
%
%If COMM is not present, N_VMax returns the maximum value of 
%the local portion of X. Otherwise, it returns the global
%maximum value.

% Radu Serban <radu@llnl.gov>
% Copyright (c) 2005, The Regents of the University of California.
% $Revision: 1.1 $Date: 2006/01/06 19:00:10 $

if nargin == 1
  
  ret = max(x);
  
else
  
  lmax = max(x);
  gmax = 0.0;
  MPI_Allreduce(lmax,gmax,'MAX',comm);
  ret = gmax;
  
end\end{lstlisting}

\vspace{0.1in}