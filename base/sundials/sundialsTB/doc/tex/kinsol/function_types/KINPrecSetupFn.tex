%% # foo.tex
\begin{samepage}
\hrule
\begin{center}
\phantomsection
{\large \verb!KINPrecSetupFn!}
\label{p:KINPrecSetupFn}
\index{KINPrecSetupFn}
\end{center}
\hrule\vspace{0.1in}

%% one line -------------------

\noindent{\bf \sc Purpose}

\begin{alltt}
KINPrecSetupFn - type for user provided preconditioner setup function.
\end{alltt}

\end{samepage}


%% definition  -------------------

\begin{samepage}

\noindent{\bf \sc Synopsis}

\begin{alltt}
This is a script file. 
\end{alltt}

\end{samepage}

%% description -------------------

\noindent{\bf \sc Description}

\begin{alltt}
KINPrecSetupFn - type for user provided preconditioner setup function.

   The user-supplied preconditioner setup subroutine should compute 
   the right-preconditioner matrix P used to form the scaled preconditioned 
   linear system:

   (Df*J(y)*(P^-1)*(Dy^-1)) * (Dy*P*x) = Df*(-F(y))

   where Dy and Df denote the diagonal scaling matrices whose diagonal elements 
   are stored in the vectors YSCALE and FSCALE, respectively.

   The preconditioner setup routine (referenced by iterative linear
   solver modules via pset (type KINSpilsPrecSetupFn)) will not be
   called prior to every call made to the psolve function, but will
   instead be called only as often as necessary to achieve convergence
   of the Newton iteration.

   NOTE: If the PRECSOLVE function requires no preparation, then a
   preconditioner setup function need not be given.

   The function PSETFUN must be defined as 
        FUNCTION FLAG = PSETFUN(Y, YSCALE, FY, FSCALE)
   The input argument FY contains the current value of f(y), while YSCALE
   and FSCALE are the scaling vectors for solution and system function,
   respectively (as passed to KINSol)

   If a user data structure DATA was specified in KINMalloc, then
   PSETFUN must be defined as
        FUNCTION [FLAG, NEW_DATA] = PSETFUN(Y, YSCALE, FY, FSCALE, DATA)
   If the local modifications to the user data structure are needed in
   other user-provided functions then, besides setting the flag FLAG,
   the PSETFUN function must also set NEW_DATA. Otherwise, it should 
   set NEW_DATA=[] (do not set NEW_DATA = DATA as it would lead
   to unnecessary copying).

   If successful, PSETFUN must return FLAG=0. For a recoverable error (in    
   which case the setup will be retried) it must set FLAG to a positive
   integer value. If an unrecoverable error occurs, it must set FLAG
   to a negative value, in which case the solver will halt.

   See also KINPrecSolveFn, KINSetOptions, KINSol

   NOTE: PSETFUN is specified through the property PrecSetupFn to KINSetOptions
   and is used only if the property LinearSolver was set to 'GMRES' or 'BiCGStab'.
\end{alltt}






\vspace{0.1in}