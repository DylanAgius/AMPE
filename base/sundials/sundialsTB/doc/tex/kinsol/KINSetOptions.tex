%% foo.tex
\begin{samepage}
\hrule
\begin{center}
\phantomsection
{\large \verb!KINSetOptions!}
\label{p:KINSetOptions}
\index{KINSetOptions}
\end{center}
\hrule\vspace{0.1in}

%% one line -------------------

\noindent{\bf \sc Purpose}

\begin{alltt}
KINSetOptions creates an options structure for KINSOL.
\end{alltt}

\end{samepage}


%% definition  -------------------

\begin{samepage}

\noindent{\bf \sc Synopsis}

\begin{alltt}
function options = KINSetOptions(varargin) 
\end{alltt}

\end{samepage}

%% description -------------------

\noindent{\bf \sc Description}

\begin{alltt}
KINSetOptions creates an options structure for KINSOL.

   Usage:

   options = KINSetOptions('NAME1',VALUE1,'NAME2',VALUE2,...) creates a KINSOL
   options structure options in which the named properties have the
   specified values. Any unspecified properties have default values. It is
   sufficient to type only the leading characters that uniquely identify the
   property. Case is ignored for property names. 
   
   options = KINSetOptions(oldoptions,'NAME1',VALUE1,...) alters an existing 
   options structure oldoptions.
   
   options = KINSetOptions(oldoptions,newoptions) combines an existing options 
   structure oldoptions with a new options structure newoptions. Any new 
   properties overwrite corresponding old properties. 
   
   KINSetOptions with no input arguments displays all property names and their
   possible values.
   
KINSetOptions properties
(See also the KINSOL User Guide)

Verbose - verbose output [ {false} | true ]
   Specifies whether or not KINSOL should output additional information
MaxNumIter - maximum number of nonlinear iterations [ scalar | {200} ]
   Specifies the maximum number of iterations that the nonlinar solver is allowed
   to take.
FuncRelErr - relative residual error [ scalar | {eps} ]
   Specifies the realative error in computing f(y) when used in difference
   quotient approximation of matrix-vector product J(y)*v.
FuncNormTol - residual stopping criteria [ scalar | {eps^(1/3)} ]
   Specifies the stopping tolerance on ||fscale*ABS(f(y))||_L-infinity
ScaledStepTol - step size stopping criteria [ scalar | {eps^(2/3)} ]
   Specifies the stopping tolerance on the maximum scaled step length:
                    ||    y_(k+1) - y_k   ||
                    || ------------------ ||_L-infinity
                    || |y_(k+1)| + yscale ||
MaxNewtonStep - maximum Newton step size [ scalar | {0.0} ]
   Specifies the maximum allowable value of the scaled length of the Newton step.
InitialSetup - initial call to linear solver setup [ false | {true} ]
   Specifies whether or not KINSol makes an initial call to the linear solver
   setup function.
MaxNumSetups - [ scalar | {10} ]
   Specifies the maximum number of nonlinear iterations between calls to the
   linear solver setup function (i.e. Jacobian/preconditioner evaluation)
MaxNumSubSetups - [ scalar | {5} ]
   Specifies the maximum number of nonlinear iterations between checks by the
   nonlinear residual monitoring algorithm (specifies length of subintervals).
   NOTE: MaxNumSetups should be a multiple of MaxNumSubSetups.
MaxNumBetaFails - maximum number of beta-condition failures [ scalar | {10} ]
   Specifies the maximum number of beta-condiiton failures in the line search
   algorithm.
EtaForm - Inexact Newton method [ Constant | Type2 | {Type1} ]
   Specifies the method for computing the eta coefficient used in the calculation
   of the linear solver convergence tolerance (used only if strategy='InexactNEwton'
   in the call to KINSol):
      lintol = (eta + eps)*||fscale*f(y)||_L2
   which is the used to check if the following inequality is satisfied:
      ||fscale*(f(y)+J(y)*p)||_L2 &lt;= lintol
   Valid choices are:
                          | ||f(y_(k+1))||_L2 - ||f(y_k)+J(y_k)*p_k||_L2 |
   EtaForm='Type1'  eta = ------------------------------------------------
                                        ||f(y_k)||_L2

                                  [ ||f(y_(k+1))||_L2 ]^alpha
   EtaForm='Type2'  eta = gamma * [ ----------------- ]
                                  [  ||f(y_k)||_L2    ]
   EtaForm='Constant'
Eta - constant value for eta [ scalar | {0.1} ]
   Specifies the constant value for eta in the case EtaForm='Constant'.
EtaAlpha - alpha parameter for eta [ scalar | {2.0} ]
   Specifies the parameter alpha in the case EtaForm='Type2'
EtaGamma - gamma parameter for eta [ scalar | {0.9} ]
   Specifies the parameter gamma in the case EtaForm='Type2'
MinBoundEps - lower bound on eps [ false | {true} ]
   Specifies whether or not the value of eps is bounded below by 0.01*FuncNormtol.
Constraints - solution constraints [ vector ]
   Specifies additional constraints on the solution components.
     Constraints(i) =  0 : no constrain on y(i)
     Constraints(i) =  1 : y(i) &gt;= 0
     Constraints(i) = -1 : y(i) &lt;= 0
     Constraints(i) =  2 : y(i) &gt; 0
     Constraints(i) = -2 : y(i) &lt; 0
   If Constraints is not specified, no constraints are applied to y.

LinearSolver - Type of linear solver [ {Dense} | Band | GMRES | BiCGStab | TFQMR ]
   Specifies the type of linear solver to be used for the Newton nonlinear solver.
   Valid choices are: Dense (direct, dense Jacobian), GMRES (iterative, scaled
   preconditioned GMRES), BiCGStab (iterative, scaled preconditioned stabilized
   BiCG), TFQMR (iterative, scaled preconditioned transpose-free QMR).
   The GMRES, BiCGStab, and TFQMR are matrix-free linear solvers.
JacobianFn - Jacobian function [ function ]
   This propeerty is overloaded. Set this value to a function that returns
   Jacobian information consistent with the linear solver used (see Linsolver).
   If not specified, KINSOL uses difference quotient approximations.
   For the Dense linear solver, JacobianFn must be of type KINDenseJacFn and must
   return a dense Jacobian matrix. For the iterative linear solvers, GMRES,
   BiCGStab, or TFQMR, JacobianFn must be of type KINJactimesVecFn and must return
   a Jacobian-vector product.
KrylovMaxDim - Maximum number of Krylov subspace vectors [ scalar | {10} ]
   Specifies the maximum number of vectors in the Krylov subspace. This property
   is used only if an iterative linear solver, GMRES, BiCGStab, or TFQMR is used
   (see LinSolver).
MaxNumRestarts - Maximum number of GMRES restarts [ scalar | {0} ]
   Specifies the maximum number of times the GMRES (see LinearSolver) solver
   can be restarted.
PrecModule - Built-in preconditioner module [ BBDPre | {UserDefined} ]
   If the PrecModule = 'UserDefined', then the user must provide at least a
   preconditioner solve function (see PrecSolveFn)
   KINSOL provides a built-in preconditioner module, BBDPre which can only be used
   with parallel vectors. It provide a preconditioner matrix that is block-diagonal
   with banded blocks. The blocking corresponds to the distribution of the variable
   vector among the processors. Each preconditioner block is generated from the
   Jacobian of the local part (on the current processor) of a given function g(t,y)
   approximating f(y) (see GlocalFn). The blocks are generated by a difference
   quotient scheme on each processor independently. This scheme utilizes an assumed
   banded structure with given half-bandwidths, mldq and mudq (specified through
   LowerBwidthDQ and UpperBwidthDQ, respectively). However, the banded Jacobian
   block kept by the scheme has half-bandwiths ml and mu (specified through
   LowerBwidth and UpperBwidth), which may be smaller.
PrecSetupFn - Preconditioner setup function [ function ]
   PrecSetupFn specifies an optional function which, together with PrecSolve,
   defines a right preconditioner matrix which is an aproximation
   to the Newton matrix. PrecSetupFn must be of type KINPrecSetupFn.
PrecSolveFn - Preconditioner solve function [ function ]
   PrecSolveFn specifies an optional function which must solve a linear system
   Pz = r, for given r. If PrecSolveFn is not defined, the no preconditioning will
   be used. PrecSolveFn must be of type KINPrecSolveFn.
GlocalFn - Local right-hand side approximation funciton for BBDPre [ function ]
   If PrecModule is BBDPre, GlocalFn specifies a required function that
   evaluates a local approximation to the system function. GlocalFn must
   be of type KINGlocalFn.
GcommFn - Inter-process communication function for BBDPre [ function ]
   If PrecModule is BBDPre, GcommFn specifies an optional function
   to perform any inter-process communication required for the evaluation of
   GlocalFn. GcommFn must be of type KINGcommFn.
LowerBwidth - Jacobian/preconditioner lower bandwidth [ scalar | {0} ]
   This property is overloaded. If the Band linear solver is used (see LinSolver),
   it specifies the lower half-bandwidth of the band Jacobian approximation.
   If one of the three iterative linear solvers, GMRES, BiCGStab, or TFQMR is used
   (see LinSolver) and if the BBDPre preconditioner module in KINSOL is used
   (see PrecModule), it specifies the lower half-bandwidth of the retained
   banded approximation of the local Jacobian block.
   LowerBwidth defaults to 0 (no sub-diagonals).
UpperBwidth - Jacobian/preconditioner upper bandwidth [ scalar | {0} ]
   This property is overloaded. If the Band linear solver is used (see LinSolver),
   it specifies the upper half-bandwidth of the band Jacobian approximation.
   If one of the three iterative linear solvers, GMRES, BiCGStab, or TFQMR is used
   (see LinSolver) and if the BBDPre preconditioner module in KINSOL is used
   (see PrecModule), it specifies the upper half-bandwidth of the retained
   banded approximation of the local Jacobian block.
   UpperBwidth defaults to 0 (no super-diagonals).
LowerBwidthDQ - BBDPre preconditioner DQ lower bandwidth [ scalar | {0} ]
   Specifies the lower half-bandwidth used in the difference-quotient Jacobian
   approximation for the BBDPre preconditioner (see PrecModule).
UpperBwidthDQ - BBDPre preconditioner DQ upper bandwidth [ scalar | {0} ]
   Specifies the upper half-bandwidth used in the difference-quotient Jacobian
   approximation for the BBDPre preconditioner (see PrecModule).


   See also
        KINDenseJacFn, KINJacTimesVecFn
        KINPrecSetupFn, KINPrecSolveFn
        KINGlocalFn, KINGcommFn
\end{alltt}






\vspace{0.1in}