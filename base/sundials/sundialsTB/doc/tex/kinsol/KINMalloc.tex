%% foo.tex
\begin{samepage}
\hrule
\begin{center}
\phantomsection
{\large \verb!KINMalloc!}
\label{p:KINMalloc}
\index{KINMalloc}
\end{center}
\hrule\vspace{0.1in}

%% one line -------------------

\noindent{\bf \sc Purpose}

\begin{alltt}
KINMalloc allocates and initializes memory for KINSOL.
\end{alltt}

\end{samepage}


%% definition  -------------------

\begin{samepage}

\noindent{\bf \sc Synopsis}

\begin{alltt}
function [] = KINMalloc(fct,n,varargin) 
\end{alltt}

\end{samepage}

%% description -------------------

\noindent{\bf \sc Description}

\begin{alltt}
KINMalloc allocates and initializes memory for KINSOL.

   Usage:   KINMalloc ( SYSFUN, N [, OPTIONS [, DATA] ] );

   SYSFUN   is a function defining the nonlinear problem f(y) = 0.
            This function must return a column vector FY containing the
            current value of the residual
   N        is the (local) problem dimension.
   OPTIONS  is an (optional) set of integration options, created with
            the KINSetOptions function. 
   DATA     is the (optional) problem data passed unmodified to all
            user-provided functions when they are called. For example,
            RES = SYSFUN(Y,DATA).

   See also: KINSysFn
\end{alltt}






\vspace{0.1in}