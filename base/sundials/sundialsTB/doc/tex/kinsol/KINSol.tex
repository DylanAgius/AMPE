%% foo.tex
\begin{samepage}
\hrule
\begin{center}
\phantomsection
{\large \verb!KINSol!}
\label{p:KINSol}
\index{KINSol}
\end{center}
\hrule\vspace{0.1in}

%% one line -------------------

\noindent{\bf \sc Purpose}

\begin{alltt}
KINSol solves the nonlinear problem.
\end{alltt}

\end{samepage}


%% definition  -------------------

\begin{samepage}

\noindent{\bf \sc Synopsis}

\begin{alltt}
function [status,y] = KINSol(y0, strategy, yscale, fscale) 
\end{alltt}

\end{samepage}

%% description -------------------

\noindent{\bf \sc Description}

\begin{alltt}
KINSol solves the nonlinear problem.

   Usage:  [STATUS, Y] = KINSol(Y0, STRATEGY, YSCALE, FSCALE)

   KINSol manages the computational process of computing an approximate 
   solution of the nonlinear system. If the initial guess (initial value 
   assigned to vector Y0) doesn't violate any user-defined constraints, 
   then KINSol attempts to solve the system f(y)=0. If an iterative linear
   solver was specified (see KINSetOptions), KINSol uses a nonlinear Krylov
   subspace projection method. The Newton-Krylov iterations are stopped
   if either of the following conditions is satisfied:

       ||f(y)||_L-infinity &lt;= 0.01*fnormtol

       ||y[i+1] - y[i]||_L-infinity &lt;= scsteptol

   However, if the current iterate satisfies the second stopping
   criterion, it doesn't necessarily mean an approximate solution
   has been found since the algorithm may have stalled, or the
   user-specified step tolerance may be too large.

   STRATEGY specifies the global strategy applied to the Newton step if it is
   unsatisfactory. Valid choices are 'None' or 'LineSearch'.
   YSCALE is a vector containing diagonal elements of scaling matrix for vector 
   Y chosen so that the components of YSCALE*Y (as a matrix multiplication) all 
   have about the same magnitude when Y is close to a root of f(y)
   FSCALE is a vector containing diagonal elements of scaling matrix for f(y) 
   chosen so that the components of FSCALE*f(Y) (as a matrix multiplication) 
   all have roughly the same magnitude when u is not too near a root of f(y)

   On return, status is one of the following:
     0: KINSol succeeded
     1: The initial y0 already satisfies the stopping criterion given above
     2: Stopping tolerance on scaled step length satisfied
    -1: Illegal attempt to call before KINMalloc
    -2: One of the inputs to KINSol is illegal.
    -5: The line search algorithm was unable to find an iterate sufficiently 
        distinct from the current iterate
    -6: The maximum number of nonlinear iterations has been reached
    -7: Five consecutive steps have been taken that satisfy the following 
        inequality:
             ||yscale*p||_L2 &gt; 0.99*mxnewtstep
    -8: The line search algorithm failed to satisfy the beta-condition
        for too many times.
    -9: The linear solver's solve routine failed in a recoverable manner,
        but the linear solver is up to date.
   -10: The linear solver's intialization routine failed.
   -11: The linear solver's setup routine failed in an unrecoverable manner.
   -12: The linear solver's solve routine failed in an unrecoverable manner.

   See also KINSetOptions, KINGetstats
\end{alltt}






\vspace{0.1in}