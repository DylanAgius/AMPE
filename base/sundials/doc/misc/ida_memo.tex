\documentclass[12pt]{letter}
\setlength{\textheight}{9.0in}
\setlength{\textwidth}{6.0in}
\setlength{\topmargin}{-0.4in}

\begin{document}

\pagestyle{empty}

\begin{letter}

\hfill November 19, 2004

\vspace{0.2in}
\centerline{\bf Memorandum}

\noindent{\bf TO}: Reviewers of IDA \\
\noindent{\bf FROM}: Alan C. Hindmarsh, Radu Serban \\
\noindent{\bf SUBJECT}: Software release for IDA

IDA is a general purpose (serial and parallel) solver for differential algebraic
equation (DAE) systems or implicit ordinary differential equation (ODE) systems.
IDA was developed within CASC under ASCI-PSE and MICS-SciDAC support. 
It is not usable in stand-alone form; it must be combined with an application program.

At this time we wish to release a new version of IDA (v2.2.0) for the 
following reasons:
\begin{itemize}
\item To promote research collaborations between CASC personnel and colleagues at 
      other sites, collaborations in which IDA will be useful as a research tool;
\item To elicit feedback from outside users which could lead to enhancements that will 
      benefit LLNL users as well;
\item To enhance the external reputation of CASC and LLNL with respect to 
      mathematical software written here.
\end{itemize}

IDA shares a common design philosophy and a number of support modules with the 
other solvers in the Suite of Nonlinear and Differential/Algebraic equation Solvers (SUNDIALS), 
namely CVODE, CVODES, and KINSOL, all of which have been previously released for unlimited 
distribution.

Being an item for general-purpose mathematical software, IDA is not expected to be 
subject to export controls on the basis of its intrinsic capabilities. 
No fundamentally new functionality has been added since the previous version released
(v2.0, UCRL-CODE-2002-59).

A brief abstract of the package is provided in the Code Abstract. 


\vspace{0.5in}                 
\hfill Alan C. Hindmarsh, CASC

\vspace{0.5in}                 
\hfill Radu Serban, CASC


\end{letter}
\end{document}