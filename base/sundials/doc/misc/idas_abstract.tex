\documentclass[12pt]{letter}
\setlength{\textheight}{9.0in}
\setlength{\textwidth}{6.0in}
\setlength{\topmargin}{-0.4in}

\begin{document}

\pagestyle{empty}

\begin{letter}

\vspace{0.2in}
\centerline{\bf Code Abstract for IDAS v1.0.0}

\begin{enumerate}

\item {\bf Identification}

Software acronym: IDAS v1.0.0

Short title: Stiff DAE integrator with sensitivity analysis capabilities

\item {\bf Developer name(s) and affiliations}

Radu Serban (CASC, LLNL)

\item {\bf Software Completion Date}

November 26, 2007

\item {\bf Brief description}

IDAS is a general purpose (serial and parallel) solver for differential algebraic
equation (DAE) systems or implicit ordinary differential equation (ODE) systems
with sensitivity analysis capabilities. It provides both forward and adjoint
sensitivity analysis options.

\item {\bf Method of solution}

Integration is by the BDF method. Corrector iteration is by Newton iteration. 
For the solution of linear systems within Newton iteration, users can select a 
dense solver, a band solver, or a preconditioned iterative solver (GMRES, BiCG-Stab,
or TFQMR). The IDAS forward sensitivity module implements a simultaneous corrector
method and a staggered corrector method. The adjoint sensitivity provides the
infrastructure required for the backward integration in time of multiple systems of 
differential equations dependent on the solution of the original DAEs. It employs a 
checkpointing scheme for efficient reproduction of forward solutions during the 
backward integration and provides a choice of either a cubic Hermite or a 
variable-order polynomial interpolation scheme.

\item {\bf Computer(s) for which software is written}

IDAS should run on any computer with an ANSI C compiler. The appropriate 
precision (single, double, or extended) is selected at the configuration phase.

\item {\bf Operating system}

No system-dependency in the software itself. But installation is system-dependent. 
The package supplied consists of a single archived file. Installation from this file 
assumes a system with the tar utility. Configuration is done through a 
configure script. Compilation of libraries is done by way of makefiles.

\item {\bf Programming language(s) used}

ANSI C (100\%)

\item {\bf Software limitations}

none

\item {\bf Unique features of the software}

IDAS is organized in a highly modular manner. The basic integrator and 
sensitivity modules are separate from, and independent of, the linear system 
solvers, as well as the vector operation modules. Thus the set of linear solvers can be 
expanded and the internal vector representation can be replaced with no impact on 
the main solver.


\item {\bf Related and auxiliary software}

IDAS is part of SUNDIALS (Suite of Nonlinear and Differential/Algebraic equation Solvers). 

\item {\bf Other Programming or Operating Information or Restrictions}

none


\item {\bf Hardware Requirements}

none


\item {\bf Time Requirements}

Timing is highly dependent on machine and problem.


\item {\bf References}

Document provided with the distribution
\begin{itemize}
\item R. Serban and C. Petra, "User Documentation for IDAS v1.0.0," 
    LLNL technical report UCRL-SM-234051, August 2007.
\end{itemize}
Additional background references
\begin{itemize}
\item A. C. Hindmarsh, P. N. Brown, K. E. Grant, S. L. Lee, R. Serban, 
    D. E. Shumaker, and C. S. Woodward, "SUNDIALS, Suite of Nonlinear and 
    Differential/Algebraic Equation Solvers," ACM Trans. Math. Softw., 
    31, pp. 363--396, 2005.
\item Y. Cao, S. Li, L.R. Petzold, and R. Serban, "Adjoint sensitivity analysis for 
    differential-algebraic equations: the adjoint DAE and its numerical solution," 
    SIAM J. Sci. Comp., 24(3), pp. 1076-1089, 2003.
\item T. Maly and L.R. Petzold, "Numerical methods and software for sensitivity analysis 
    of differential-algebraic systems," Appl. Numer. Math., 20, pp. 57-79, 1996
\item W.F. Feehery, J.E. Tolsma, and P.I. Barton, "Efficient sensitivity analysis of large-
   scale differential-algebraic systems," Appl. Numer. Math., 25, pp. 41-54, 1997
\end{itemize}
\end{enumerate}

\end{letter}
\end{document}