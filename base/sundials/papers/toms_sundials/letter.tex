\documentclass[12pt]{letter}
\batchmode

\setlength{\textheight}{9.0in}
\setlength{\topmargin}{-0.4in}

\begin{document}
\begin{letter}


\reference{Manuscript TOMS-2003-0041}

\opening{Dear Dr. Boisvert,}

We are hereby submitting a revised version of our paper, ``SUNDIALS:
Suite of Nonlinear and Differential/Algebraic Equation Solvers.''

Attached below, we have reproduced the referees' comments, with our
responses inserted.

We have made a few changes to the paper that were not called for
by the referees.  First, having recently added a rootfinding
feature to CVODE, we added a short paragraph about this feature at
the end of Section 2.1, and a few lines in Section 6.  Second,
some minor changes in Sections 5 and 6 have been made, to reflect
a number of recent changes in the design of the NVECTOR module and
in names of constants in the user interface.  I hope this is
acceptable.


\closing{Sincerely,}

Carol S. Woodward\\
Center for Applied Scientific Computing\\
Lawrence Livermore National Laboratory

\end{letter}


\newpage
\section{Answers to Referee 1 Comments}

General

This is a clearly written paper setting out the current state of the
Lawrence Livermore National Laboratory project described in the title.
It outlines how existing solvers were re-written with the aims of:
better support for solving large systems in parallel, including a wide
variety of linear system solvers both direct and iterative; better
extraction of software components usable in several solvers; a user
interface suitable for both `simple' and `expert' use. Also described
is the partly completed work to add sensitivity analysis to the
solvers.

The basic solvers (ODE, DAE, algebraic) are described.  Much of course
is standard, but there is a wealth of detail about algorithm
heuristics that have evolved over years of experience. This is
invaluable information for any practitioner in the field. The more so
because appearing in the open literature helps protect it from
interest groups trying to patent common knowledge, as is currently
happening in some areas of scientific computing.

A detailed description of preconditioning methods and options follows.

Next is a description of the way in which forward and adjoint methods
are applied to compute sensitivities. It is done at the model level,
that is by discretizing the derivatives of the mathematical equations
rather than the Automatic Differentiation route of differentiating the
discretization.

There follow sections on the code organization and on how a user sets
up the system to solve a problem, and extracts the results.

The abstract and conclusions are appropriate. The references cover
theory, usage and experience with the solvers and run from the
seventies to the present.

I recommend publication subject to some minor improvements suggested
below.  I would not wish to see the revised version before publication.


Software architecture

In the abstract, the authors say "a highly modular structure to allow
incorporation of different preconditioning and/or linear solver
methods; and clear interfaces allowing for users to provide their own
data structures underneath the solvers"; and in the conclusion section
they state: "The design philosophy of providing clear interfaces to
the user and allowing the user to supply their own data structures
make the solvers reasonably easy to add into existing simulation
codes".

Designing good interfaces in complex numerical software, such as
SUNDIALS, is not a trivial task. The authors have accumulated
valuable experience doing that, and others should benefit from
this experience. An illustration of how ``clear'' interfaces are
achieved and how new methods can be added would be helpful.  A
discussion on design decisions is valuable too. If it is
impractical to add to the present paper and is explained
elsewhere, references to where the reader can find such
information should be given.

{\em Response:} A paragraph on the interface design has been added as
Subsection 5.3, and a paragraph about user-supplied linear solvers
has been added to Subsection 5.1.


Heuristics

Many heuristics are discussed in this paper. The authors give values
for numerous constants, but do not explain why they chose these
values.  On p.6, why is the safety factor 1/6? On p.5, the stepsize
$h_n$ is replaced by $h_n/4$. Why not $h_n/2$ for example? On p.6, the
safety factor is limited above by 0.2. Why is 0.2 chosen?

I accept this paper is not the place to justify `magic numbers' in
heuristics. But a researcher or a practitioner may be interested in
knowing the reasons for choosing a certain value of a constant or a
parameter. If there are technical reports where they are justified, it
would be useful to have references to them.

Also, wherever a snippet of rationale is provided, as p.12 `The above
strategy balances ... ', it immediately makes the text more accessible
to the non-specialist reader.

{\em Response:} A paragraph on heuristics was added at the start of
the section.  It discusses the difficulty of explaining the choices of
heuristic here, but refers to literature on older codes.


Specific comments

General

Quite a few paragraphs would read better if split. I have made some
suggestions, and leave to the authors to decide on better paragraph
splitting.


Page 4.

The notation $y_{n(m)}$ jars. It means the approximation, at the $m$th
(Newton or other) iteration, to the computed solution $y_n$ at time
point $n$, but of course if read literally it indicates that $n$ is
somehow a function of $m$.  I know you use $y^i_n$ to mean the $i$th
component of $y_n$ so superscript $m$ is out, but I think $y_n(m)$ or
$y_{n,m}$ reads a lot better.

{\em Response:} Done.  We now use $y_{n,m}$ throughout.

line -17. A formula for $\|\cdot\|_{WRMS}$ would be helpful.

{\em Response:} Done.

Page 5. line 2. What is ``non-fatal convergence failure''?

{\em Response:} Answering this here would be out of place, since it
involves a user-defined error flag.  We dropped the offending word.

Page 6.

line 6. A reference to Curtis-Powell-Reid might be included.

{\em Response:} Done.

line 14. `A critical part of CVODE ... ' The comma should become a
dash, or vice versa.  But is this sentence needed? Anyone reading this
far knows well enough that professional ODE software controls local error.

{\em Response:} Done (with commas).  Sadly, we feel the editorial for
error control is needed for some readers.

line -9. "the step is restarted from scratch" It is not clear what this means.

{\em Response:} Done.

line -8. The ratio $h'/h$ is limited above to 0.2. It is not clear if
0.2 stays until the end of the integration. This should be clarified;
the same for 0.1.

{\em Response:} Done.

Page 7.

line 1. `on the step just completed ... done'. Would `on the previous step ...
done on this step' be clearer?

{\em Response:} Done - reworded for clarity.

line 19,22. It is little confusing with LTE denoting two different things.
I think a formula for LTE($q'$) would help the reader.

{\em Response:} Fixed ambiguous notation, using a subscript on LTE in the
paragraph about changing order.  We feel a subscript on LTE in the previous
paragraph would be unnecessary clutter.  We also feel that a formula for
any of the LTE's is not appropriate.

Page 8.

Inexact Newton Iteration. Using $m$ rather than $n$ fits with earlier
notation.

{\em Response:} Kept $n$ notation, as this leaves the presentation
consistent with literature on Newton's method.  Added
clarification of the meaning of $n$ for this iteration.

The assignment $u_{n+1} = u_n + \delta_n$ threw me, when it is clear
from the next paragraph that a `cautious Newton' $u_{n+1} = u_n +
\lambda\delta_n$ is actually used, and only part way down p9 is it
explained that either of these may be used. Possibly change 2(b) to
`Set $u_{n+1} = u_n + $ multiple of $\delta_n$'. Alter p9 text accordingly.

{\em Response: } Done. Changed line 2(b) to specify that $\lambda
\delta_n, \lambda \leq 1$, is used.  Changed following text
appropriately.

line -13. `These solutions are only approximate ... new approximate
solution.'  mixes up two meanings of approximate solution. It might
better read `The update $\delta_n$ is only an approximate solution of
system 2(a). At each stage in the iteration process, a scalar multiple
of the update is added to the previous approximate solution, $u_n$, to
produce a new approximate solution.'

{\em Response: } Done. Removed use of approximate solution when
referring to the iterate.

Page 10.

line 3, 5, 7, 9, 11. The $\eta$ symbol is also used on p. 7, line
14,16, but for a different purpose.

{\em Response: } Again, here we chose to stay with the $\eta$
notation to remain consistent with the literature in the area.  We
do not feel that this re-use of notation in a different context
will be confusing.

line 13. You should explain what KINSOL does if the solution it is
seeking violates a constraint such as $u^i > 0$.

{\em Response: } Done.  A brief description of the algorithm for
scaling steps to satisfy constraints has been added.

Page 13.
line -18. The strategy for starting IDA is described, but there is no
corresponding description for CVODE. For consistency, such explanation
could be given.

{\em Response:} Actually it is given for CVODE, but it is short.
CVODE starts at order 1 and then uses the same order change algorithm
as on all steps.

Page 14. The order selection approach may be easier to read if
presented in the form of an algorithm.

{\em Response:} We considered that but decided to leave it as is.

Page 21. Sect 4.3 and elsewhere. I found it slightly confusing that a
(purely) algebraic system is referred to as a `non linear system' as
if `nonlinear' is the key aspect.  I suggest `algebraic system'.

{\em Response: } Added the modifier ``algebraic'' to the system at
the start of subsection 4.3.  As our technology specifically
targets nonlinear systems, we chose to leave that term in.

Page 22. Sect 5. It would be useful to give reasons for choosing C
rather than Fortran 90/95, which (as far as I can see) can accommodate
each of the listed design features.  `Our programmers know C best' is
an acceptable reason.

{\em Response: } Added two more criteria for doing the C rewrites
to the list, use of languages with stable compilers and
facilitation of use with object-oriented codes being written in C
and C++.  This information was then moved to the introduction to
satisfy the other referee.

Wording and paragraphs

Page 2. New paragraph e.g. at "SUNDIALS is being expanded ... ".

{\em Response: } Done.

Page 3.

line 2, 3. "contained" is redundant.

line -22. "which" refers to $N$-space", not "ODE...". Reword.

line -21. You may omit "abstract".

{\em Response:} All three done.

line -9. New paragraph at "For nonstiff problems, CVODE...".

{\em Response:} We disagree; the paragraph is not that long.

line -4. "recent history of the step sizes" is not very precise. You
may say which step sizes.

{\em Response:} Done.

Page 4.

line 3. New paragraph at "Functional iteration...".

{\em Response:} Given the following change, we feel this paragraph
break is not needed.

line 18. New paragraph at "For large..."

{\em Response:} Done.

line -5. Comma after "as possible".

{\em Response:} Done.

Page 5.

line 5, 7. See "reevaluate" and "re-evaluate".

{\em Response:} Done.

line 20. New paragraph at "We initialize..."

{\em Response:} We disagree; the paragraph seems better unbroken.

Page 7.
line -20. Comma before "but instead".

{\em Response:} Done.

Page 8.
line 1. New paragraph at "The implementation...".

{\em Response:} Paragraph broken one sentence earlier.

Page 10.

line 14. You may say that $i$ denotes the $i$th component of $u$.

{\em Response: } Done.

line -5. New paragraph at "A second...".

{\em Response:} Done, but this forces another break at page 11, line 3
(in original manuscript).

Page 13.
line 10. For consistency, the subscript WRMS may be omitted.

{\em Response:} Done.

Page 18.

line -9. Period after "...each step".

line -5. Period after "matrix".

line -1. Period after "[Feehery ...]".

Page 22.
line 14. Remove " " in "NVECTOR".

{\em Response:} Done.

Page 29. The formats of the first and the 17th reference are not consistent.

{\em Response:} Fixed.

Software

The serial version installs nicely on Sun Solaris with the GNU C
compiler, and all the examples execute without any problems. Since I
do not have currently access to a parallel machine, I have not tried
the parallel version of SUNDIALS.

However, I had slight difficulties installing SUNDIALS on Macintosh
running OS X. I believe this can be easily fixed by the authors, and I
encourage them to do so.

The SUNDIALS suite is very well organized across directories. The
programs are clearly written and well commented.  As a minor remark, I
find the main solver files quite long: kinsol.c 1410 lines, cvode.c
3048, ida.c 3062 and cvodes.c 5791.


\newpage
\section{Answers to Referee 2 Comments}

General Comments on the Manuscript

This paper provides a comprehensive description of a large software
modification/development effort involving three codes, CVODE for
initial value ordinary differential equations, KINSOL, for nonlinear
algebraic equations, and IDA for systems of differential algebraic
equations, which are collectively referred to as the SUNDIAL suite. I
found it very enjoyable to read a paper where lots of detail regarding
the development of the software and in particular the underlying
heuristics employed within the software is provided. The paper
describes high quality software and is useful in directing attention
to this important project.  I would therefore recommend in favor of
its publication.

(I) One point that arises in several places in the paper is that it is
not clear in the paper if each code is written so that it can be run
in serial or parallel mode or that there are separate serial and
parallel versions of each code. There is a comment on Page 2 which
indicates that CVODE is a single code that includes both options but I
am not sure about the other two codes.  Throughout the paper there are
references to serial and parallel versions of the codes.  Also it
would be useful to include a brief comment on how the parallel mode
would work.  To my knowledge ANSI standard C does not include parallel
language features so it seems unclear what is meant by the phrase
suitable for ... parallel machine environments in the abstract.  I
assume that the authors mean that software is written in a highly
modular fashion that means that it can easily be customized to run e
ciently in a parallel environment based on a specific parallelization
facility such as Open MP or MPI, for example.

{\em Response: } We have added a paragraph in the introduction
stating the difference between running the codes in serial and
parallel and why we no longer distinguish between serial and
parallel versions.

(II) A second point is that it would be useful to provide a
paragraph commenting on the numerical performance for each of the
SUNDIAL codes compared with other available codes, with
appropriate references.

{\em Response: } We have not done such performance comparison
testing and do not feel that it is within the scope of this paper
to provide such comparisons.

Detailed Comments on the Manuscript

Page 2, Line 8: Add an appropriate reference after "GMRES iteration".

{\em Response: } Done.

Page 2, Line 17: After the sentence ending with "rewrite solvers in C",
I think it would be helpful to the reader to explain why the desire
for a parallel implementation forces one to choose the C language;
this would seem to imply that one cannot have a parallel
implementation unless it is written in C. Perhaps there are additional
reasons one can mention for providing a C implementation.

{\em Response: } Moved text from the Code Organization section
which motivated the C language choice to the Introduction.

Page 2, Lines 21-24: I would suggest a slight rewriting for
clarification as I am not sure about the serial/parallel capabilities
of the software packages mentioned here. It is quite clear the PVODE
and the original CVODE have been merged into one code with both serial
and parallel capabilities called CVODE. Is it also the case that NKSOL
and DASPK have been rewritten as the new codes KINSOL and IDA,
respectively, where the new codes are written in C with both serial
and parallel capabilities? Or is it the case that the new codes are
simply serial C codes?

{\em Response: } Changes made in answer to the first issue above
should address these questions.

Page 2, Lines 27-28: These lines appear to say that all three codes,
CVODE, KINSOL, and IDA, have been augmented to include sensitivity
analysis capabilities. However, at the bottom of this same page the
text appears to indicate that only CVODE has been transformed into
CVODES, a new version with this capability, while for the other two
codes, this capability is planned. It is clear from section 4 that
only CVODES is currently available so it would be useful to clarify
the statement in the introduction. Will the three codes CVODES,
KINSOLS, and IDAS replace the CVODE, KINSOL, and IDA codes?

{\em Response: } Clarified in these lines that sensitivity solvers
are either in existence or planned and made more specific status
comments two paragraphs down.

Page 3, Lines 25-27: Add a reference to further reading after the
brief explanation of stiffness given here.

Page 3, Lines 32-33: I would suggest changing "one of two multistep
methods" to "one of two families of multistep formulas".

Page 4, Line 5 and 8: The notation, $y_{n(m+1)}$, has not been defined.

Page 4, Line 12: Perhaps the phrase "available history data" could be
explained a bit; this may be more detail than you want to include so
instead perhaps you could refer to "solution information associated
with previous steps"?

{\em Response:} All four done.

Page 4, Lines 14-15: If the current CVODE runs in both sequential and
parallel environments, i.e., there is no longer a separate version for
sequential mode and one for parallel mode. Thus on page 4, I think it
1 would be better to refer to "serial mode only" rather than "serial
version only", here and elsewhere in the paper.

{\em Response:} Added explanatory note.

Page 4, Line 16-17: Add references directing the reader to additional
descriptions for the diagonal approximate Jacobian solver and SPGMR.

Page 4, Line 19: Replace "a preconditioned GMRES" with "the SPGMR"

Page 4, Line 25: Add an equation showing the reader in more detail
what the $\|\cdot\|_{WRMS}$ norm looks like.

Page 4, Line 28: In equation (4), define RTOL, ATOL$_i$, and $y^i$.

{\em Response:} All four done

Page 4, Line 29: Replace "a vector whose norm" with "a vector whose $i$th
component is an error-like quantity associated with $y^i$ which has a
$\|\cdot\|_{WRMS}$ norm of".

{\em Response:} Reworded for clarity.

Page 4, Line -1: Does "steps" here mean "time steps" rather than "iterations"?

{\em Response:} Yes; clarified.

Page 5, Line 1: I suggest replacing the current text with "the value
of $\gamma$ from the last update, which we will call $\bar{\gamma}$,
and the current $\gamma$ value, which we will call $\gamma$, satisfy
$|\gamma/\bar{\gamma} - 1| > 0.3 $"

{\em Response:} Reworded.

Page 5, Line 2: Explain what is meant by a "non-fatal convergence failure"?

{\em Response:} Reworded.  A fatal convergence here refers to a fatal
error flag that the user may have returned from the preconditioner
routine.  Since this section is about algorithms, it seems best not to
bother going into that here.

Page 5, Line -14: Change "diverged" to "to have failed".

{\em Response:} Reworded.

Page 5, Lines 15-16: Change "final computed value" to "final iterate
computed". Also define $y_{n(0)}$ and $\epsilon$.

Page 5, Line -14,-15: Replace "convergence fails with J or P current" by
"the iteration fails to converge with a current J or P".

Page 5, Line -9: Define the "local error test constant".

Page 6, Line 3: What is $\sigma_ 0$?

Page 6, Line 6: Provide a reference for the CPR algorithm.

Page 6, Line 23: After "$ \leq 1$" add "(where we recall that we are
using the $\|\cdot\|_{WRMS}$ norm)".

Page 6, Line 23-26: I would suggest a rewrite of the sentence beginning
with "Using the above, ..."  to improve its clarity. Is $C'$ known?

Page 6, Lines 33-34: Please clarify the wording: "limited above to" and
"limited below to".

{\em Response:} All eight done. Yes, $C'$ is known.

Page 9, Line 18: Since the reader will not see the details of the
preconditioning schemes until Section 3, I would recommend deleting
the sentence beginning with "In this case, ...".

{\em Response: } We chose to leave this sentence in as it gives the
specific form of preconditioning that is applied in KINSOL.

Page 9, Line -12: Give a brief indication of what criterion is used to
decide when one is close to the solution.

{\em Response: } We have added a reference for further details of
the linesearch algorithm.

Page 10, Line -11: Add a reference after "so-called semi-explicit
index-one systems".

{\em Response:} Done.

Page 11, Line 9: It would be helpful to the reader to point out that
this equation is a special case of the more general equation (which
could have a number) on Page 3.

{\em Response:} We disagree; the value of that is not clear.  The
literature for IDA and CVODE simply use different forms for BDF.

Page 11, Lines 27-28: Until the reader has seen Section 3, I think it
is better to defer the details of the preconditioning so I would
recommend leaving out the "so that GMRES is applied ..."  part of this
sentence. Also I do not understand the footnote.

{\em Response:} We disagree.  The specific preconditioned system is needed
to understand the stopping test described one page later. Further
explanation was added to the footnote.

Page 12, Lines 22, 23: Change "$S = 20$" to "20" and "$S = 100$" to "100".

{\em Response:} Done.

Page 13, Line 15: At this point in this section there is mention of "an
associated polynomial interpolant".  I think it would be helpful to
mention the existence of this interpolant right after the BDF methods
are introduced in equation (10).

{\em Response:} We disagree.  This would seem to add considerable
detail that has no purpose.

Page 14, Line -3: Add a few sentences at the beginning of this section
to introduce the basic idea of a Krylov method with a preconditioner.

{\em Response:} Done.

Page 17, Lines -4,-3: I would suggest replacing "dimension of the array
of model parameters" with "number of model parameters" and "dimension of
the array of outputs" with "number of output functionals".

{\em Response:} Done.

Page 22, First Paragraph: This material motivating the rewriting of
the Fortran codes in C is very helpful, a nd I think should be moved
to the introduction.

{\em Response: } Done.

Page 22, Lines 17-21: The discussion of the NVECTOR implementation and
NVECTOR structure are not clear.  Reference to Section 5.2 with an
example provided there might help.

{\em Response: } Done.  We have added clarifying discussion to
Section 5.2 and referenced the solver user guides for more
information.

Page 26, Table I: On Page 12 the maximum number of Newton iterations
for IDA is 4 while in this table it is 3.

{\em Response:} Fixed; the correct value is 4.

Pages 29-30: Some of the journal names are given in abbreviated form
in some places and in full form in other places.

{\em Response:} Done; all references use consistent formats now.


\end{document}
