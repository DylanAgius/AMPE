\section{Sensitivity Analysis}\label{s:sensitivity_analysis}

Many times, models depend on parameters, either through their defining
function --- $f(t,y)$ for the ODE in (\ref{ODE}), $F(t,y,{\dot y})$ for the DAE
(\ref{e:DAE}), and $F(u)$ for nonlinear systems (\ref{nonlinear_system})
--- or through initial conditions in the case of ODEs and DAEs.
In addition to the solution $y$ or $u$, we often want to quantify how
the solution (or some other output functional that depends on the solution)
is influenced by changes in these model parameters.

Depending on the number of model parameters and the
number of functional outputs, one of two sensitivity methods
is more appropriate.
%
The {\em forward sensitivity} method is mostly suitable when we need
the gradients of many outputs (for example the entire solution vector)
with respect to relatively few parameters.
In this approach, the model is differentiated with respect to each
parameter in turn to yield an additional system of the same size as
the original one, the result of which is the solution sensitivity.
The gradient of any output function depending on the solution can
then be directly obtained from these sensitivities by applying the
chain rule of differentiation.
%
The {\em adjoint sensitivity} method is more practical than
the forward approach when the number of parameters is large and
when we need the gradients of only few output functionals.
In this approach, the solution sensitivities need not be computed
explicitly. Instead, for each output functional of interest, we form
and solve an additional system, adjoint to the original one, the
solution of which can then be used to evaluate the gradient of the
output functional with respect to any set of model parameters.

For each of the basic solvers described in Section~\ref{s:basic_solvers},
extensions that are sensitivity-enabled are already available (CVODES),
or under development (IDAS), or under consideration depending on
the need (KINSOLS).
The various algorithmic features of CVODES and IDAS are documented
in~\cite{CLPS:03}. A detailed description of the CVODES software package
is presented in~\cite{SeHi:03}, while full usage description is given
in~\cite{HiSe:04cvodes}.
% Maybe also add an upcoming IDAS user guide?
%
%==========================================================================

\subsection{CVODES}\label{ss:cvodes}

CVODES is an extension of CVODE that, besides solving ODE initial
value problems of the form (\ref{ODE}), also provides forward and
adjoint sensitivity analysis capabilities.
%
Here, we assume that the system depends on a vector of parameters,
$p = [p_1,\ldots,p_{N_p}]$,
\begin{equation}\label{e:ODE_with_p}
\dot{y} = f(t,y,p), \quad y(t_0,p) = y_0(p) \, ,
\end{equation}
including the case where the initial value vector $y_0$ depends on $p$,
and we consider a scalar output functional of the form $g(t,y,p)$.
%
In addition to $y$ as a function of $t$, we want the total derivative
$d g / d p  = ({\partial g}/{\partial y}) s + {\partial g}/{\partial p}$,
where $s = dy/dp \in R^{N \times N_p}$ is the so-called sensitivity matrix.
%
Each column $s_i = d y / d p_i$ of $s$ satisfies the sensitivity ODE
\begin{equation}\label{e:fwdODE}
\dot{s}_i = J s_i + \frac{\partial f}{\partial p_i} \, ,
\quad s_i(t_0) = \frac{d y_0}{d p_i} \, ,
\end{equation}
where $J$ is the system Jacobian defined in (\ref{Newtonmat}).

%------------------------------------------------------------------
\subsubsection{Forward Sensitivity}
CVODES can be used to integrate an extended system
$Y = [y,s_1,\ldots,s_{N_s}]$ forward in time, where
$[s_1,\ldots,s_{N_s}]$ are a subset of the columns of $s$.
%
CVODES provides the following three choices for the sequence in which
the states and sensitivity variables are advanced in time at each step.
%
\begin{itemize}
\item Simultaneous Corrector: the nonlinear system (\ref{NLS}) is solved
  simultaneously for the states and all sensitivity variables~\cite{MaPe:96},
  using a coefficient matrix for the Newton update, which is
  simply the block-diagonal portion of the Newton matrix.
\item Staggered Corrector 1: the correction stages for the sensitivity
  variables take place after the states have been corrected and have passed
  the error test. To prevent frequent Jacobian updates, the linear sensitivity
  systems are solved with a modified Newton iteration~\cite{FTB:97}.
\item Staggered Corrector 2: a variant of the previous one,
  in which the error test for the sensitivity variables is also staggered,
  one sensitivity system at a time.
\end{itemize}
%
The matrices in the {\em staggered corrector} methods and all of the
diagonal blocks in the {\em simultaneous corrector} method are
identical to the matrix $M$ in (\ref{Newtonmat}), and
therefore the linear systems corresponding to the sensitivity equations
are solved using the same preconditioner and/or linear system solver that
were specified for the original ODE problem.
The sensitivity variables may be suppressed from the step size control
algorithm, but they are always included in the nonlinear system convergence
test.

The right-hand side of the sensitivity equations may be supplied by a
user routine, or approximated by difference quotients at the user's option.
In the latter case, CVODES offers both forward and central finite
difference approximations.
%
We use increments that take into account several problem-related features,
namely, the relative ODE error tolerance {\sc rtol},
the machine unit roundoff $U$,
scale factors $\bar p$ for the problem parameters $p$,
and the weighted root-mean-square norm of the sensitivity vector $s_i$.
%
Using central finite differences as an example, the two terms $J s_i$
and ${\partial f}/{\partial p_i}$ in the right-hand side of (\ref{e:fwdODE})
can be evaluated separately:
\begin{gather}
  J s_i \approx \frac{f(t, y+\sigma_y s_i, p)-
    f(t, y-\sigma_y s_i, p)}{2\,\sigma_y} \, , \label{e:fd2} \\
  {\partial f}/{\partial p_i} \approx \frac{f(t,y,p + \sigma_i e_i)-
    f(t,y,p - \sigma_i e_i)}{2\,\sigma_i} \, , \tag{\ref{e:fd2}$'$} \\
  \sigma_i = |{\bar p}_i| \sqrt{\max(\mbox{\sc rtol}, U)} \, , \quad
  \sigma_y = \frac{1}{\max(1/\sigma_i, \|s_i\|_{\mbox{\scriptsize WRMS}} / |{\bar p}_i|)} \, , \nonumber
\end{gather}
simultaneously:
\begin{gather}
  J s_i + {\partial f}/{\partial p_i} \approx
  \frac{f(t, y+\sigma s_i, p + \sigma e_i) -
    f(t, y-\sigma s_i, p - \sigma e_i)}{2\,\sigma} \, , \label{e:dd2} \\
  \sigma = \min(\sigma_i, \sigma_y) \, , \nonumber
\end{gather}
or adaptively switching between (\ref{e:fd2})+(\ref{e:fd2}$'$) and (\ref{e:dd2}),
depending on the relative size of the estimated finite difference
increments $\sigma_i$ and $\sigma_y$.

%------------------------------------------------------------------
\subsubsection{Adjoint Sensitivity}
CVODES can also be used to carry out adjoint sensitivity analysis,
in which the original system for $y$ is integrated forward,
an adjoint system is then integrated backward, and finally the desired
sensitivities are obtained from the backward solution.
To be specific about how the adjoint approach works, we consider
the following situation. We assume as before that $f$ and/or $y_0$
involves the parameter vector $p$ and that there is a
functional $g(t,y,p)$ for which we desire the
total derivative $(dg/dp)|_{t=t_{\mbox{\tiny f}}}$
at the final time $t_{\mbox{\scriptsize f}}$.
We first integrate the original problem (\ref{e:ODE_with_p}) forward
from $t_0$ to $t_{\mbox{\scriptsize f}}$. The next step in the procedure is to integrate
from $t_{\mbox{\scriptsize f}}$ to $t_0$ the adjoint system
\begin{equation}\label{e:ODEadj}
  \dot{\lambda} = -J^T \lambda \, , \quad
  \lambda(t_{\mbox{\scriptsize f}}) = \left. \left( \frac{\partial g}{\partial y}
    \right)^T \right| _{t=t_{\mbox{\tiny f}}} \, .
\end{equation}
When this backward integration is complete, then the desired
sensitivity array is given by
\begin{equation}\label{e:grad_g}
  \left. \frac{dg}{dp}\right|_{t=t_{\mbox{\tiny f}}} =
  \lambda^T(t_0) \frac{dy_0}{dp}
  + \int_{t_0}^{t_{\mbox{\tiny f}}} \lambda^T \frac{\partial f}{\partial p} dt +
  \left. \frac{\partial g}{\partial p} \right|_{t=t_{\mbox{\tiny f}}} \, .
\end{equation}
Other situations, with different forms for the desired sensitivity
information, are covered by different adjoint systems~\cite{CLPS:03}.

For the efficient evaluation of integrals such as the one in (\ref{e:grad_g}),
CVODES allows for special treatment of quadrature equations by excluding them
from the nonlinear system solution, while allowing for inclusion or exclusion
of the corresponding variables from the step size control algorithm.

During the backward integration, we regenerate $y(t)$ values, as needed,
in evaluating the right-hand side of the adjoint system.
%
CVODES settles for a compromise between storage space and execution time
by implementing a checkpoint scheme combined with piecewise cubic Hermite
interpolation: at the cost of, at most, one additional forward integration,
this approach offers the best possible estimate of memory requirements for
adjoint sensitivity analysis.
%
Finally, we note that the adjoint sensitivity module in CVODES provides
the infrastructure to integrate backwards in time any ODE terminal-value
problem dependent on the solution of the IVP (\ref{e:ODE_with_p}),
not just adjoint systems such as (\ref{e:ODEadj}). In particular,
for ODE systems arising from semi-discretization of time-dependent PDEs,
this feature allows for integration of either the discretized adjoint
PDE system or the adjoint of the discretized PDE.

%==========================================================================

\subsection{IDAS}\label{ss:idas}

IDAS, an extension to IDA with sensitivity analysis capabilities, is
currently under development and will be soon released as part of SUNDIALS.

{\em Forward sensitivity analysis} for systems of DAEs system is similar to that
for ODEs.  Writing the system as $F(t,y,\dot{y},p)=0$ and defining
$s={dy}/{dp}$ as before, we obtain DAEs for the individual sensitivity vectors,
%
\begin{equation}\label{e:DAEfwd}
  \frac{\partial F}{\partial y} s_i + \frac{\partial F}{\partial \dot{y}} \dot{s}_i
  + \frac{\partial F}{\partial p_i}  = 0 \, , \quad
  s_i(t_0) = dy_0 / dp_i \, , ~ \dot{s}_i(t_0) = d\dot{y}_0 / dp_i \, .
\end{equation}
%
IDAS implements the same three options for correction of the sensitivity
variables as CVODES. For the simultaneous corrector approach, the coefficient
matrix for the Newton update of the extended system (\ref{e:DAE})+(\ref{e:DAEfwd})
is again approximated by its diagonal blocks, each of them identical to the
matrix $J$ of (\ref{e:DAE_Jacobian}). For the generation of the residuals of the sensitivity
equations, IDAS provides several difference quotient approximations equivalent
to those described in Section~\ref{ss:cvodes}.

The use of adjoint DAE systems for {\em adjoint sensitivity} analysis is also
similar to the ODE case. As an example, if $\lambda$ satisfies
\begin{equation}\label{e:DAEadj}
  \frac{d}{dt} \left[ \left( \frac{\partial F}{\partial \dot{y}}\right) ^T \lambda \right]
  - \left(\frac{\partial F}{\partial y}\right)^T \lambda
  = - \left( \frac{\partial g}{\partial y}\right)^T \, ,
\end{equation}
with appropriate conditions at $t_{\mbox{\scriptsize f}}$, then the total derivative of
$G(p) = \int_{t_0}^{t_{\mbox{\tiny f}}} g(t,y,p) dt$ is obtained as
\begin{equation*}
  \frac{dG}{dp} =
  \int_{t_0}^{t_{\mbox{\tiny f}}} \left(
    \frac{\partial g}{\partial p} - \lambda^T \frac{\partial F}{\partial p}
  \right) dt
  - \left.\left(
      \lambda^T \frac{\partial F}{\partial \dot{y}} s
    \right)\right|_{t_0}^{t_{\mbox{\tiny f}}} \, .
\end{equation*}
However, unlike the ODE case, homogeneous final conditions for the
adjoint variables may not always be enough (such is the case for
Hessenberg index-2 DAEs).
%
Moreover, for implicit ODEs and index-1 DAEs the adjoint system may not
be stable to integration from the right, even if the original system (\ref{e:DAE})
is stable from the left. To circumvent this problem for such
systems, IDAS integrates backwards in time the so-called
{\em augmented adjoint DAE system} defined as
\begin{equation} \label{e:DAEadj_augmented}
\begin{split}
  \dot {\bar\lambda} -  \left(\frac{\partial F}{\partial y}\right)^T \lambda
  &  = - \left( \frac{\partial g}{\partial y}\right)^T \\
  \bar \lambda - \left( \frac{\partial F}{\partial \dot{y}}\right) ^T \lambda
  & = 0 \, ,
\end{split}
\end{equation}
which can be shown to preserve stability~\cite{CLPS:03}.

IDAS employs a combination of checkpointing with piecewise cubic Hermite
interpolation for generation of the solution $y(t)$ needed during the
backward integration phase in (\ref{e:DAEadj}) or (\ref{e:DAEadj_augmented}).
As in CVODES, for efficiency, pure quadrature equations are treated separately
in that their correction phase does not include a nonlinear system solution.
At the user's discretion, the quadrature variables can be included or excluded
from the step size control algorithm.

%==========================================================================

\subsection{KINSOLS}

In the case of a nonlinear algebraic system, the sensitivity
equations are considerably simpler.  If the system is written
$F(u,p) = 0$ and we define $s = d u / d p$, then for the
individual sensitivity vectors $s_i$,
\begin{equation}\label{e:NLSfwd}
  J s_i = -\frac{\partial F}{\partial p_i} \, ,
\end{equation}
where $ J = \partial F / \partial u$.

{\em Forward sensitivity} analysis for nonlinear systems thus reduces to solving a
number of linear systems equal to the number of model parameters.
The Jacobian-vector product and right-hand side of (\ref{e:NLSfwd}) can be provided
by the user or evaluated with directional derivatives. In the latter case
we approximate $J s_i$ with the formulas presented in Section~\ref{ss:kinsol}
and ${\partial F}/{\partial p_i}$ with:
\begin{equation*}
\frac{\partial F}{\partial p_i} \approx \frac{F(u,p + \sigma_i e_i)-
    F(u,p)}{\sigma_i} \, ,
\end{equation*}
where $\sigma_i = |{\bar p}_i| \sqrt{U}$.

When the dimension $N_p$ of the problem parameters $p$ is large, the
{\em adjoint sensitivity} is again a much more efficient method for computing
the total derivative of some functional $g(u,p)$. If $\lambda$ is the solution of
the adjoint system
\begin{equation*}
  J^T \lambda = \left( \frac{\partial g}{\partial u} \right)^T \, ,
\end{equation*}
then the desired gradient becomes
${dg}/{dp} = -\lambda^T ({\partial F}/{\partial p}) + ({\partial g}/{\partial p})$.
